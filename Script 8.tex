
\documentclass[openany, amssymb, psamsfonts]{amsart}
\usepackage{mathrsfs,comment}
\usepackage[usenames,dvipsnames]{color}
\usepackage[normalem]{ulem}
\usepackage{url}
\usepackage{lipsum}
\usepackage[all,arc,2cell]{xy}
\UseAllTwocells
\usepackage{enumerate}
\newcommand{\bA}{\mathbf{A}}
\newcommand{\bB}{\mathbf{B}}
\newcommand{\bC}{\mathbf{C}}
\newcommand{\bD}{\mathbf{D}}
\newcommand{\bE}{\mathbf{E}}
\newcommand{\bF}{\mathbf{F}}
\newcommand{\bG}{\mathbf{G}}
\newcommand{\bH}{\mathbf{H}}
\newcommand{\bI}{\mathbf{I}}
\newcommand{\bJ}{\mathbf{J}}
\newcommand{\bK}{\mathbf{K}}
\newcommand{\bL}{\mathbf{L}}
\newcommand{\bM}{\mathbf{M}}
\newcommand{\bN}{\mathbf{N}}
\newcommand{\bO}{\mathbf{O}}
\newcommand{\bP}{\mathbf{P}}
\newcommand{\bQ}{\mathbf{Q}}
\newcommand{\bR}{\mathbf{R}}
\newcommand{\bS}{\mathbf{S}}
\newcommand{\bT}{\mathbf{T}}
\newcommand{\bU}{\mathbf{U}}
\newcommand{\bV}{\mathbf{V}}
\newcommand{\bW}{\mathbf{W}}
\newcommand{\bX}{\mathbf{X}}
\newcommand{\bY}{\mathbf{Y}}
\newcommand{\bZ}{\mathbf{Z}}

%% blackboard bold math capitals
\newcommand{\bbA}{\mathbb{A}}
\newcommand{\bbB}{\mathbb{B}}
\newcommand{\bbC}{\mathbb{C}}
\newcommand{\bbD}{\mathbb{D}}
\newcommand{\bbE}{\mathbb{E}}
\newcommand{\bbF}{\mathbb{F}}
\newcommand{\bbG}{\mathbb{G}}
\newcommand{\bbH}{\mathbb{H}}
\newcommand{\bbI}{\mathbb{I}}
\newcommand{\bbJ}{\mathbb{J}}
\newcommand{\bbK}{\mathbb{K}}
\newcommand{\bbL}{\mathbb{L}}
\newcommand{\bbM}{\mathbb{M}}
\newcommand{\bbN}{\mathbb{N}}
\newcommand{\bbO}{\mathbb{O}}
\newcommand{\bbP}{\mathbb{P}}
\newcommand{\bbQ}{\mathbb{Q}}
\newcommand{\bbR}{\mathbb{R}}
\newcommand{\bbS}{\mathbb{S}}
\newcommand{\bbT}{\mathbb{T}}
\newcommand{\bbU}{\mathbb{U}}
\newcommand{\bbV}{\mathbb{V}}
\newcommand{\bbW}{\mathbb{W}}
\newcommand{\bbX}{\mathbb{X}}
\newcommand{\bbY}{\mathbb{Y}}
\newcommand{\bbZ}{\mathbb{Z}}

%% script math capitals
\newcommand{\sA}{\mathscr{A}}
\newcommand{\sB}{\mathscr{B}}
\newcommand{\sC}{\mathscr{C}}
\newcommand{\sD}{\mathscr{D}}
\newcommand{\sE}{\mathscr{E}}
\newcommand{\sF}{\mathscr{F}}
\newcommand{\sG}{\mathscr{G}}
\newcommand{\sH}{\mathscr{H}}
\newcommand{\sI}{\mathscr{I}}
\newcommand{\sJ}{\mathscr{J}}
\newcommand{\sK}{\mathscr{K}}
\newcommand{\sL}{\mathscr{L}}
\newcommand{\sM}{\mathscr{M}}
\newcommand{\sN}{\mathscr{N}}
\newcommand{\sO}{\mathscr{O}}
\newcommand{\sP}{\mathscr{P}}
\newcommand{\sQ}{\mathscr{Q}}
\newcommand{\sR}{\mathscr{R}}
\newcommand{\sS}{\mathscr{S}}
\newcommand{\sT}{\mathscr{T}}
\newcommand{\sU}{\mathscr{U}}
\newcommand{\sV}{\mathscr{V}}
\newcommand{\sW}{\mathscr{W}}
\newcommand{\sX}{\mathscr{X}}
\newcommand{\sY}{\mathscr{Y}}
\newcommand{\sZ}{\mathscr{Z}}


\renewcommand{\phi}{\varphi}
\renewcommand{\emptyset}{\O}

\newcommand{\abs}[1]{\lvert #1 \rvert}
\newcommand{\norm}[1]{\lVert #1 \rVert}
\newcommand{\sm}{\setminus}


\newcommand{\sarr}{\rightarrow}
\newcommand{\arr}{\longrightarrow}

\newcommand{\hide}[1]{{\color{red} #1}} % for instructor version
%\newcommand{\hide}[1]{} % for student version
\newcommand{\com}[1]{{\color{blue} #1}} % for instructor version
%\newcommand{\com}[1]{} % for student version
\newcommand{\meta}[1]{{\color{green} #1}} % for making notes about the script that are not intended to end up in the script
%\newcommand{\meta}[1]{} % for removing meta comments in the script

\DeclareMathOperator{\ext}{ext}
\DeclareMathOperator{\ho}{hole}
%%% hyperref stuff is taken from AGT style file
\usepackage{hyperref}  
\hypersetup{%
  bookmarksnumbered=true,%
  bookmarks=true,%
  colorlinks=true,%
  linkcolor=blue,%
  citecolor=blue,%
  filecolor=blue,%
  menucolor=blue,%
  pagecolor=blue,%
  urlcolor=blue,%
  pdfnewwindow=true,%
  pdfstartview=FitBH}   
  
\let\fullref\autoref
%
%  \autoref is very crude.  It uses counters to distinguish environments
%  so that if say {lemma} uses the {theorem} counter, then autrorefs
%  which should come out Lemma X.Y in fact come out Theorem X.Y.  To
%  correct this give each its own counter eg:
%                 \newtheorem{theorem}{Theorem}[section]
%                 \newtheorem{lemma}{Lemma}[section]
%  and then equate the counters by commands like:
%                 \makeatletter
%                   \let\c@lemma\c@theorem
%                  \makeatother
%
%  To work correctly the environment name must have a corrresponding 
%  \XXXautorefname defined.  The following command does the job:
%
\def\makeautorefname#1#2{\expandafter\def\csname#1autorefname\endcsname{#2}}
%
%  Some standard autorefnames.  If the environment name for an autoref 
%  you need is not listed below, add a similar line to your TeX file:
%  
%\makeautorefname{equation}{Equation}%
\def\equationautorefname~#1\null{(#1)\null}
\makeautorefname{footnote}{footnote}%
\makeautorefname{item}{item}%
\makeautorefname{figure}{Figure}%
\makeautorefname{table}{Table}%
\makeautorefname{part}{Part}%
\makeautorefname{appendix}{Appendix}%
\makeautorefname{chapter}{Chapter}%
\makeautorefname{section}{Section}%
\makeautorefname{subsection}{Section}%
\makeautorefname{subsubsection}{Section}%
\makeautorefname{theorem}{Theorem}%
\makeautorefname{thm}{Theorem}%
\makeautorefname{excercise}{Exercise}%
\makeautorefname{cor}{Corollary}%
\makeautorefname{lem}{Lemma}%
\makeautorefname{prop}{Proposition}%
\makeautorefname{pro}{Property}
\makeautorefname{conj}{Conjecture}%
\makeautorefname{defn}{Definition}%
\makeautorefname{notn}{Notation}
\makeautorefname{notns}{Notations}
\makeautorefname{rem}{Remark}%
\makeautorefname{quest}{Question}%
\makeautorefname{exmp}{Example}%
\makeautorefname{ax}{Axiom}%
\makeautorefname{claim}{Claim}%
\makeautorefname{ass}{Assumption}%
\makeautorefname{asss}{Assumptions}%
\makeautorefname{con}{Construction}%
\makeautorefname{prob}{Problem}%
\makeautorefname{warn}{Warning}%
\makeautorefname{obs}{Observation}%
\makeautorefname{conv}{Convention}%


%
%                  *** End of hyperref stuff ***

%theoremstyle{plain} --- default
\newtheorem{thm}{Theorem}[section]
\newtheorem{cor}{Corollary}[section]
\newtheorem{exercise}{Exercise}
\newtheorem{prop}{Proposition}[section]
\newtheorem{lem}{Lemma}[section]
\newtheorem{prob}{Problem}[section]
\newtheorem{conj}{Conjecture}[section]
%\newtheorem{ass}{Assumption}[section]
%\newtheorem{asses}{Assumptions}[section]

\theoremstyle{definition}
\newtheorem{defn}{Definition}[section]
\newtheorem{ass}{Assumption}[section]
\newtheorem{asss}{Assumptions}[section]
\newtheorem{ax}{Axiom}[section]
\newtheorem{con}{Construction}[section]
\newtheorem{exmp}{Example}[section]
\newtheorem{notn}{Notation}[section]
\newtheorem{notns}{Notations}[section]
\newtheorem{pro}{Property}[section]
\newtheorem{quest}{Question}[section]
\newtheorem{rem}{Remark}[section]
\newtheorem{warn}{Warning}[section]
\newtheorem{sch}{Scholium}[section]
\newtheorem{obs}{Observation}[section]
\newtheorem{conv}{Convention}[section]

%%%% hack to get fullref working correctly
\makeatletter
\let\c@obs=\c@thm
\let\c@cor=\c@thm
\let\c@prop=\c@thm
\let\c@lem=\c@thm
\let\c@prob=\c@thm
\let\c@con=\c@thm
\let\c@conj=\c@thm
\let\c@defn=\c@thm
\let\c@notn=\c@thm
\let\c@notns=\c@thm
\let\c@exmp=\c@thm
\let\c@ax=\c@thm
\let\c@pro=\c@thm
\let\c@ass=\c@thm
\let\c@warn=\c@thm
\let\c@rem=\c@thm
\let\c@sch=\c@thm
\let\c@equation\c@thm
\numberwithin{equation}{section}
\makeatother

\bibliographystyle{plain}

%--------Meta Data: Fill in your info------
\title{University of Chicago Calculus IBL Course}

\author{Agustin Esteva}

\date{Jan 30. 2024}

\begin{document}

\begin{abstract}

16210's Script 8.\\ Let me know if you see any errors! Contact me at aesteva@uchicago.edu.


\end{abstract}

\maketitle

\tableofcontents

\setcounter{section}{8}
Now that we have constructed $\bbR$ and proved the fundamental facts about it, we will  work with the real numbers $\bbR$ instead of an arbitrary continuum $C$. We will leave behind Dedekind cuts and think of elements of $\bbR$ as numbers. Accordingly,  from now on we will use lower-case letters like $x$ for real numbers and will write $+$ and $\cdot$ for $\oplus$ and $\otimes$. We will also now use the standard notation $(a, b)$ for the region $\underline{ab} = \{x \in \bbR \mid a < x < b\}$. Even though the notation is the same, this is \emph{not} the same object as the ordered pair $(a, b)$.

More generally, we adopt the following standard notation:
\begin{eqnarray}
(a, b) & = & \{x\in \bbR\mid a<x<b\} \nonumber \\
\left[a,b\right)  & = & \{x\in \bbR\mid a\leq x<b\} \nonumber\\
\left(a,b\right] & =& \{x\in \bbR\mid a<x\leq b\}  \nonumber  \\
\left[a,b\right] & =& \{x\in \bbR\mid a\leq x\leq b\}  \nonumber\\
\left(a,\infty\right) & =& \{x\in \bbR\mid a<x\} \nonumber  \\
\left[a,\infty\right) & =& \{x\in \bbR\mid a\leq x\} \nonumber  \\
\left(-\infty,b\right) & =& \{x\in \bbR\mid x<b\} \nonumber  \\
\left(-\infty, b\right] & = & \{x\in \bbR\mid x\leq b\} \nonumber .
\end{eqnarray}
Notice that if $a=b,$ then $(a,b), [a,b)$ and $(a,b]$ are all empty, but $[a,b]=\{a\}.$ We call $[a,a]=\{a\}$ a ``degenerate'' interval.


\subsection*{Example 8.1}
\begin{exmp}
\label{8.1}

\end{exmp}
\vspace{4pt}     \hrule   \vspace{4pt}
\begin{proof}:\\
\begin{enumerate}
\item $(a,b)$ is a region, then by corollary 4.10, it is open. Thus, by Theorem 5.1, it is not closed. 
\item $\left[a,b\right)$ is neither closed nor open: It is not closed since $b$ is a $LP(\left[a,b\right)$ (Corollary 5.13), and $b\notin \left[a,b\right)$. It is not open because since $a$ is the first point of $\left[a,b\right)$, then for any region $(a',z)$ containing $a$, $a'\notin \left[a,b\right)$ since $a'<a<z$. Because $\bbR$ is connected, there exists some $x\in \bbR$ such that $a'<x<a$. Therefore, $x\notin \left[a,b\right)$ and so $R\not\subset \left[a,b\right)$. Therefore, any region containing $a$ is not a subset of $\left[a,b\right)$, therefore, $\left[a,b\right)$ is not open.
\item $\left(a,b\right]$ is not closed or open by the same reasoning as above.
\item $\left[a,b\right]$ is closed by Corollary 5.14. By Theorem 5.10, it is not open. 
\item $\left(a,\infty\right)$ is open by Corollary 4.12. Thus, it is not closed.
\item $\left(-\infty,b\right]$ is closed since its complement, $\left(a,\infty\right)$ is open (Definition 4.7). Thus, is is not open.
\item $\left(-\infty,b\right)$ is open by Corollary 4.12. Thus, is it not closed
\item $\left[a,\infty\right)$ is closed since its complement, $\left(-\infty,b\right)$ is open (Definition 4.7). Thus, it is not open. 
\end{enumerate}
\end{proof} \vspace{4pt}     \hrule   \vspace{4pt}

\subsection*{Definition 8.2}
\begin{defn}
\label{8.2}
A nonempty set $I\subset \bbR$ is an {\em interval} if, for all $x,y\in I,$ with $x<y,$ $[x,y]\subset I.$
\end{defn}

\subsection*{Lemma 8.3}
\begin{lem}
A nonempty proper set $I\subset \bbR$ is an {\em interval} if, and only if, it takes one of the eight forms above.
\end{lem}
\vspace{4pt}     \hrule   \vspace{4pt}
\begin{proof}:\\
\begin{enumerate}
\item If it takes one of the eight forms above, then it is an interval:
\begin{enumerate}
\item For all $x,y\in (a,b)$ where $x<y$, $a<x<y<b$, and therefore it follows that for all $z\in [x,y]$, $a<x\leq z \leq y<b$, so $z\in (a,b)$ Thus, $[x,y] \subset (a,b)$.
\item Using the same method as above, the result holds for $(a,\infty)$, and $(-\infty, b)$
\item For all $x,y \in [a,b)$, where $x<y$, $a\leq x <y <b$ and therefore it follows that for all $z\in [x,y]$, $a\leq x\leq z \leq y<b$, so $z\in [a,b)$. Thus, $[x,y] \subset [a,b)$.
\item Using the same method as above, the result holds for $(a,b]$, $[a,\infty)$, and $(-\infty, b]$
\item For all $x,y \in [a,b]$, where $x<y$, $a\leq x <y \leq b$ and therefore it follows that for all $z\in [x,y]$, $a\leq x\leq z \leq y\leq b$, so $z\in [a,b]$. Thus, $[x,y] \subset [a,b]$. 
\end{enumerate}
\item If I is an interval where $I$ is a proper subset of $\bbR$, then it takes one of the eight forms above:\\\\Because $I$ is an interval then since $I \neq \emptyset$, then there exists an $x\in I$. Let $a = \inf I$ and $b = \sup I$.
\begin{enumerate}
\item If $b$ does not exist, then $I$ is not bounded above (Theorem 5.16). Therefore, for all $n\in I$ where $x\leq n$, there exist some $m\in I$ such that $n<m$ (5.5). Thus, since $x<m$, then $[x,m]\subset I$ (\ref{8.2}). Thus, for all $n\in \bbR$, where $x\leq n$, then $n\in I$, and so $[x,\infty)\subset  I$.
\item If $a$ does not exist, then by identical logic, $(-\infty, x]\subset I$
\item If $b\in I$ then because $x\leq b$, then $[x,b]\subset I$ (8.2). \\For all $n\in \bbR$ where $b<n$, then $n\notin I$. Thus, $(b,\infty) \subset \bbR \sm I$.
\item $If a \in I$, by identical logic, $[a,x]\subset I$ and $(-\infty, a) \subset \bbR \sm I$.
\item If $b\notin I$, then because $I \neq \emptyset$ and $b$ exists, then $b \in LP(I)$. Consider some $p\in \bbR$ such that $x<p< b$. Thus, by 5.10, there exists some $q\in I$ such that $q\leq b$ and thus $[x,q] \subset I$. Therefore, for all $p\in \bbR$, where $x\leq p < q$, $p\in I$. Thus, $[x,b)\subset I$. Similar to above, if $n\in \bbR$ and $b\leq n$, then $n\notin I$ and so $[b,\infty)\subset \bbR\sm I$
\item If $a\notin I$, , by identical logic, $(a,x]\subset I$ and $(-\infty, a] \subset \bbR \sm I$. 
\end{enumerate}
Thus, there are 9 possible combinations: 
\begin{enumerate}
\item Because $(a,x] \subset I$ and $[x,b)\subset I$ and $(-\infty,a]\not\subset I$ and $[b,\infty)\not\subset I$, then $(a,b)= I$.
\item Because $(-\infty, a)\not\subset I$, $[a,x]\subset I$, $[x,b]\subset I$, and $(b,\infty) \not\subset I$, then $[a,b]= I$.
\item By similar logic, $[a,b)= I$.
\item By similar logic, $(a,b]= I$.
\item Since $[a,x]\subset I$ and $[x,\infty]\subset I$ and $(-\infty, a)\not\subset I$, then $[a,\infty)= I$
\item By similar logic, $(a,\infty) = I$
\item By similar logic, $(-\infty, b)= I$
\item By similar logic, $(-\infty, b]= I$
\item Since $(-\infty, x]\subset I$ and $[x,\infty] \subset I$, then $(-\infty, \infty )= I$. Thus, $I = \bbR$ and therefore is not a proper subset of $\bbR$, which is a contradiction. 
\end{enumerate}
Thus, if $I$ is an interval, then it takes one of the eight forms above.
\end{enumerate}
\end{proof} \vspace{4pt}     \hrule   \vspace{4pt}

\subsection*{Definition 8.4}
\begin{defn}
\label{8.4}
	The \emph{absolute value} of a real number $x$ is the non-negative number $\abs{x}$ defined by
	\[
		\abs{x} = \begin{cases}
			x \quad &\text{if $x \geq 0$,} \\
			-x \quad &\text{if $x < 0$.} 
		\end{cases}
	\]
\end{defn}

\subsection*{Example 8.5}
\begin{exmp}
\label{8.5}
Show that $\abs{x}=\abs{-x},$ for $x\in \bbR.$  (Note that this also means that $\abs{x-y}=\abs{y-x}$ for any $x,y\in \bbR.$ )
\end{exmp}
\vspace{4pt}     \hrule   \vspace{4pt}
\begin{proof}:\\
\begin{enumerate}
    \item If $x>0$, where $x\in \bbR$, then by Definition \ref{8.4}, $\abs{x} = x$. Moreover, since $x> 0$, then by Lemma 7.23, $-x<0$. Thus, since $-x<0$, then $\abs{-x} = -(-x)$. By Corollary 7.11, $-(-x) = x$. Therefore, $\abs{-x} = x$. Therefore, $x = |-x| = |x|$.
    \item If $x< 0$, where $x\in \bbR$, then by Definition \ref{8.4}, $\abs{x} = -x$. Moreover, by Lemma 7.23, $0<-x$. Therefore, by Definition \ref{8.4}, $\abs{-x}=-x$. Thus, $-x = |-x| = |x|$
    \item If $x=0$, then by Definition \ref{8.4}, since $x\geq 0$, then $\abs{x} = x$. Moreover, since $-x = -1(x) = -1 \cdot 0 = 0$ (Lemma 7.18 and Theorem 7.13), then by Definition \ref{8.4}, since $-x\geq 0$, then $\abs{-x} = x$. Therefore, $x = |x| = |-x|$.
\end{enumerate}
Thus, for any $x\in \bbR$, $\abs{x}=\abs{-x}$
\end{proof} \vspace{4pt}     \hrule   \vspace{4pt}

\subsection*{Definition 8.6}
\begin{defn} 
\label{8.6}
	\label{defn:distance}
	The \emph{distance} between $x \in \bbR$ and $y \in \bbR$ is defined to be $\abs{x - y}$. 
\end{defn}

\subsection*{Remark 8.7}
\begin{rem}
It follows from Definition \ref{defn:distance} that $|x|$ is the distance between $x$ and $0.$
\end{rem}

\subsection*{Example 8.8}
\begin{exmp} \label{8.8} Prove that, if $x,y$ are real numbers then
\begin{enumerate}
\item[a)] $|xy|=|x|\cdot |y|$
\vspace{4pt}     \hrule   \vspace{4pt} \begin{proof} :\\
\begin{enumerate}
    \item If $x,y >0$, then by Definition 7.21, $0<xy$. Thus, by Definition \ref{8.4}, $\abs{xy} = xy = x\cdot y$. Moreover, since $x>0$, $\abs{x} = x$ and similarly, $\abs{y} = y$. Thus, $\abs{x}\cdot \abs{y} = x\cdot y$. 
    \item If $x,y<0$, then by Remark 7.25, $0<xy$. Thus, by Definition \ref{8.4}, $\abs{xy} = xy = x\cdot y$. Moreover, since $x<0$, then $\abs{x} = -x$ and similarly, $\abs{y} = -y$. Thus, $\abs{x}\cdot \abs{y} = -x\cdot -y$, so by Lemma 7.20, $\abs{x}\cdot \abs{y} = x\cdot y$.
    \item If $x<0$ and $0<y$, then by Definition 7.24, $xy<0$. Thus, by Definition \ref{8.4}, $\abs{xy} = -(xy)$ Therefore, by Lemma 7.19, $\abs{xy} = -x\cdot y$. Moreover, since $x<0$, then $\abs{x} = -x$ and similarly, since $0<y$, then $\abs{y} = y$. Thus, $\abs{x}\cdot \abs{y} = -x\cdot y$.
    \item If $x>0$ and $y<0$, then it can be proven using the method above that $\abs{xy} = \abs{x}\cdot \abs{y}$
    \item  If $x=0$ or $y=0$, then by Theorem 7.13, $xy = 0$. It follows by Definition \ref{8.4} that $\abs{xy}=xy = 0$. WLOG, let $x=0$. Therefore, since $\abs{x}=x=0$, then by Theorem 7.13, $\abs{x}\cdot \abs{y} = 0\cdot \abs{y} = 0$. 
\end{enumerate}
Therefore, for any $x,y\in \bbR$, $\abs{xy} = \abs{x}\cdot \abs{y}$.
\end{proof} \vspace{4pt}     \hrule   \vspace{4pt}
\item[b)] $x\leq |x|$ and $-x\leq |x|.$
\vspace{4pt}     \hrule   \vspace{4pt}
\begin{proof} :\\
\begin{enumerate}
    \item If $0<x$, then by Definition \ref{8.4}, $\abs{x}=x$. Moreover, by Lemma 7.23, $-x<0$. Thus, by transitivity, $-x<x$ so therefore, $-x< \abs{x}$ and $x=\abs{x}$.
    \item If $x<0$, then by Definition \ref{8.4}, $\abs{x} = -x$. Moreover, by Lemma 7.23, $0<-x$. Therefore, by transitivity $x<-x$ so therefore $x<\abs{x}$ and $-x = \abs{x}$.
    \item If $x=0$, then as proved in example \ref{8.5}, $\abs{x}=x=0$. Moreover, since $-x = 0$, then $\abs{x} = -x$.
\end{enumerate}
Therefore, for any $x\in \bbR$, $x\leq \abs{x}$ and $-x\leq \abs{x}$
\end{proof} \vspace{4pt}     \hrule   \vspace{4pt}
\end{enumerate}
\end{exmp}

\subsection*{Lemma 8.9}
\begin{lem}
	For any real numbers $x$, $y$, and $z$, we have 
		\begin{enumerate}[(a)]
		\item \quad $\abs{x + y} \leq \abs{x} + \abs{y}$,
\vspace{4pt}     \hrule   \vspace{4pt}
\begin{proof} :\\
\begin{enumerate}
    \item If $x,y>0$, then $x+y>0$. Thus, by Definition \ref{8.4}, $\abs{x+y}= x+y$. Thus, since $\abs{x} = x$ and $\abs{y}=y$, then $\abs{x}+\abs{y} = x+y$. Thus, $\abs{x+y}=\abs{x}+\abs{y}$.
    \item If $x,y<0$, then $x+y <0$. Thus, $\abs{x+y} = -(x+y) = -x+(-y)$. Thus, because $|y| = -y$ and $|x| = -x$, then $|x|+|y| = -x + (-y)$.
    \item If $x>0$ and $y<0$, then:
    \begin{enumerate}
        \item If $x+y>0$, then by Definition \ref{8.4}, $\abs{x+y}= x+y$. Thus, since $\abs{x} = x$ and $\abs{y}=-y$, then $\abs{x}+\abs{y} = x+-y$. Thus, since $y<-y$, then $x+y<x+(-y)$ and so $\abs{x+y} < \abs{x} + \abs{y}$
        \item If $x+y<0$, then by Definition \ref{8.4}, $\abs{x+y}= -(x+y)=-x+(-y)$. Thus, since $\abs{x} = x$ and $\abs{y}=-y$, then $\abs{x}+\abs{y} = x+(-y)$. Thus, since $-x<x$, then $-x+(-y)<x+(-y)$, and so $\abs{x+y} < \abs{x} + \abs{y}$
        \item If $x+y = 0$, then $\abs{x+y} = x+y = 0$. Thus, since $\abs{x}=x$ and $\abs{y}=-y$, then since $y<-y$, then $x+y < x+ (-y)$, and so $\abs{x+y} < \abs{x} + \abs{y}$
    \end{enumerate}
    \item If $x<0$ and $y>0$, then by using the same process as above, $\abs{x+y}<\abs{x}+\abs{y}$
    \item If $x=0$ and $y\neq 0$, then $x+y = y$. Therefore, since $\abs{x+y} = \abs{y}$ and $\abs{x}=0$, then $\abs{x+y} = \abs{x}+\abs{y}=\abs{y}$.
    \item If $x\neq 0$ and $y=0$, then process is the same as above.
\end{enumerate}
Therefore, for all $x,y\in \bbR$, $\abs{x+y}\leq \abs{x}+\abs{y}$
\end{proof} \vspace{4pt}     \hrule   \vspace{4pt}
		\item \quad $\abs{x - z} \leq \abs{x - y} + \abs{y - z}$, and
\vspace{4pt}     \hrule   \vspace{4pt}
\begin{proof} :\\
By part a, if $a,b\in \bbR$, then $\abs{a+b}\leq \abs{a}+\abs{b}$. Therefore, because $a\in \bbR$, then let $x-y = a$. Because $b\in \bbR$, then let $y-z = b$. Thus, $a+b = (x-y) + (y-z)$ and by the associative property, $a+b = (x+(-z)) + (y+(-y))$. Thus, by the additive inverse $a+b = x-z$. Therefore, $\abs{a+b}\leq \abs{a}+\abs{b}$ can be rewritten as  $\abs{x-z}\leq \abs{x-y}+\abs{y-z}$
\end{proof} \vspace{4pt}     \hrule   \vspace{4pt}
		\item \quad $\abs{\abs{x} - \abs{y}} \leq \abs{x-y}$.
\vspace{4pt}     \hrule   \vspace{4pt}
\begin{proof} :\\
Because $\abs{a+y} \leq \abs{a}+\abs{y}$, then, $\abs{a+y} - \abs{y} \leq \abs{a}$. Therefore, let $a= x-y$. Thus, $\abs{x-y+y} - \abs{y} \leq \abs{x-y}$. It follows that $\abs{x} - \abs{y} \leq \abs{x-y}$.
\begin{enumerate}
    \item If $\abs{x} - \abs{y} <0$, then  $\abs{\abs{x} - \abs{y}} = -(\abs{x} - \abs{y})$. Therefore, $\abs{\abs{x} - \abs{y}} = \abs{y}-\abs{x}$. By the same logic as above, $\abs{y}-\abs{x} \leq \abs{y-x}$. Therefore, by Example \ref{8.5}, since $\abs{y-x}=\abs{x0y}$, then $\abs{y}-\abs{x} \leq \abs{x-y}$ and so $\abs{\abs{x} - \abs{y}}\leq \abs{x+y}$.
    \item If $\abs{x} - \abs{y} \geq 0$, then $\abs{\abs{x} - \abs{y}} = \abs{x} - \abs{y}$. Thus, $\abs{\abs{x} - \abs{y}}\leq \abs{x+y}$
\end{enumerate}

\end{proof} \vspace{4pt}     \hrule   \vspace{4pt}
	\end{enumerate}
\end{lem}
\subsection*{Example 8.10}
\begin{exmp}
\label{8.10}
	Let $a, \delta \in \bbR$ with $\delta > 0$. Prove that
	\[
	(a - \delta, a + \delta) = \{x \in \mathbb{R} \mid \abs{x - a} < \delta \}.
	\]
\end{exmp}

\vspace{4pt}     \hrule   \vspace{4pt}
\begin{proof} :\\
For all $x\in (a - \delta, a + \delta)$:
\begin{enumerate}
    \item If $x\geq a$, then $x-a \geq 0$. Therefore, $x-a = \abs{x-a}$. Since $x<a+ \delta$, It follows that $x+(-a)<a+\delta+(-a)$, so by associativity and the additive identity, $x-a<\delta$. Therefore, $\abs{x-a} < \delta$. 
    \item If $x<a$, then $a-x>0$. Thus, $a-x = \abs{a-x}$.  since, $a- \delta<x$, then by the same process as above, $a-x  < \delta$. Therefore, $\abs{a-x} < \delta$ and by Example \ref{8.5}, since $\abs{a-x} = \abs{x-a}$, then $\abs{x-a} < \delta$
\end{enumerate}
Thus, for all $x\in (a - \delta, a + \delta)$, $x\in \bbR$ and $\abs{x-a}<\delta$. Thus, $(a - \delta, a + \delta) \subset \{x \in \mathbb{R} \mid \abs{x - a} < \delta \}$.\\
For all $x\in \bbR$ such that $\abs{x-a}<\delta$, then:
\begin{enumerate}
    \item If for some $x\in \abs{x-a}<\delta$, $x-a \geq 0$, then $\abs{x-a} = x-a$. Therefore, $x-a < \delta$, and therefore $x<a+\delta$. Moreover, since $x-a\geq 0$, then $x\geq a$ and so $a\leq x$. Thus, because $\delta >0$, then $a-\delta < a \leq x$. Thus, $a-\delta <x<a+\delta$
    \item If for some $x\in \abs{x-a}<\delta$, $x-a <0$, then since $a-x >0$, then $\abs{a-x} = a-x$. Therefore, since $\abs{a-x} = \abs{x-a}$, then $a-x < \delta$, and therefore $a-\delta<x$. Moreover, since $x-a\leq 0$, then $a\geq x$. Thus, $a+\delta > a \geq x$. Therefore, $a-\delta < x < a+\delta$.
\end{enumerate}
Therefore, for all $x\in \bbR$ such that $\abs{x-a}<\delta$, then $a-\delta < x < a+\delta$ and therefore, $\{x \in \mathbb{R} \mid \abs{x - a} < \delta \} \subset (a - \delta, a + \delta)$\\
Therefore, $(a - \delta, a + \delta) = \{x \in \mathbb{R} \mid \abs{x - a} < \delta \}.$
\end{proof} \vspace{4pt}     \hrule   \vspace{4pt}

\subsection*{Lemma 8.11}
\begin{lem}
\label{8.11}
	Let $I$ be an open set containing the point $p \in \bbR$. Then
\begin{enumerate}
\item[i)] there exists a number $\delta > 0$ such that $(p - \delta, p + \delta) \subset I$
\vspace{4pt}     \hrule   \vspace{4pt}
\begin{proof} :\\
Since $I$ is open, then because $p\in I$, there exists a region $(a,b)$ containing $p$ such that $(a,b)\subset I$:
\begin{enumerate}
    \item If $b-p<p-a$, then since $\bbR$ is connected, there must exist some $x \in \bbR$ such that $a<p<x <b$. Because $x\in \bbR$, then let $x= p+\delta$. Therefore, since $(p+\delta) -p < b-p$, it follows that $p+\delta<b$. Since $b-p < p - a$, then $(p+\delta) -p < p-a$ and so $(-p+\delta) <-a$ which means that $a<p-\delta<p$. Therefore, there exists some $\delta \in I$ such that $(p-\delta, p+\delta) \subset (a,b) \subset I$
    \item If $p-a<b-p$, then using the same logic, there will exist some $\delta \in I$ such that $(a-\delta, a+\delta) \subset (a,b) \subset I$.
\end{enumerate}
\end{proof}\vspace{4pt}     \hrule   \vspace{4pt} 
\item[ii)] there exists a natural number $N$ such that for all natural numbers $k \geq N$ we have $(p - \frac1k, p + \frac1k) \subset I$.
\vspace{4pt}     \hrule   \vspace{4pt}  \begin{proof} :\\
\begin{enumerate}
    \item Since $p<p+\delta$, then $0<\delta$, so $\delta \in \bbR^+$. Thus by Corollary 6.12, there exists some $\frac{1}{n}$, where $n\in \bbN$, such that $\frac{1}{n}<\delta$. Assume, for the sake of contradiction, that $\frac{1}{n}\leq 0$.Therefore, by multiplying by $n$, $\frac{1}{n}\cdot n \leq 0\cdot n$, and thus $1\leq 0$, which is a contradiction 7.27. Thus, $0<\frac{1}{n}<\delta$. It follows that $0+p<p+\frac{1}{n}<p+\delta$. Thus $a<p< p+\frac{1}{n}<p+\delta<b$, where $(a,b)\subset I$.
    \item Because $p<p+\frac{1}{n}$, then by adding $-\frac{1}{n}$, $p-\frac{1}{n}<p$. Moreover, since $\frac{1}{n}<\delta$, then $-\delta<-\frac{1}{n}<0$, so therefore $p-\delta<p-\frac{1}{n}<p$. Thus, by part $i$, $a<p-\delta<p-\frac{1}{n}<p$. 
\end{enumerate} 
Because there exists some $\frac{1}{n}$ such that $a<p-\frac{1}{n}<p<p+\frac{1}{n}<b$, then if $k\geq n$, then by multiplying the expression by $\frac{1}{n}\frac{1}{k}$, then $\frac{1}{n}\geq \frac{1}{k}$. Thus $p+ \frac{1}{k}\leq p+\frac{1}{n}$. Moreover, since $\frac{1}{k}\leq \frac{1}{n}$, then by the same logic as above, $\frac{-1}{n}\leq \frac{-1}{k}$ and so $p+\frac{-1}{n}\leq p+\frac{-1}{k}$. Thus, by transitivity, because $a<p-\frac{1}{k}<p<p+\frac{1}{k}<b$, then for all $k\geq n$, $(p-\frac{1}{k}, p+ \frac{1}{k}) \subset (a,b) \subset I$
\end{proof}\vspace{4pt}     \hrule   \vspace{4pt} 
\end{enumerate}
\end{lem}

We now discuss the relationship between connectedness and intervals. First however we need to introduce the subspace topology.

\subsection*{Definition 8.12}
\begin{defn} Let $A\subset X\subset \bbR$. 
\label{8.12}
We say that $A$ is {\em  open in $X$} 
if it is the intersection of $X$ with an open set,
and {\em closed in $X$} if it is the intersection of $X$ with a closed set. (This is called the subspace topology on $X$).
\end{defn}

\subsection*{Remark 8.13}
\begin{rem}\label{8.13}
$A\subset \bbR$ open, as defined in Script 3, is equivalent to $A$  open in $\bbR$.
\end{rem}

\subsection*{Example 8.14}
\begin{exmp} \label{8.14} Let $A\subset X\subset \bbR$. Show that $X\setminus A$ is closed in $X$ if, and only if, $A$ is open in $X$.
\end{exmp}
\vspace{4pt}     \hrule   \vspace{4pt} \begin {proof}:\\
\begin{enumerate}
    \item If $X\sm A$ is closed in $X$, then by Definition \ref{8.12}, there must exist some $B\in \bbR$ such that $B \cap X = X\sm A$ and $B$ is closed. Because $B$ is closed, then by Definition 4.7, $\bbR\sm B$ is open. WTS:$(\bbR \sm B) \cap X = A$:
    \begin{enumerate}
        \item If $x\in A$, then because $X = A \cup (X\sm A)$, then $x\in X$. If $x\in A$, then assume $x\in X\sm A$. Then because because $X\sm A \subset B$, then $x\in B$. However, since $A \cap B = \emptyset$, then $x\notin B$. Thus, $x\in \bbR\sm B$. Therefore $A \subset ((\bbR \sm B)\cap X)$
        \item If $x= \bbR \sm B$, then $x\notin B$ and $x\notin X\sm A$. Thus, for all $x\in (\bbR \sm B) \cap X$, $x\notin X\sm A$ and $x\in X$. Therefore, $x\in A$ and so $((\bbR \sm B)\cap X) \subset A$
    \end{enumerate}
Thus, $(\bbR \sm B) \cap X = A$. Because $\bbR \sm B$ is open, then by Definition \ref{8.12}, $A$ is open in $X$.
    \item If $A$ is open in $X$, then by a very similar method as above, $X\sm A$ is closed in $X$.
\end{enumerate}

\end{proof}\vspace{4pt}     \hrule   \vspace{4pt} 


\subsection*{Example 8.15}
\begin{exmp}
\renewcommand{\theenumi}{\alph{enumi}}:\\
\begin{enumerate}
\item[a)] Let $[a,b]\subset \bbR$. Give an example of a set $A\subset [a,b]$ such that $A$ is open in $[a,b]$ but not in $\bbR$.
\vspace{4pt}     \hrule   \vspace{4pt} \begin {proof}:\\
If $X = [a,b]$ and $Y = (x,y)$, where $a<x<b<y$, then $X\cap Y = A = (x,b]$. Because for all $z\in A$, $z\in X$, then $A\subset X$. By Example \ref{8.1}, $[x,b)$ is not open in $\bbR$. Moreover, since $Y$ is open, then $A$ is open in $X$.
\end{proof}\vspace{4pt}     \hrule   \vspace{4pt} 

\item[b)] Give an example of sets $A\subset X\subset \bbR$ such that $A$ is closed in $X$ but not in $\bbR$. 
\vspace{4pt}     \hrule   \vspace{4pt} \begin {proof}:\\
If $X = (a,b)$ and $Y = [x,y]$, where $a<x<b<y$, then $X\cap Y = A = [x,b)$. Because for all $z\in A$, $z\in X$, then $A\subset X$. By Example \ref{8.1}, $[x,b)$ is not closed in $\bbR$. Moreover, since $Y$ is closed, then $A$ is closed in $X$.
\end{proof}\vspace{4pt}     \hrule   \vspace{4pt} 
\end{enumerate}
\end{exmp}

\subsection*{Theorem 8.16}
\begin{thm}
Let $X\subset \bbR$. Then $X$ is connected if, and only if, $X$ is an interval.
\end{thm}
\vspace{4pt}     \hrule   \vspace{4pt} 
\begin{proof}:\\
    \begin{itemize}
        \item If $X$ is an interval, then assume, for the sake of contradiction, that it is not connected. Thus, it follows from Definition 4.23 that $X = A\cup B$, where $A,B$ are non-empty, disjoint, and open sets in $X$. Let $a\in A$ and $b\in B$, where WLOG, $a<b$. Consider the set $Y = \{x<b|x\in A\}$. 
        \begin{enumerate}
            \item Because $a\in A$ and $a<b$, then $a\in Y$ and so $Y\neq \emptyset$.
            \item Because for all $x\in Y$, $x<b$, then $b$ is an upper bound.
        \end{enumerate}
        Thus by Lemma 6.6, since $X\subset \bbR$, then $s=\sup Y$ exists.         \begin{enumerate}
            \item If $s\notin Y$, then by Theorem 5.12, $s\in \overline{Y}$
            \item If $s\in Y$, then $s\in \overline{Y}$
        \end{enumerate}
        Thus, $s\in \overline{Y}$. By Theorem 3.13, $s\in \overline{A}$. Because $B$ is open in $X$, then by \ref{8.15}, $A = X\sm B$ is closed in $X$. Since $A$ is closed in $X$, then there exists some closed $A''$ such that $A''\cap X = A$. Thus, since $A \subset A''$, then by $3.13$, $\overline{A}\subset \overline{A''}$. Thus, $s\in \overline{A''} = A''$. Thus, since $s\in X$ ($a\leq s$ because $s= \sup Y$ and $s\leq b$ because $b$ is an upper bound of $Y$, and thus since $s\in [a,b]$ and $[a,b]\subset X$, then $s\in X$), then $s\in X \cap A'' = A$. Since $A$ is open in $X$, then there exists some open set $A'$ such that $X \cap A' = A$. Thus, $s\in A'$. Since $A'$ is open, then there exists the region $(r,t)$ containing $s$ such that $(r,t)\subset A'$. 
               \begin{enumerate}
            \item Note that $t\leq b$, since if $b<t$, then $b\in (r,t)$ and thus $b\in A'$. Since $b\in B$ and $B\subset X$, then $b\in X$. If $b\in A'$, then because $X\cap A' = A$, then $b\in A$, which is a contradiction.
        \end{enumerate}
        Thus, $r<s<t\leq b$. Thus, because regions are infinite, there exists some $s'\in (r,t)$ such that $a\leq s<s'<t\leq b$. Thus, $s'\in A'$. Because $a,b \in X$ and $X$ is an interval, then since $s'\in [ab]$, then $s'\in X$. Thus, $s'\in A$ and $s'<b$, then $s'\in Y$. Thus, $s\neq \sup Y$, which is a contradiction. 
        \item If $X$ is not an interval, then $X$ it is not connected. Since $X$ is not an interval, then for some $x,y \in X$, $[x,y]\not\subset X$. Thus, for some $z\in [x,y], z\notin X$. 
        \begin{enumerate}
            \item Assume, $z=y$, then, because $y\in X$, then $z\in X$, thus, $z\neq y$. Therefore, $z<y$
            \item Similarly, $z\neq x$. Thus, $x<z$
        \end{enumerate}
        Thus, let $A = \{s\in X | s <z\}$ and $B = \{s \in X | s>z\}$. It can be seen that $X = A \cup B$ and $A,B$ are disjoint. Because $y\in X$ and $y>z$, then $y\in B$ and $B \neq \emptyset$. Similarly, $A\neq \emptyset$
        \begin{enumerate}
            \item Thus, $A$ can be written as $X \cap A'$, where $A' = (-\infty, z)$. Because $A'$ is open (8.1), then by Definition \ref{8.12}, $A$ is open in $X$.
            \item Similarly, if $B' = (z,\infty)$, then $B$ is open in $X$
        \end{enumerate}
        By Definition 4.22, since $A,B$ are nonempty, open in X, and disjoint, then $X$ is not connected.
    \end{itemize}
\end{proof}\vspace{4pt}     \hrule   \vspace{4pt} 

Finally, we briefly explore functions on intervals. 

\subsection*{Definition 8.17}
\begin{defn}
\label{8.17}
Let $I$ be an interval and let $f:I\to\bbR.$ 
\begin{enumerate}
\item[a)] We say that $f$ is {\em increasing} on $I$ if, whenever $x,y\in I,$ with $x<y,$ $f(x)\leq f(y).$
\item[b)] We say that $f$ is {\em decreasing} on $I$  if, whenever $x,y\in I,$ with $x<y,$ $f(x)\geq f(y).$
\item[c)] We say that $f$ is {\em strictly increasing} on $I$ if, whenever $x,y\in I,$ with $x<y,$ $f(x)< f(y).$
\item[d)] We say that $f$ is {\em strictly decreasing} on $I$  if, whenever $x,y\in I,$ with $x<y,$ $f(x)> f(y).$
\end{enumerate}
\end{defn}




Note that ``strictly increasing'' means the same thing as ``order preserving.''

\subsection*{Lemma 8.18}
\begin{lem} If $f$ is strictly increasing or strictly decreasing on an interval $I$ then $f$ is injective on $I.$
\end{lem}
\vspace{4pt}     \hrule   \vspace{4pt} 
\begin{proof}:\\
\begin{itemize}
    \item Assume, for the sake of contradiction, that if $f$ is strictly increasing then if $f(x) = f(y)$, $x\neq y$. Since $x\neq y$, then without loss of generality, let $x<y$. Since $f$ is strictly increasing, then since $x<y$, $f(x)<f(y)$. Therefore, $f(x)\neq f(y)$, which is a contradiction. Thus, if $f(x) = f(y)$ and $f$ is strictly increasing, then $x = y$ and thus, $f$ is injective.
    \item Using similar logic, if $f$ is strictly decreasing, then $f$ is injective. 
\end{itemize}
\end{proof}\vspace{4pt}     \hrule   \vspace{4pt} 


\section*{Acknowledgments} 
Another scipt, another round of applause for my professor Oron Propp and his excellent office hours. Without him, my $8.16$ would be wrong but at least it would have good notation. Also thanks to Victor Hugo Almendra Hernández, he's a great grader and a great help. Thanks to my many great peers, chief among the help being Lina Piao and Rhea Kanuparthi (for helping me check), and Richard Gale (for presenting 8.3). 
\begin{thebibliography}{9}

\bibitem{My brain} Agustin.org


\end{thebibliography}

\end{document}

