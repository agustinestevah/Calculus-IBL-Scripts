\documentclass[openany, amssymb, psamsfonts]{amsart}
\usepackage{mathrsfs,comment}
\usepackage[usenames,dvipsnames]{color}
\usepackage[normalem]{ulem}
\usepackage{url}
\usepackage[all,arc,2cell]{xy}
\UseAllTwocells
\usepackage{enumerate}
\newcommand{\bA}{\mathbf{A}}
\newcommand{\bB}{\mathbf{B}}
\newcommand{\bC}{\mathbf{C}}
\newcommand{\bD}{\mathbf{D}}
\newcommand{\bE}{\mathbf{E}}
\newcommand{\bF}{\mathbf{F}}
\newcommand{\bG}{\mathbf{G}}
\newcommand{\bH}{\mathbf{H}}
\newcommand{\bI}{\mathbf{I}}
\newcommand{\bJ}{\mathbf{J}}
\newcommand{\bK}{\mathbf{K}}
\newcommand{\bL}{\mathbf{L}}
\newcommand{\bM}{\mathbf{M}}
\newcommand{\bN}{\mathbf{N}}
\newcommand{\bO}{\mathbf{O}}
\newcommand{\bP}{\mathbf{P}}
\newcommand{\bQ}{\mathbf{Q}}
\newcommand{\bR}{\mathbf{R}}
\newcommand{\bS}{\mathbf{S}}
\newcommand{\bT}{\mathbf{T}}
\newcommand{\bU}{\mathbf{U}}
\newcommand{\bV}{\mathbf{V}}
\newcommand{\bW}{\mathbf{W}}
\newcommand{\bX}{\mathbf{X}}
\newcommand{\bY}{\mathbf{Y}}
\newcommand{\bZ}{\mathbf{Z}}

%% blackboard bold math capitals
\newcommand{\bbA}{\mathbb{A}}
\newcommand{\bbB}{\mathbb{B}}
\newcommand{\bbC}{\mathbb{C}}
\newcommand{\bbD}{\mathbb{D}}
\newcommand{\bbE}{\mathbb{E}}
\newcommand{\bbF}{\mathbb{F}}
\newcommand{\bbG}{\mathbb{G}}
\newcommand{\bbH}{\mathbb{H}}
\newcommand{\bbI}{\mathbb{I}}
\newcommand{\bbJ}{\mathbb{J}}
\newcommand{\bbK}{\mathbb{K}}
\newcommand{\bbL}{\mathbb{L}}
\newcommand{\bbM}{\mathbb{M}}
\newcommand{\bbN}{\mathbb{N}}
\newcommand{\bbO}{\mathbb{O}}
\newcommand{\bbP}{\mathbb{P}}
\newcommand{\bbQ}{\mathbb{Q}}
\newcommand{\bbR}{\mathbb{R}}
\newcommand{\bbS}{\mathbb{S}}
\newcommand{\bbT}{\mathbb{T}}
\newcommand{\bbU}{\mathbb{U}}
\newcommand{\bbV}{\mathbb{V}}
\newcommand{\bbW}{\mathbb{W}}
\newcommand{\bbX}{\mathbb{X}}
\newcommand{\bbY}{\mathbb{Y}}
\newcommand{\bbZ}{\mathbb{Z}}

%% script math capitals
\newcommand{\sA}{\mathscr{A}}
\newcommand{\sB}{\mathscr{B}}
\newcommand{\sC}{\mathscr{C}}
\newcommand{\sD}{\mathscr{D}}
\newcommand{\sE}{\mathscr{E}}
\newcommand{\sF}{\mathscr{F}}
\newcommand{\sG}{\mathscr{G}}
\newcommand{\sH}{\mathscr{H}}
\newcommand{\sI}{\mathscr{I}}
\newcommand{\sJ}{\mathscr{J}}
\newcommand{\sK}{\mathscr{K}}
\newcommand{\sL}{\mathscr{L}}
\newcommand{\sM}{\mathscr{M}}
\newcommand{\sN}{\mathscr{N}}
\newcommand{\sO}{\mathscr{O}}
\newcommand{\sP}{\mathscr{P}}
\newcommand{\sQ}{\mathscr{Q}}
\newcommand{\sR}{\mathscr{R}}
\newcommand{\sS}{\mathscr{S}}
\newcommand{\sT}{\mathscr{T}}
\newcommand{\sU}{\mathscr{U}}
\newcommand{\sV}{\mathscr{V}}
\newcommand{\sW}{\mathscr{W}}
\newcommand{\sX}{\mathscr{X}}
\newcommand{\sY}{\mathscr{Y}}
\newcommand{\sZ}{\mathscr{Z}}


\renewcommand{\phi}{\varphi}
\renewcommand{\emptyset}{\O}

\newcommand{\abs}[1]{\lvert #1 \rvert}
\newcommand{\norm}[1]{\lVert #1 \rVert}
\newcommand{\sm}{\setminus}


\newcommand{\sarr}{\rightarrow}
\newcommand{\arr}{\longrightarrow}

\newcommand{\hide}[1]{{\color{red} #1}} % for instructor version
%\newcommand{\hide}[1]{} % for student version
\newcommand{\com}[1]{{\color{blue} #1}} % for instructor version
%\newcommand{\com}[1]{} % for student version
\newcommand{\meta}[1]{{\color{green} #1}} % for making notes about the script that are not intended to end up in the script
%\newcommand{\meta}[1]{} % for removing meta comments in the script

\DeclareMathOperator{\ext}{ext}
\DeclareMathOperator{\ho}{hole}
%%% hyperref stuff is taken from AGT style file
\usepackage{hyperref}  
\hypersetup{%
  bookmarksnumbered=true,%
  bookmarks=true,%
  colorlinks=true,%
  linkcolor=blue,%
  citecolor=blue,%
  filecolor=blue,%
  menucolor=blue,%
  pagecolor=blue,%
  urlcolor=blue,%
  pdfnewwindow=true,%
  pdfstartview=FitBH}   
  
\let\fullref\autoref
%
%  \autoref is very crude.  It uses counters to distinguish environments
%  so that if say {lemma} uses the {theorem} counter, then autrorefs
%  which should come out Lemma X.Y in fact come out Theorem X.Y.  To
%  correct this give each its own counter eg:
%                 \newtheorem{theorem}{Theorem}[section]
%                 \newtheorem{lemma}{Lemma}[section]
%  and then equate the counters by commands like:
%                 \makeatletter
%                   \let\c@lemma\c@theorem
%                  \makeatother
%
%  To work correctly the environment name must have a corrresponding 
%  \XXXautorefname defined.  The following command does the job:
%
\def\makeautorefname#1#2{\expandafter\def\csname#1autorefname\endcsname{#2}}
%
%  Some standard autorefnames.  If the environment name for an autoref 
%  you need is not listed below, add a similar line to your TeX file:
%  
%\makeautorefname{equation}{Equation}%
\def\equationautorefname~#1\null{(#1)\null}
\makeautorefname{footnote}{footnote}%
\makeautorefname{item}{item}%
\makeautorefname{figure}{Figure}%
\makeautorefname{table}{Table}%
\makeautorefname{part}{Part}%
\makeautorefname{appendix}{Appendix}%
\makeautorefname{chapter}{Chapter}%
\makeautorefname{section}{Section}%
\makeautorefname{subsection}{Section}%
\makeautorefname{subsubsection}{Section}%
\makeautorefname{theorem}{Theorem}%
\makeautorefname{thm}{Theorem}%
\makeautorefname{excercise}{Exercise}%
\makeautorefname{cor}{Corollary}%
\makeautorefname{lem}{Lemma}%
\makeautorefname{prop}{Proposition}%
\makeautorefname{pro}{Property}
\makeautorefname{conj}{Conjecture}%
\makeautorefname{defn}{Definition}%
\makeautorefname{notn}{Notation}
\makeautorefname{notns}{Notations}
\makeautorefname{rem}{Remark}%
\makeautorefname{quest}{Question}%
\makeautorefname{exmp}{Example}%
\makeautorefname{ax}{Axiom}%
\makeautorefname{claim}{Claim}%
\makeautorefname{ass}{Assumption}%
\makeautorefname{asss}{Assumptions}%
\makeautorefname{con}{Construction}%
\makeautorefname{prob}{Problem}%
\makeautorefname{warn}{Warning}%
\makeautorefname{obs}{Observation}%
\makeautorefname{conv}{Convention}%


%
%                  *** End of hyperref stuff ***

%theoremstyle{plain} --- default
\newtheorem{thm}{Theorem}[section]
\newtheorem{cor}{Corollary}[section]
\newtheorem{exercise}{Exercise}
\newtheorem{prop}{Proposition}[section]
\newtheorem{lem}{Lemma}[section]
\newtheorem{prob}{Problem}[section]
\newtheorem{conj}{Conjecture}[section]
%\newtheorem{ass}{Assumption}[section]
%\newtheorem{asses}{Assumptions}[section]

\theoremstyle{definition}
\newtheorem{defn}{Definition}[section]
\newtheorem{ass}{Assumption}[section]
\newtheorem{asss}{Assumptions}[section]
\newtheorem{ax}{Axiom}[section]
\newtheorem{con}{Construction}[section]
\newtheorem{exmp}{Example}[section]
\newtheorem{notn}{Notation}[section]
\newtheorem{notns}{Notations}[section]
\newtheorem{pro}{Property}[section]
\newtheorem{quest}{Question}[section]
\newtheorem{rem}{Remark}[section]
\newtheorem{warn}{Warning}[section]
\newtheorem{sch}{Scholium}[section]
\newtheorem{obs}{Observation}[section]
\newtheorem{conv}{Convention}[section]

%%%% hack to get fullref working correctly
\makeatletter
\let\c@obs=\c@thm
\let\c@cor=\c@thm
\let\c@prop=\c@thm
\let\c@lem=\c@thm
\let\c@prob=\c@thm
\let\c@con=\c@thm
\let\c@conj=\c@thm
\let\c@defn=\c@thm
\let\c@notn=\c@thm
\let\c@notns=\c@thm
\let\c@exmp=\c@thm
\let\c@ax=\c@thm
\let\c@pro=\c@thm
\let\c@ass=\c@thm
\let\c@warn=\c@thm
\let\c@rem=\c@thm
\let\c@sch=\c@thm
\let\c@equation\c@thm
\numberwithin{equation}{section}
\makeatother

\bibliographystyle{plain}

%--------Meta Data: Fill in your info------
\title{University of Chicago Calculus IBL Course;\\ Script 1}

\author{Agustin Esteva}

\date{Dec 12. 2023}

\begin{document}

\begin{abstract}
Sets and functions are among the most fundamental objects in mathematics.  A formal treatment
of set theory was first undertaken at the end of the 19th Century and was finally codified
in the form of the Zermelo-Fraenkel axioms.  While fascinating in its own right, pursuit of these
formalisms at this point would distract us from our main purpose of studying Calculus.  Thus, we
present a simplified version that will suffice for our immediate purposes.


\end{abstract}

\maketitle

\tableofcontents
\section{Script 1}
\subsection{Definition 1.1}
\begin{defn} (Working Definition)
A {\em set} is an object $S$ with the property that, given any $x$, we have the dichotomy that precisely
one of the two conditions $x\in S$ or $x\not\in S$ is true.  In the former case, we say that $x$ is an 
{\em element} of $S$, and in the latter, we say that $x$ is not an element of $S$.
\end{defn}


A set is often presented in one of the following forms:
\begin{itemize}
\item
A complete listing of its elements.

Example:  the set $S=\{1,2,3,4,5\}$ contains precisely the 
five smallest positive integers.


\item
A listing of some of its elements with ellipses to indicate unnamed elements.

Example 1:  the set $S=\{3, 4, 5, \ldots, 100\}$ contains the positive integers from 3 to 100,
including 6 through 99, even though these latter are not explicitly named.  


Example 2:  the set $S=\{2, 4, 6, \ldots, 2n, \ldots \}$ is the set of all positive even integers.


\item
A two-part indication of the elements of the set by first identifying the source of all elements
and then giving additional conditions for membership in the set.

Example 1: 
$S=\{x\in {\mathbb N}\mid \mbox{$x$ is prime}\}$ is the set of primes.  


Example 2:
$S=\{x\in {\mathbb Z}\mid \mbox{$x^2<3$}\}$ is the set of integers whose squares are less than 3.
\end{itemize}
\subsection{Definition 1.2}
\begin{defn}  
Two sets $A$ and $B$ are {\em equal} if they contain precisely the same elements, that is, $x\in A$
if and only if $x\in B$.  When $A$ and $B$ are equal, we denote this by $A=B$.
\end{defn}
\subsection{Definition 1.3}
\begin{defn}  
A set $A$ is a {\em subset} of a set $B$ if every element of $A$ is also an element of $B$, that is,
if $x\in A$, then $x\in B$.  When $A$ is a subset of $B$, we denote this by $A\subset B$.  If $A\subset B$ but $A\neq B$ 
we say that $A$ is a {\em proper} subset of $B.$ 
\end{defn}
\subsection{Example 1.4}
\begin{exmp} Let $A=\{1, \{2\}\}$.
\begin{proof}:\\
Is $1\in A$? Yes, because 1 is an element of A\\
Is $2\in A$? No, because $\{2\}$ is a set within A, not an element\\
Is $\{1\}\subset A$?  Yes, the set $\{1\}$ is a subset because every element of the set can also be found in A \\
Is $\{2\}\subset A$?  No, the element of \{2\}: 2, exists as a set in A, not an element \\
Is $1\subset A$?  No, $1$ is an element, not a set, so the relation doesn't even make sense. \\
Is $\{1\}\in A$?  No, the set $\{1\}$ doesn't exist within the set of $A$ \\
Is $\{2\}\in A$?   Yes, because the set of $\{2\}$ exists as an element in A\\
Is $\{\{2\}\}\subset A$? Yes, because the element of \{\{2\}\} is \{2\} which can be found in A \\
\end{proof}
\end{exmp}
\subsection{Definition 1.5}
\begin{defn}  Let $A$ and $B$ be two sets. 
The \emph{union} of $A$ and $B$ is the set
\[
A \cup B = \{x \mid \text{$x \in A$ or $x \in B$} \}.
\]
\end{defn}
\subsection{Definition 1.6}
\begin{defn}  Let $A$ and $B$ be two sets. 
The \emph{intersection} of $A$ and $B$ is the set
\[
A \cap B = \{ x \mid \text{$x \in A$ and $x \in B$} \}.
\]
\end{defn}
\subsection{Theorem 1.7}
\begin{thm} \label{basicsets} \meta{no proof required}
Let $A$ and $B$ be two sets.  Then:

\begin{enumerate}
\item[a)]
$A=B$ if and only if $A\subset B$ and $B\subset A$.
\item[b)]
$A\subset A\cup B$.

\item[c)]
$A\cap B\subset A$.
\end{enumerate}
\end{thm}

A special example of the intersection of two sets is when the two sets have no elements in common.
This motivates the following definition.
\subsection{Definition 1.8}
\begin{defn}  
The \emph{empty set} is the set with no elements, and it is denoted $\emptyset$.  That is,
no matter what $x$ is, we have $x\not\in \emptyset$.
\end{defn}  
\subsection{Definition 1.9}
\begin{defn}  
Two sets $A$ and $B$ are \emph{disjoint} if $A\cap B=\emptyset$.
\end{defn}  
\subsection{Example 1.10}
\begin{exmp}
     
Show that if $A$ is any set, then $\emptyset\subset A$. 
\begin{proof}Assume, for the sake of contradiction, that $\emptyset \not\subset A$. It follows that 
$\emptyset$ must have an element, $x$, that is not in set $A$. However, that is a contradiction because $x\in \emptyset$ but $\emptyset$ is defined as having no elements. So therefore $\emptyset \subset A$
\end{proof}
\end{exmp} 
\subsection{Definition 1.11}
\begin{defn}  
Let $A$ and $B$ be two sets. 
The \emph{difference} of $B$ from $A$ is the set
\[
A \setminus B = \{ x \in A \mid x \notin B \}.
\]
\end{defn}

The set $A \setminus B$ is also called the \emph{complement} of $B$ relative to $A$.
When the set $A$ is clear from the context, this set is sometimes denoted $B^{c}$, but we will 
try to avoid this imprecise formulation and use it only with warning.
\subsection{Example 1.12}

\begin{exmp}
Let $A=\{x\in\bbN\mid x\text{ is even}\}; B=\{x\in\bbN\mid x\text{ is odd}\}; C=\{x\in\bbN\mid x\text{ is prime}\}; D=\{x\in\bbN\mid x\text{ is a perfect square}\}.$
Find all possible set differences
\begin{proof}
\begin{enumerate} [1]
\item $A \backslash B =$ \{x is even $\mid$ x is not odd\} $= A$ 
\item $A \backslash C =$ \{x is even $\mid$ x is not prime\} $= B \setminus 2$ 
\item $A \backslash D =$ \{x is even $\mid$ x is not a perfect square\} $= \{x\in even \mid x\notin n^2\}$, where \emph{n} is an integer
\item$B \backslash A =$ \{x is odd $\mid$ x is not even\} $= B$ 
\item$B \backslash C =$ \{x is odd $\mid$ x is not prime\} $= \{x\in 2n+1 \mid x\notin prime\}$, where \emph{n} is an integer
\item $B \backslash D =$ \{x is odd $\mid$ x is not a perfect square\} $= \{x\in 2n+1 \mid x\notin n^2\}$, where \emph{n} is an integer
\item $C \backslash A =$ \{x is prime $\mid$ x is not even\} $= C \setminus 2$ 
\item $C \backslash B =$ \{x is prime $\mid$ x is not odd\} $= 2$ 
\item $C \backslash D =$ \{x is prime $\mid$ x is not a perfect square\} $= C$
\item $D \backslash A =$ \{x is a perfect square $\mid$ x is not even\} $=  \{x\in n^2 \mid x\notin even\}$, where \emph{n} is an integer
\item $D \backslash B =$ \{x is a perfect square $\mid$ x is not odd\} $= \{x\in n^2 \mid x\notin odd\}$, where \emph{n} is an integer
\item $D \backslash C =$ \{x is perfect square $\mid$ x is not prime\} $= D$ 
    \end{enumerate}
\end{proof}
\end{exmp} 
\subsection{Theorem 1.13}

\begin{thm} 
Let $X$ be a set, and let $A, B\subset X$.  Then:
\begin{enumerate}
\item[a)]
$X\setminus (A\cup B)=(X\setminus A)\cap (X\setminus B)$
\begin{proof}:\\
Suppose $x\in X\setminus (A\cup B)$,\\
then because $A,B \subset X$,\\
$x\in X$ and $x\notin A$ and $x\notin B$.\\
Therefore, $x\in (X\setminus A$) and $x\in (X\setminus B$)\, so \\
$X\setminus (A\cup B)\subset (X\setminus A)\cap (X\setminus B)$\\\\
Suppose $x\in (X\setminus A)\cap (X\setminus B)$,\\
then, $x\notin A$ and $x\notin B$,\\
and therefore $x\in X$ so\\
$x\in X\setminus (A \cup B)$. \\
Therefore:\\
$(X\setminus A)\cap (X\setminus B)\subset X\setminus (A\cup B)$
\end{proof}






\item[b)]
$X\setminus (A\cap B)=(X\setminus A)\cup (X\setminus B)$

\begin{proof}:\\
Suppose $x \in (X\setminus A\cap B)$,\\
then, because $A,B \subset X$:\\
$x\in X$ and $x\notin (A \cap B)$\\
If $x\notin (A \cap B)$, then \\
$x\notin A \cup x\notin B$\\
$x\in (X\setminus A) \cup x\in (X\setminus B)= x\in (X\setminus A)\cup (X\setminus B)$, so \\
$X\setminus (A\cap B)\subset (X\setminus A)\cup (X\setminus B)$\\\\
Suppose $x\in (X\setminus A)\cup (X\setminus B)$, \\
then, $x\in X$, and $x \notin (A\cap B)$\\
Therefore, because $A,B \subset X$:\\
$x\in X \setminus (A\cap B)$, so\\
$(X\setminus A)\cup (X\setminus B) \subset X\setminus (A\cap B)$

\end{proof}
\end{enumerate}
\end{thm}
Sometimes we will encounter families of sets. The definitions of intersection/union can be extended to infinitely many sets. 

\subsection{Definition 1.14}
\begin{defn}

 Let $\mathcal{A}=\{A_\lambda\mid \lambda\in I\}$ be a collection of sets indexed by a nonempty set $I.$ Then the intersection and union of $\mathcal{A}$ are the sets
$$\bigcap_{\lambda\in I} A_\lambda =\{x\mid x\in A_\lambda, \text{ for all } \lambda\in I\},$$
and
$$\bigcup_{\lambda\in I}A_\lambda =\{x\mid x\in A_\lambda, \text{ for some }\lambda\in I\}.$$
\end{defn}
\subsection{Theorem 1.15}
\begin{thm}  
Let $X$ be a set, and let  $\mathcal{A}=\{A_\lambda\mid \lambda\in I\}$ be a collection of subsets of $X.$ Then:
\end{thm}

\begin{enumerate} [1]

\item$X\setminus \left( \bigcup_{\lambda\in I}A_\lambda\right) =\bigcap_{\lambda\in I} (X\setminus A_\lambda)$
\begin{proof}:\\
Suppose $q\in X\setminus \left( \bigcup_{\lambda\in I}A_\lambda\right)$, \\
then, because $\mathcal{A} \subset X$, \\
$q\in X$ and $q\notin$ $\mathcal{A}$, for all $\lambda$\\
so since $q\in X \setminus \mathcal{A}$ for all $\lambda$ \\
$q\in \bigcap_{\lambda\in I} (X\setminus A_\lambda)$ and therefore $X\setminus \left( \bigcup_{\lambda\in I}A_\lambda\right) \subset \bigcap_{\lambda\in I} (X\setminus A_\lambda)$\\\\
Suppose $q\in \bigcap_{\lambda\in I} (X\setminus A_\lambda)$,\\
then $q\in (X\setminus \mathcal{A})$ for all $\lambda$\\
so since $\mathcal{A} \subset X$, $q\in X$ and $q\notin \mathcal{A}$ for all $\lambda$\\
$q\in X\setminus \left( \bigcup_{\lambda\in I}A_\lambda\right)$ and therefore\\
$q\in \bigcap_{\lambda\in I} (X\setminus A_\lambda) \subset X\setminus \left( \bigcup_{\lambda\in I}A_\lambda\right)$
\end{proof}


\item$X\setminus \left( \bigcap_{\lambda\in I}A_\lambda\right)  =\bigcup_{\lambda\in I} (X\setminus A_\lambda).$

\begin{proof}:\\
Suppose $q\in X\setminus \left( \bigcap_{\lambda\in I}A_\lambda\right)$, then \\
because $\mathcal{A} \subset X$, \\
$q\in X$ and $q\notin \mathcal{A}$ for some $\lambda$\\
Therefore, $q\in (X\setminus \mathcal{A})$ for some $\lambda$,\\
so $q\in \bigcup_{\lambda\in I} (X\setminus A_\lambda).$\\
$ X\setminus \left( \bigcap_{\lambda\in I}A_\lambda\right) \subset \bigcup_{\lambda\in I} (X\setminus A_\lambda)$\\\\
Suppose $q\in \bigcup_{\lambda\in I} (X\setminus A_\lambda)$, then\\
$q\in (X\setminus \mathcal{A}$) for some $\lambda$\\
so because $\mathcal{A} \subset X$, \\
$q\in X$ and $q\notin \mathcal{A}$ for some $\lambda$\\
Therefore $q\in X\setminus \left( \bigcap_{\lambda\in I}A_\lambda\right)$\\
$\bigcup_{\lambda\in I} (X\setminus A_\lambda)\subset X\setminus \left( \bigcap_{\lambda\in I}A_\lambda\right)$
\end{proof}
\end{enumerate}
\subsection{Definition 1.16}
\begin{defn}
Let $A$ and $B$ be two nonempty sets. 
The \emph{Cartesian product} of $A$ and $B$ is the set of ordered pairs
\[
A \times B = \{ (a, b) \mid \text{$a \in A$ and $b \in B$} \}.
\]
If $(a, b)$ and $(a', b') \in A \times B$, we say that $(a, b)$ and $(a', b')$ are
\emph{equal} if and only if $a = a'$ and $b = b'$. In this case, we write $
(a, b) = (a', b').$
\end{defn} 
\subsection{Definition 1.17}
\begin{defn} Let $A$ and $B$ be two nonempty sets.  
A \emph{function} $f$ from $A$ to $B$ is a subset $f \subset A \times B$ such that for all $a \in A$ there exists a unique $b \in B$ satisfying $(a, b) \in f$.  To express the idea that $(a, b) \in f$, we most
often write $f(a) = b$.  To express that $f$ is a function from $A$ to $B$ in symbols we write $f \colon A \rightarrow B$.  
\end{defn}
\subsection{Example 1.18}
\begin{exmp}  
Let the function $f \colon \mathbb{Z} \rightarrow \mathbb{Z}$ be defined by
$f(n)=2n$.  Write $f$ as a subset of $\mathbb{Z} \times \mathbb{Z}$.  

$f=\{ (n, 2n) \mid n \in \mathbb Z\}$



\end{exmp}
\subsection{Definition 1.19}
\begin{defn}
Let $f \colon A \rightarrow B$ be a function.  The \emph{domain} of $f$ is $A$ and the \emph{codomain} of $f$ is $B.$\\
If $X \subset A$, then the \emph{image of $X$ under $f$} is the set
\[
f(X) = \{ f(x) \in B \mid  x \in X \}.
\]
If $Y \subset B$, then the \emph{preimage of $Y$ under $f$} is the set
\[
f^{-1}(Y) = \{ a \in A \mid f(a) \in Y \}.
\]
\end{defn}
\subsection{Example 1.20}
\begin{exmp}
    Must $f(f^{-1}(Y))=Y$ and $f^{-1}(f(X))=X?$ For each, either prove that it always holds or give a counterexample.
\end{exmp}
\begin{proof}:\\
    Let $f \colon \mathbb R \rightarrow \mathbb R$ be a function where $f(x)=x^2$. \\
\\
If $Y=\mathbb R^{-}$, then $f^{-1}(Y)=\emptyset$ because there is no number which can be squared to obtain a negative number, so therefore:\\
$f(f^{-1}(Y))=f(\emptyset)=\emptyset \neq Y$. \\
\\
If $X = {4}$, then $f(4) = 16$, therefore, $f^{-1}(f(4)) = \{-4,4\}$. Because $\{-4,4\}\neq {4}$:\\
$f^{-1}(f(4)) \neq X$
\end{proof}
\subsection{Definition 1.21}
\begin{defn}
    A function $f \colon A \rightarrow B$ is \emph{surjective} (also known as `onto') if, 
for every $b\in B$, there is some $a\in A$ such that $f(a) = b$.  The function $f$ is \emph{injective} (also known as `one-to-one') if for all $a, a' \in A$, if $f(a) = f(a')$, then $a = a'$.  The function $f$ is \emph{bijective}, (also known as a bijection or a `one-to-one' correspondence) if it is surjective and injective.
\end{defn}
\subsection{Example 1.22}
\begin{exmp}
    Let $f:{\mathbb N}\rightarrow {\mathbb N}$ be defined by $f(n)=n^2$.  Is $f$ injective?  Is $f$ surjective?
\end{exmp}
\begin{proof}:\\
    Injective: yes, because $n,n' = n^2$, then $n=n'$\\
Surjecive: no, $\exists$ $ f(n)\in \mathbb N $ $ \forall $ $ n \in \mathbb N$ such that $f(n) \neq n^2$, example: $8 \in \mathbb N$ but no $f(n) = 8$
\end{proof}
\subsection{Example 1.23}
\begin{exmp}
    Let $f:{\mathbb N}\rightarrow {\mathbb N}$ be defined by $f(n)=n+2$.  Is $f$ injective?  Is $f$ surjective?
\end{exmp}
\begin{proof}:\\
    injective: yes, because $n+2 = n'+2$, $n=n'$\\
surjective:  no, $\exists$ $f(n)\in \mathbb N $ $ \forall $ $ n \in \mathbb N$ such that $f(n) \neq n+2$, example:
$1 \in \mathbb N$ but no $f(n) = 1$
\end{proof}
\subsection{Example 1.24}
\begin{exmp}
    Let $f:{\mathbb Z}\rightarrow {\mathbb Z}$ be defined by $f(x)=x^2$.  Is $f$ injective?  Is $f$ surjective?
\end{exmp}
\begin{proof}:\\
    injective: no, if $x^2 = x'^2$, then an injective function would have $x=x'$, but since negative numbers are in the domain of $\mathbb Z$, then $x\neq x'$.
Example: $f(4)=f(-4)$, but $4\neq -4$ \\
Surjecive: no, $\exists$ $ f(n)\in \mathbb N $ $ \forall $ $ n \in \mathbb N$ such that $f(n) \neq n^2$, example: $8 \in \mathbb N$ but no $f(n) = 8$
\end{proof}
\subsection{Example 1.25}
\begin{exmp}
    Let $f:{\mathbb Z}\rightarrow {\mathbb Z}$ be defined by $f(x)=x+2$.  Is $f$ injective?  Is $f$ surjective?
\end{exmp}
\begin{proof}:\\
    injective: yes, because $n+2 = n'+2$, $n=n'$\\
surjective: yes, $\forall $ $ x+2\in \mathbb Z$ $  \exists $ $  x \in \mathbb Z $ such that $f(x) = x+2$
\end{proof}
\subsection{Definition 1.26}
\begin{defn}
Let $f:A\longrightarrow B$ and $g:B\longrightarrow C. $ Then the \emph{composition} $g\circ f: A\longrightarrow C$ is defined by $(g\circ f)(x)=g(f(x)),$ for all $x\in A.$ 
\end{defn}
\subsection{Proposition 1.27}
\label{Proposition 1.27}
\begin{prop}
Let $A$, $B$, and $C$ be sets and suppose that $f:A\longrightarrow B$  and  $g:B\longrightarrow C.$  Then $g\circ f:A\longrightarrow C$ and
\end{prop}
\begin{enumerate}
\item[a)] if $f$ and $g$ are both injections, so is $g\circ f.$\\
\begin{proof}:\\
if $g(f(a))=g(f(a'))$, then $a=a'$\\
Let $f(a)=b$ and $f(a')=b'$\\
Because g is injective, $g(b) = g(b')$ and $b=b'$, therefore, because f is injective, $f(a)=f(a')$ and $a=a'$ 
\end{proof}
\item[b)] if $f$ and $g$ are both surjections, so is $g\circ f.$\\
\begin{proof}:\\
$\forall c\in C, \exists a\in A \mid g(f(a)) = c$\\
Since g is surjective, $\forall c\in C, \exists b\in B \mid g(b) = c$\\
Since f is surjective, $\forall b\in B, \exists a\in A \mid f(a) = b$\\
Therefore, $\forall c\in C, \exists a\in A \mid g(f(a)) = c$
\end{proof}
\item[c)] if $f$ and $g$ are both bijections, so is $g\circ f.$
\begin{proof}:\\
Because $g\circ f$ is injective and $g\circ f$ is surjective, $g\circ f$ is bijective. 
\end{proof}
\end{enumerate}
\subsection{Proposition 1.28}
\begin{prop} 
Suppose that $f \colon A \rightarrow B$ is bijective.  
Then there exists a bijection $g \colon B \rightarrow A$ that satisfies $(g\circ f)(a)=a, \forall a\in A$, and $(f\circ g)(b)=b,$ for all $b\in B.$ 
The function $g$ is often called the \emph{inverse} of $f$ and  denoted $f^{-1}$. It should not be confused with the preimage.
\end{prop}
\begin{proof}:\\
    \indent\indent Since $f$ is bijective: for each $b \in B$, there exists at least one $a \in A$ such that $f(a) = b$ \indent\indent because $f$ is surjective. And, for each $b \in B$, there exists at most one $a \in A$ such that \indent\indent $f(a) = b$ because $f$ is injective. Therefore, for each $b \in B$, there exists exactly one 
\indent\indent $a \in A$ such that $f(a) = b$.\\

\indent\indent Then, define $g : B \rightarrow A$ as $g(b) = a$ for all $b \in B$ where $f(a) = b$.\\
\indent\indent\indent\indent Given any $b_1, \: b_2 \in B$ where $g(b_1) = g(b_2)$, $f(g(b_1)) = f(g(b_2)) = b_1 = b_2$. Therefore,\\
\indent\indent\indent\indent $g$ is injective. Additionally, for all $a \in A$, $f(a) = b \implies g(b) = a$. Therefore, $g$ is\\
\indent\indent\indent\indent surjective. Therefore, $g$ is bijective.\\

\indent\indent Since $f$ and $g$ are bijective, $g \circ f : A \rightarrow A$ and $f \circ g : B \rightarrow B$ are both bijective
\indent\indent (Proposition \ref{Proposition 1.27}).

\indent\indent\indent\indent For all $a \in A$, there exists exactly one $b$ where $f(a) = b$ and $g(b) = a$. Therefore, for\\
\indent\indent\indent\indent all $a \in A$, $(g \circ f)(a) = g(f(a)) = g(b) = a$ and, for all $b \in B$, $(f \circ g)(b) = f(g(b)) = $\\
\indent\indent\indent\indent $f(a) = b$.
\end{proof}
\subsection{Definition 1.29}
\begin{defn}
We say that two sets $A$ and $B$ are in \emph{bijective correspondence} when there exists a bijection from $A$ to $B$ or, equivalently, from $B$ to $A$.
\end{defn}
\subsection{Definition 1.30}
\begin{defn}  
Let $n \in \mathbb{N}$ be a natural number.  We define $[n]$ to be the set $\{1, 2, \dotsc, n \}$.  
Additionally, we define $[0]=\emptyset$.
\end{defn}
\subsection{Definition 1.31}
\begin{defn}  
A set $A$ is \emph{finite} if $A=\emptyset$ or if there exists a natural number $n$ and a bijective correspondence between $A$ and the set $[n].$   If $A$ is not finite, we say that $A$ is \emph{infinite}.
\end{defn}
\subsection{Theorem 1.32}
\begin{thm}  \meta{No proof required}

Let $n, m\in {\mathbb N}$ with $n<m$.  \\ Then there does not exist an injective function
$f:[m]\rightarrow [n]$.
\end{thm}
{\it Hint: Fix $k\in\mathbb N.$ Prove, by induction on $n$, that for all $n\in\mathbb{N},$ there is no injective function $f:[n+k]\longrightarrow [n].$} 
\subsection{Theorem 1.33}
\begin{thm}
Let $A$ be a finite set. Suppose that $A$ is in bijective correspondence both with $[m]$ and with $[n]$.  Then $m = n$.
\end{thm}
\label{1.33}
\begin{proof}:\\
    Suppose $a\in A, n\in [n], m \in [m]$, and $f\colon A\rightarrow [n]$, and $g\colon A\rightarrow [m].$\\
Because $A$ is in bijective correspondance with both $[n],[m]$, both $f$ and $g$ are bijective.\\
Because $f$ and $g$ are surjective, then for every $n,m$, there exists an $a$ such that $f(a) = n$ and $g(a) =m$\\
Because $f$ is injective, then there exists a unique $a$ for every $f(a)$ and $g(a)$\\
Therefore, both $[m]$ and $[n]$ have the same number of elements as $A$ and so $m=n$. \\\\
Alternate proof:\\
$f\colon [n]\rightarrow A$, and $g\colon A\rightarrow [m].$\\
Since $g\circ f$ is bijective because both f and g are bijective (Proposition \ref{Proposition 1.27}), then $g\circ f$ it is also injective. Theorem 1.32 states that because $g\circ f: [n]\longrightarrow [m]$, is injective, $n\not < m$ and $m\not > n$ so therefore $n=m$
\end{proof}
\subsection{Definition 1.34}
\begin{defn}[Cardinality of a finite set]
 If $A$ is a finite set that is in bijective correspondence with $[n]$, then we say that the \emph{cardinality} of $A$ is $n$, and we write $\abs{A} = n$.  (By   Theorem~\ref{1.33}, there is exactly one such natural number $n$.) We also say that $A$ contains $n$ elements. We define the cardinality of the empty set to be $0.$
\end{defn}
\subsection{Example 1.35}
\begin{exmp}
    Let $A$ and $B$ be finite sets. 
\begin{enumerate}
\item[a)]
 If $A\subset B$, then $|A|\leq |B|$.
\begin{proof}:\\
If $A\subset B$, then $|A|\leq |B|$ implies that if $|A|\geq |B|$, then $ A\not\subset B$\\
If there are more elements in A than in B, then A cannot be a subset of B because not every element of A is in B.\\
Therefore, because the contrapositive is true,  If $A\subset B$, then $|A|\leq |B|$
\end{proof}



\item[b)]
Let  $A\cap B=\emptyset.$ Then $|A\cup B|=|A|+|B|.$ 
\item[c)] 
$|A\cup B|+|A\cap B|=|A|+|B|$
\begin{proof}:\\
Suppose $ n= |A\cup B|$ and $ m=|A\cap B|$, then $m=|A|+|B|-|A\cap B|$ and $n=|A\cap B|$. \\
Therefore, $m+n = m=|A|+|B|-|A\cap B|+|A\cap B| = |A|+|B|$
\end{proof}
\item[d)]  
 $|A\times B|=|A|\cdot |B|$.
 \end{enumerate}
\end{exmp}

\subsection{Definition 1.36}
\label{1.36}
\begin{defn}
    An infinite set $A$ is said to be {\em countable} if it is in bijective correspondence with $\bbN.$ An infinite set that is not countable is called {\em uncountable}
\end{defn}
\subsection{Example 1.37}
\begin{exmp}
    Prove that $\bbZ$ is a countable set.
\end{exmp}
\begin{proof}:\\
    Suppose $f\colon \bbN \rightarrow \bbZ$. \\
\[ f(n) = \begin{cases} 
          \frac{n}{2} & n\; is\; even\\
          -\frac{n-1}{2} & n\; is\; odd
       \end{cases}\]
Map of $f$:\\

$1\leftrightarrow 0$, $2\leftrightarrow 1$, $3\leftrightarrow -1$, ...\\

Because $f$ is shown to map out the elements of $\mathbb{N}$ to the elements of $\mathbb{Z}$, the latter is countable.
\end{proof}
\subsection{Example 1.38}
\label{1.38}
\begin{exmp}
    Prove that every infinite subset of a countable set is also countable.
\end{exmp}
\begin{proof}
    $A\subset B$, where B is a countable set, and  A is an infinite set.\\
Because B is countable, there must exist $f\colon B \rightarrow \bbN$, where f is bijective (\emph{Def. 1.36})
Because B is countable, the elements of B can be ordered as $f(A_i)$, such that the first element of B is $f(A_1)$ and so on. Because $A\subset B$, there must exist $g\colon A\rightarrow f(A_i)$. The function g is injective because $f(A_1)\neq f(A_2) $since$ A_1 \neq A_2$, and g is surjective because for every $f(A_i)$, there exists an $A$ because A is infinite. Therefore, $g$ is bijective. \\ 
The composition $(f\circ g)$ is therefore bijective and is defined from $A\longrightarrow \bbN$
\end{proof}
\subsection{Example 1.39}
\label{1.39}
\begin{exmp}
    Prove that if there is an injection $f:A\longrightarrow B$ where $B$ is countable and $A$ is infinite,  then $A$ is countable. {\it Hint: Use Exercise~\ref{1.38}}
\end{exmp}
\begin{proof}:\\
    Suppose $f(A)$ is the image of A under B. Therefore, $f(A)\subset B$ and $f(A)$ is countable because of Exercise 1.38. If there is a bijection $f:A\rightarrow f(A)$, then A will be countable. $f$ is an injection because since $f(A)\subset B$, and $a\in A$ then if $f(a)=f(a')$, then $a'=a$. $f$ must also be surjective because there exists an $a\in A$ for every $f(A)=b$, where $b\in f(A)$. Since $f$ is bijective, $A$ has bijective correspondence with a countable set, and by Definition \ref{1.36} it is therefore countable.
\end{proof}
\subsection{Example 1.40}
\begin{exmp}
    Prove that $\bbN\times \bbN$ is countable by considering the function $f:\bbN\times\bbN\longrightarrow \bbN$ given by $f(n,m)=(10^n-1)10^m$.
\end{exmp}
\begin{proof}:\\
$\bbN\times \bbN$ can be counted using Cantor Diagonals\\
Alternate Proof:

Because $\bbN\times \bbN$ and $\bbN$ is infinite, an injection in $f$ would prove that $\bbN\times \bbN$is countable. (\ref{1.39})\\
$f$ is injective if $(10^n-1)10^m=(10^p-1)10^q$, then $(n,m)=(p,q)$. I will prove by contradiction that if $(n,m)\neq(p,q)$, then $(10^n-1)10^m\neq(10^p-1)10^q$\\

Case I: $n\neq p$, $m=q$:\\
$10^m=10^q$\\
$(10^n-1)10^m=(10^p-1)10^q$
$10^n-1=10^p-1$\\
$n=p$ which is a contradiction\\

Case II: $n=p$, $m\neq q$:\\
$10^n-1=10^p-1$\\
$(10^n-1)10^m=(10^p-1)10^q$
$10^m=10^q$\\
$m=q$ which is a contradiction\\

Case III: $n\neq p$, $m\neq q$:\\
Let $m<q$\\
$(10^n-1)=(10^p-1)(10^q/10^m)$\\
$(10^n-1)=(10^p-1)(10^{q-m})$\\
$10^{q-m}$ can be divided by 10, so it is therefore even.\\
odd=(odd)(even)\\
odd=even, which is a contradiction.

Because $(n,m)\neq(p,q)$, then $(10^n-1)10^m\neq(10^p-1)10^q$, then if $(10^n-1)10^m=(10^p-1)10^q$, then $(n,m)=(p,q)$. Proving that f is injective, therefore $\bbN\times \bbN$ is countable. 
\end{proof}





\end{document}