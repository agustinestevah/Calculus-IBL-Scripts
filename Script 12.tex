
\documentclass[openany, amssymb, psamsfonts]{amsart}
\usepackage{mathrsfs,comment}
\usepackage[usenames,dvipsnames]{color}
\usepackage[normalem]{ulem}
\usepackage{url}
\usepackage{tikz}
\usepackage{tkz-euclide}
\usepackage{lipsum}
\usepackage[all,arc,2cell]{xy}
\UseAllTwocells
\usepackage{enumerate}
\newcommand{\bA}{\mathbf{A}}
\newcommand{\bB}{\mathbf{B}}
\newcommand{\bC}{\mathbf{C}}
\newcommand{\bD}{\mathbf{D}}
\newcommand{\bE}{\mathbf{E}}
\newcommand{\bF}{\mathbf{F}}
\newcommand{\bG}{\mathbf{G}}
\newcommand{\bH}{\mathbf{H}}
\newcommand{\bI}{\mathbf{I}}
\newcommand{\bJ}{\mathbf{J}}
\newcommand{\bK}{\mathbf{K}}
\newcommand{\bL}{\mathbf{L}}
\newcommand{\bM}{\mathbf{M}}
\newcommand{\bN}{\mathbf{N}}
\newcommand{\bO}{\mathbf{O}}
\newcommand{\bP}{\mathbf{P}}
\newcommand{\bQ}{\mathbf{Q}}
\newcommand{\bR}{\mathbf{R}}
\newcommand{\bS}{\mathbf{S}}
\newcommand{\bT}{\mathbf{T}}
\newcommand{\bU}{\mathbf{U}}
\newcommand{\bV}{\mathbf{V}}
\newcommand{\bW}{\mathbf{W}}
\newcommand{\bX}{\mathbf{X}}
\newcommand{\bY}{\mathbf{Y}}
\newcommand{\bZ}{\mathbf{Z}}

%% blackboard bold math capitals
\newcommand{\bbA}{\mathbb{A}}
\newcommand{\bbB}{\mathbb{B}}
\newcommand{\bbC}{\mathbb{C}}
\newcommand{\bbD}{\mathbb{D}}
\newcommand{\bbE}{\mathbb{E}}
\newcommand{\bbF}{\mathbb{F}}
\newcommand{\bbG}{\mathbb{G}}
\newcommand{\bbH}{\mathbb{H}}
\newcommand{\bbI}{\mathbb{I}}
\newcommand{\bbJ}{\mathbb{J}}
\newcommand{\bbK}{\mathbb{K}}
\newcommand{\bbL}{\mathbb{L}}
\newcommand{\bbM}{\mathbb{M}}
\newcommand{\bbN}{\mathbb{N}}
\newcommand{\bbO}{\mathbb{O}}
\newcommand{\bbP}{\mathbb{P}}
\newcommand{\bbQ}{\mathbb{Q}}
\newcommand{\bbR}{\mathbb{R}}
\newcommand{\bbS}{\mathbb{S}}
\newcommand{\bbT}{\mathbb{T}}
\newcommand{\bbU}{\mathbb{U}}
\newcommand{\bbV}{\mathbb{V}}
\newcommand{\bbW}{\mathbb{W}}
\newcommand{\bbX}{\mathbb{X}}
\newcommand{\bbY}{\mathbb{Y}}
\newcommand{\bbZ}{\mathbb{Z}}

%% script math capitals
\newcommand{\sA}{\mathscr{A}}
\newcommand{\sB}{\mathscr{B}}
\newcommand{\sC}{\mathscr{C}}
\newcommand{\sD}{\mathscr{D}}
\newcommand{\sE}{\mathscr{E}}
\newcommand{\sF}{\mathscr{F}}
\newcommand{\sG}{\mathscr{G}}
\newcommand{\sH}{\mathscr{H}}
\newcommand{\sI}{\mathscr{I}}
\newcommand{\sJ}{\mathscr{J}}
\newcommand{\sK}{\mathscr{K}}
\newcommand{\sL}{\mathscr{L}}
\newcommand{\sM}{\mathscr{M}}
\newcommand{\sN}{\mathscr{N}}
\newcommand{\sO}{\mathscr{O}}
\newcommand{\sP}{\mathscr{P}}
\newcommand{\sQ}{\mathscr{Q}}
\newcommand{\sR}{\mathscr{R}}
\newcommand{\sS}{\mathscr{S}}
\newcommand{\sT}{\mathscr{T}}
\newcommand{\sU}{\mathscr{U}}
\newcommand{\sV}{\mathscr{V}}
\newcommand{\sW}{\mathscr{W}}
\newcommand{\sX}{\mathscr{X}}
\newcommand{\sY}{\mathscr{Y}}
\newcommand{\sZ}{\mathscr{Z}}


\renewcommand{\phi}{\varphi}
\renewcommand{\emptyset}{\O}

\newcommand{\abs}[1]{\lvert #1 \rvert}
\newcommand{\norm}[1]{\lVert #1 \rVert}
\newcommand{\sm}{\setminus}


\newcommand{\sarr}{\rightarrow}
\newcommand{\arr}{\longrightarrow}

\newcommand{\hide}[1]{{\color{red} #1}} % for instructor version
%\newcommand{\hide}[1]{} % for student version
\newcommand{\com}[1]{{\color{blue} #1}} % for instructor version
%\newcommand{\com}[1]{} % for student version
\newcommand{\meta}[1]{{\color{green} #1}} % for making notes about the script that are not intended to end up in the script
%\newcommand{\meta}[1]{} % for removing meta comments in the script

\DeclareMathOperator{\ext}{ext}
\DeclareMathOperator{\ho}{hole}
%%% hyperref stuff is taken from AGT style file
\usepackage{hyperref}  
\hypersetup{%
  bookmarksnumbered=true,%
  bookmarks=true,%
  colorlinks=true,%
  linkcolor=blue,%
  citecolor=blue,%
  filecolor=blue,%
  menucolor=blue,%
  pagecolor=blue,%
  urlcolor=blue,%
  pdfnewwindow=true,%
  pdfstartview=FitBH}   
  
\let\fullref\autoref
%
%  \autoref is very crude.  It uses counters to distinguish environments
%  so that if say {lemma} uses the {theorem} counter, then autrorefs
%  which should come out Lemma X.Y in fact come out Theorem X.Y.  To
%  correct this give each its own counter eg:
%                 \newtheorem{theorem}{Theorem}[section]
%                 \newtheorem{lemma}{Lemma}[section]
%  and then equate the counters by commands like:
%                 \makeatletter
%                   \let\c@lemma\c@theorem
%                  \makeatother
%
%  To work correctly the environment name must have a corrresponding 
%  \XXXautorefname defined.  The following command does the job:
%
\def\makeautorefname#1#2{\expandafter\def\csname#1autorefname\endcsname{#2}}
%
%  Some standard autorefnames.  If the environment name for an autoref 
%  you need is not listed below, add a similar line to your TeX file:
%  
%\makeautorefname{equation}{Equation}%
\def\equationautorefname~#1\null{(#1)\null}
\makeautorefname{footnote}{footnote}%
\makeautorefname{item}{item}%
\makeautorefname{figure}{Figure}%
\makeautorefname{table}{Table}%
\makeautorefname{part}{Part}%
\makeautorefname{appendix}{Appendix}%
\makeautorefname{chapter}{Chapter}%
\makeautorefname{section}{Section}%
\makeautorefname{subsection}{Section}%
\makeautorefname{subsubsection}{Section}%
\makeautorefname{theorem}{Theorem}%
\makeautorefname{thm}{Theorem}%
\makeautorefname{excercise}{Exercise}%
\makeautorefname{cor}{Corollary}%
\makeautorefname{lem}{Lemma}%
\makeautorefname{prop}{Proposition}%
\makeautorefname{pro}{Property}
\makeautorefname{conj}{Conjecture}%
\makeautorefname{defn}{Definition}%
\makeautorefname{notn}{Notation}
\makeautorefname{notns}{Notations}
\makeautorefname{rem}{Remark}%
\makeautorefname{quest}{Question}%
\makeautorefname{exmp}{Example}%
\makeautorefname{ax}{Axiom}%
\makeautorefname{claim}{Claim}%
\makeautorefname{ass}{Assumption}%
\makeautorefname{asss}{Assumptions}%
\makeautorefname{con}{Construction}%
\makeautorefname{prob}{Problem}%
\makeautorefname{warn}{Warning}%
\makeautorefname{obs}{Observation}%
\makeautorefname{conv}{Convention}%


%
%                  *** End of hyperref stuff ***

%theoremstyle{plain} --- default
\newtheorem{thm}{Theorem}[section]
\newtheorem{cor}{Corollary}[section]
\newtheorem{exercise}{Exercise}
\newtheorem{prop}{Proposition}[section]
\newtheorem{lem}{Lemma}[section]
\newtheorem{prob}{Problem}[section]
\newtheorem{conj}{Conjecture}[section]
%\newtheorem{ass}{Assumption}[section]
%\newtheorem{asses}{Assumptions}[section]

\theoremstyle{definition}
\newtheorem{defn}{Definition}[section]
\newtheorem{ass}{Assumption}[section]
\newtheorem{asss}{Assumptions}[section]
\newtheorem{ax}{Axiom}[section]
\newtheorem{con}{Construction}[section]
\newtheorem{exmp}{Example}[section]
\newtheorem{notn}{Notation}[section]
\newtheorem{notns}{Notations}[section]
\newtheorem{pro}{Property}[section]
\newtheorem{quest}{Question}[section]
\newtheorem{rem}{Remark}[section]
\newtheorem{warn}{Warning}[section]
\newtheorem{sch}{Scholium}[section]
\newtheorem{obs}{Observation}[section]
\newtheorem{conv}{Convention}[section]

%%%% hack to get fullref working correctly
\makeatletter
\let\c@obs=\c@thm
\let\c@cor=\c@thm
\let\c@prop=\c@thm
\let\c@lem=\c@thm
\let\c@prob=\c@thm
\let\c@con=\c@thm
\let\c@conj=\c@thm
\let\c@defn=\c@thm
\let\c@notn=\c@thm
\let\c@notns=\c@thm
\let\c@exmp=\c@thm
\let\c@ax=\c@thm
\let\c@pro=\c@thm
\let\c@ass=\c@thm
\let\c@warn=\c@thm
\let\c@rem=\c@thm
\let\c@sch=\c@thm
\let\c@equation\c@thm
\numberwithin{equation}{section}
\makeatother

\bibliographystyle{plain}

%--------Meta Data: Fill in your info------
\title{University of Chicago Calculus IBL Course}

\author{Agustin Esteva}

\date{Mar 29. 2024}

\begin{document}

\begin{abstract}

16310's Script 12.\\ Let me know if you see any errors! Contact me at aesteva@uchicago.edu.


\end{abstract}

\maketitle

\tableofcontents

\setcounter{section}{12}
Throughout this sheet, we let $f\colon A \to \bbR$ be a real valued function with domain $A \subset \bbR$. We also now assume the domain $A \subset \bbR$ is open. 
\section*{Definition 12.1: Derivative}
\begin{defn}\label{12.1}
	The \emph{derivative} of $f$ at a point $a \in A$ is the number $f'(a)$ defined by the following limit:
	\[
		f'(a) = \lim_{h \to 0} \frac{f(a + h) - f(a)}{h}, 
	\]
	provided the limit on the right hand side exists.
	If $f'(a)$ exists, we say that $f$ is \emph{differentiable at $a$}. If $f$ is differentiable at all points of its domain, we say that $f$ is \emph{differentiable}. In this case, the values $f'(a)$ define a new function $f'\colon A \to \bbR$ called the \emph{derivative} of $f$.
\end{defn}

\section*{Remark 12.2}
\begin{rem} \label{12.2}If $A$ is not open the limit in Definition \ref{12.1} may not exist. For example, if $f\colon [a,b]\to \bbR$ then we cannot define the derivative at the endpoints. 
 For any $c$ in the domain of $f,$ we define the {\em right-hand-derivative} $f'_{+}(c)$ and the {\em left-hand-derivative} $f'_{-}(c)$  by
$$f'_{+}(c)=\lim_{h\to 0^+} \frac{f(c+h)-f(c)}{h}\qquad \text{and} \qquad f'_{-}(c)= \lim_{h\to 0^{-}}\frac{f(c+h)-f(c)}{h}.$$

 We say that $f$ is differentiable on $[a,b]$ if $f$ is differentiable on $(a,b)$ and $f'_{+}(a)$ and  $f'_{-}(b)$ exist. 
\end{rem}

\section*{Lemma 12.3}
\begin{lem} \label{12.3}
	Let $a \in \bbR$. Then 
	\[
		\lim\limits_{x \to a}f(x) = \lim\limits_{h \to 0}f(a + h),
	\]
	assuming that one of the two limits exists. (So if the limit on the left exists, so does the one on the right, and they are equal. Similarly, if the limit on the right exists, then so does the one on the left, and they are equal.)
\end{lem} 

 \vspace{4pt}     \hrule   \vspace{4pt} \begin{proof}:\\
\textbf{Lemma $\triangle$:} $a\in LP(A)$ if an only if $0\in LP(A - a)$
\begin{proof}:\\
    \begin{itemize}
        \item ($\implies$ :) If $a\in LP(A)$, then assume, for the sake of contradiction, that $0\notin LP(A-a)$. Thus, there exists some region $R_0 = (r,t)$ containing $0$ such that $(r,t) \cap (A-a)\setminus\{0\} = \emptyset$. Thus, because $r<0<t$, then $r+a<a<t+a$. Because $R_a = (r+a, t+a)$ is a region containing $a$ and $a\in LP(A)$, then there exists some $x\in R_a$ such that $x\in A$ and $x\neq a$. Thus, WLOG, $r+a<a<x<t+a$ and so $r<0<x-a<t$. Thus, because $(x-a)\in A-a$, then $(x-a)\in (R_0 \cap (A-a)\sm\{0\}$, which is a contradiction. Thus, $0\in LP(A-a)$
        \item ($\impliedby$ :) The proof is near identical logic as above.
    \end{itemize}
\end{proof}
Lemma $\triangle$ is used because if one defines:
\[f: A \to \bbR\]
\[g: A-a \to \bbR\]
such that $g(x) = f(a+x)$, then the following proof is valid since one can just substitute $g(h)$ for $f(a+h)$. 
 \begin{itemize}
     \item Assume that $\lim\limits_{h\to 0}f(a+h) = L$ exists. Then, $0\in LP(A-a)$ and for all $\epsilon>0$, there exists a $\delta >0$ such that if $(a+h) \in A$ and $0<|a+h - a|<\delta$, then $|f(a+h)-L|<\epsilon$. Thus, since $|h| = |(a+h) - a|$, then if $(a+h) = x$, then for all $\epsilon >0$, if $x\in A$ and $0<|x-a|<\delta$, then $|f(x) - L|<\epsilon$ and thus, because by our Lemma above, $a\in LP(A)$, then $\lim\limits_{x\to a}f(x) = L$
     \item Assume that $\lim\limits_{x\to a}f(x) = L$ exists. Thus, $a\in LP(A)$, and for all $\epsilon>0$, there exists a $\delta>0$ such that if $x\in A$ and $0<|x-a|<\delta$, then $|f(x) - L|< \epsilon$. Therefore, since $0\in LP (A - a)$ (lemma above), then $x = a+h$, and then if $0<|h| < \delta$, then $f(a+h) - L < \epsilon$, and thus $\lim\limits_{h\to 0}f(a+h) = L$.
 \end{itemize}
Because in all cases, $\lim\limits_{x\to a}f(x) = L$ and $\lim\limits_{h\to 0}f(a+h) = L$, then $\lim\limits_{x\to a}f(x) = \lim\limits_{h\to 0}f(a+h)$.\\
\end{proof} \vspace{4pt}     \hrule   \vspace{4pt}

\textbf{GOD LEMMA} \label{GOD LEMMA}$\lim\limits_{x\to a}(x-a) = \lim\limits_{h\to 0}h$ if either one exists.
 \vspace{4pt}     \hrule   \vspace{4pt} \begin{proof}:\\
By Corollary 11.12, $x$ and $a$ are continuous at $a$. Thus, by Corollary 11.10 and Theorem 11.5, \[\lim\limits_{x\to a}(x-a) = \lim\limits_{x\to a}(x) - \lim\limits_{x\to a}(a) = \lim\limits_{x\to a}(x) - a\] 
By Lemma \ref{12.3}, $\lim\limits_{x\to a}(x) = \lim\limits_{h\to 0 } (a+h)$. Thus, \[\lim\limits_{x\to a}(x-a) = \lim\limits_{h\to 0 } (a+h) - a = \lim\limits_{h\to 0 } h\]
\end{proof} \vspace{4pt}     \hrule   \vspace{4pt}

\section*{Theorem 12.4}
\begin{thm} \label{12.4}Let $a\in \bbR.$
Then	 $f$ is differentiable at $a$ if and only if  $\displaystyle \lim_{x \to a} \frac{f(x) - f(a)}{x - a}$ exists. Moreover, if $f$ is differentiable at $a,$ then the derivative of $f$ at $a$ is given by this limit:
	\[
		f'(a) = \lim_{x \to a} \frac{f(x) - f(a)}{x - a}.
	\]
\end{thm}
 \vspace{4pt}     \hrule   \vspace{4pt} \begin{proof}:\\
 \begin{itemize}
     \item ($\implies$ :) If $f$ is differentiable at $a$, then by Definition \ref{12.1}, 
\[f'(a) = \lim\limits_{h\to 0} \frac{f(a+h)-f(a)}{h}\]
\begin{itemize}
    \item By Theorem \ref{12.5}, $f$ is continuous at $a$ and so by Lemma \ref{12.3}, $\lim\limits_{x\to a} f(x) = \lim\limits_{h\to 0}f(a+h)$.
    \item Because $f(x)$ is continuous at $a$, then if $a=x$, $f(a)$ is continuous at $a$.
    \item As proved in Corollary 11.12, $h$ is continuous at $a$.
\end{itemize}
Thus, by Theorem 11.9 and my GOD LEMMA:
\[\lim\limits_{h\to 0} \frac{f(a+h)-f(a)}{h}=  \frac{\lim\limits_{h\to 0} f(a+h)-\lim\limits_{h\to a}f(a)}{\lim\limits_{h\to 0} h} = \frac{\lim\limits_{x\to a} f(x) - f(a)}{\lim\limits_{x\to a } (x-a)} = \lim\limits_{x\to a} \frac{f(x)-f(a)}{x-a}\] Note that this is valid as $h\neq 0$ since $0<|h|<\delta$ for all $\epsilon>0$. 
\item ($\impliedby$:) Similar as above logic applies.
 \end{itemize}
 \textbf{Unfortunately, although this proof and the lemma before it are sick, they do not work (hint: "dividing by 0"!). Look at the "Correct proof" below for corrected version)}

\end{proof} \vspace{4pt}     \hrule   \vspace{4pt}
 \vspace{4pt}     \hrule   \vspace{4pt} \begin{proof} Correct proof:
 \begin{itemize}
     \item ($\implies$ :) Since $f$ is differentiable at $a$, then $
     \lim\limits_{h\to 0}\frac{f(a+h) - f(a)}{h}$ exists. Thus, let $g(a+h) = \frac{f(a+h) - f(a)}{h}$. Thus, by Lemma \ref{12.3}, $\lim\limits_{h\to 0} g(a+h) = \lim\limits_{x\to a}g(x)$.  It follows that $x = a+h$ and $h = x-a$. Therefore, $\lim\limits_{h\to 0} \frac{f(a+h) - f(a)}{h} = \lim\limits_{x\to a}\frac{f(x) - f(a)}{x-a}$. 
     \item ($\impliedby$ :) Since $\lim\limits_{x\to a} \frac{f(x)-f(a)}{x-a}$ exists, then let $g(x) = \frac{f(x) - f(a)}{x-a}$. By Lemma \ref{12.3}, because $\lim\limits_{x\to a}g(x) = \lim\limits_{h\to 0}g(a+h)$, then $\lim\limits_{x\to a}\frac{f(x) - f(a)}{x-a} = \lim\limits_{h\to 0}\frac{f(a+h) - f(a)}{a+h-a} = \lim\limits_{h\to 0} \frac{f(a+h) - f(a)}{h}$. Thus, by Definition \ref{12.1}, because $f'(a)$ exists, then $f$ is differentiable at $a$. 
 \end{itemize}
\end{proof} \vspace{4pt}     \hrule   \vspace{4pt}

\section*{Theorem 12.5}
\begin{thm}
	\label{12.5}
	If $f$ is differentiable at $a$, then $f$ is continuous at $a$.
\end{thm}
\vspace{4pt}     \hrule   \vspace{4pt} \begin{proof}:\\
Because $f$ is differentiable at $a$, then by Definition \ref{12.1}, $f'(a) = \lim\limits_{h\to 0}\frac{f(a+h) - f(a)}{h}$. Thus, for some $\epsilon>0$ there exists a $\delta'>0$ such that if $(a+h)\in A$ and $0<|(a+h)-a|<\delta'$, then $|\frac{f(a+h)- f(a)}{h} - f'(a)|< \epsilon$ and therefore:
\begin{align*}
    |\frac{f(a+h)- f(a)}{h} - f'(a)| &< \epsilon\\
    |\frac{f(a+h)-f(a)}{h} - \frac{f'(a)h}{h}| &< \epsilon\\
    |f(a+h)-f(a) - f'(a)h| &< \epsilon |h|
\end{align*}
Therefore, for all $\epsilon >0$, there exists a $\delta = \min(\delta', \frac{\epsilon}{f'(a)+ \epsilon})$, such that if if $(a+h) \in A$ and $0< |h| <\delta$, then $0<|h|<\frac{\epsilon}{f'(a) + \epsilon}$, and so :
\begin{align*}
|f(a+h) - f(a)| &\leq |f(a+h) - (f(a) - f'(a)h)| + |(f(a) - f'(a)h) - f(a)|\\
|f(a+h) - (f(a) - f'(a)h)| + |(f(a) - f'(a)h) - f(a)| &= |f(a+h) - (f(a) - f'(a)h)| + |f'(a)h|\\
|f(a+h) - (f(a) - f'(a)h)| + |f'(a)h| &< \epsilon|h| + |f'(a)||h|\\
|f(a+h) - (f(a) - f'(a)h)| + |f'(a)h| &< |h|(\epsilon + |f'(a)|)\\
|f(a+h) - (f(a) - f'(a)h)| + |f'(a)h| &< \frac{\epsilon}{f'(a) + \epsilon}(\epsilon + |f'(a)|)\\
|f(a+h) - (f(a) - f'(a)h)| + |f'(a)h| &< \epsilon
\end{align*}
Thus, because $|f(a+h) - f(a)| < \epsilon$, then $\lim\limits_{h\to 0}f(a+h) = f(a)$ and therefore, by Lemma \ref{12.3}, $\lim\limits_{x\to a}f(x) = f(a)$, and so by Theorem 11.5, $f$ is continuous at $a$.
\end{proof} \vspace{4pt}     \hrule   \vspace{4pt}
\vspace{4pt}     \hrule   \vspace{4pt} \begin{proof} Easier Proof:\\
Let $x\in A$. If $f$ is differentiable at $a$ then by Definition \ref{12.1}, $f'(a) = \lim\limits_{h\to 0} \frac{f(a+h) - f(a)}{h}$ exists. Thus by Theorem \ref{12.4}, $f'(a) = \lim\limits_{h\to 0} \frac{f(x) - f(a)}{x-a}$ exists. Thus, WTS that $\lim\limits_{x\to a}f(x) = f(a)$. 
\begin{align*}
    \lim\limits_{x\to a}\frac{f(x)- f(a)}{x-a} &= f'(a)\\
    \lim\limits_{x\to a}\frac{f(x)- f(a)}{x-a} \cdot \lim\limits_{x\to a}(x-a)&= f'(a) \lim\limits_{x\to a}(x-a)\\
    \tag{Theorem 11.9} \lim\limits_{x\to a} (\frac{f(x)- f(a)}{x-a} \cdot (x-a)) &= f'(a)\lim\limits_{x\to a}(x-a)\\
\tag{Evaluating RHS*}    \lim\limits_{x\to a} (f(x) - f(a)) &= 0\\
\tag{Theorem 11.12} \lim\limits_{x\to a}f(x) &= f(a)
\end{align*}
Thus, by Theorem 11.5, $f$ is continuous at $a$.\\
*: Because $g = x-a$ is continuous, then $\lim_{x\to a}g(x) = g(a) = a-a = 0$.
\end{proof} \vspace{4pt}     \hrule   \vspace{4pt}

\section*{Example 12.6}
\begin{exmp}
	Show that the converse of Theorem~\ref{12.5} is not true.
\end{exmp}
\vspace{4pt}     \hrule   \vspace{4pt} \begin{proof}
By Example 11.7, $f(x) = |x|$ is continuous. Thus, $f(x)$ is continuous at all $x\in \bbR$ and therefore, $f(x)$ is continuous at $f(0) = 0$. However, consider $\lim\limits_{h\to 0} \frac{f(h) - f(0)}{h} = \frac{|h| - |0|}{h} = \frac{|h|}{h}$. Thus, $\lim\limits_{h\to 0^-}\frac{|h|}{h} = -1$ and $\lim\limits_{h\to 0^+}\frac{|h|}{h}= 1$. Thus, by additional exercise 2 on script 11 (see script 11 for proof), since $\lim\limits_{h\to 0^-}\frac{|h|}{h} \neq \lim\limits_{h\to 0^+}\frac{|h|}{h}$, then $\lim\limits_{h\to 0}\frac{|h|}{h} = f'(0)$ does not exist.
\end{proof} \vspace{4pt}     \hrule   \vspace{4pt}

\section*{Example 12.7}
\begin{exmp}\label{12.7}
	Show that, for all $n\in\bbN$, 
	\[
		x^n - a^n = (x - a)(x^{n-1} + a x^{n-2} + a^2 x^{n-3} + \dotsb + a^{n-2} x + a^{n-1}),
	\]
	or equivalently,
	\[
		x^n - a^n = (x-a)\left(\sum_{i=0}^{n-1} x^{n-1-i} a^i\right).
	\]
\end{exmp}
\vspace{4pt}     \hrule   \vspace{4pt} \begin{proof}
Proof by induction:
\begin{enumerate}
    \item If $n= 1$, then $x^1- a^1 = (x-a)$. Thus, because $n-1 = 0$ and $x^0 = 1$, then since $(x-a)\sum_{i=0}^{n-1}x^{n-1-i}a^i = (x-a)(x^{0}a^i) = (x-a)(1)$, then \[x^1-a^1 = (x-a)\left(\sum_{i=0}^{1-1}x^{1-1-i}a^i\right) = (x-a)(x^0a^0= (x-a)(1) = x-a\]
    \item If $n=k-1$, assume that \[x^{k-1}-a^{k-1} = (x-a)\left(\sum_{i=0}^{k-2} x^{k-2-i} a^i\right)\]
    \item If $n=k$, then since \[(x^{k} - a^{k}) = x(x^{k-1} - a^{k-1}) +x(a)^{k-1} - a^k =x(x^{k-1} - a^{k-1})  + a^{k-1}(x-a)\] then by our inductive step:\[(x^{k} - a^{k}) = x(x-a)\left(\sum_{i=0}^{k-2} x^{k-2-i} a^i\right) + a^{k-1}(x-a) = (x-a)(\left(\sum_{i=0}^{k-2} x^{k-1-i} a^i\right) + a^{k-1})\] \[x^k - a^k = (x-a)\left(\sum_{i=0}^{k-1} x^{k-1-i} a^i\right)\]
Thus, for all $n\in N$, $x^n - a^n = (x-a)\left(\sum_{i=0}^{n-1} x^{n-1-i} a^i\right)$
\end{enumerate}
\end{proof} \vspace{4pt}     \hrule   \vspace{4pt}

\section*{Example 12.8}
\begin{exmp}
\begin{enumerate}
\label{12.8}
\item[a)]Let $n\in \bbN.$ Suppose $f:\bbR\to\bbR$ is given by $f(x)=x^n.$ Use Exercise~\ref{12.7} to prove that $f'(a)=na^{n-1}$, for all $a\in \bbR.$
\vspace{4pt}     \hrule   \vspace{4pt} \begin{proof}
Because $f(x) = x^n$ is a polynomial, then it is continuous, and therefore $\lim\limits_{x\to a} f(x) = a^n$.
Note that $f'(a) = \lim\limits_{h\to 0}\frac{f(a+h)-f(a)}{h} = \lim\limits_{h\to 0}\frac{f(x) - f(a)}{x-a}
\lim\limits_{h\to 0}\frac{(x)^n - a^n}{x-a}$
\begin{enumerate}
    \item If $n=1$, then $f'(a) = \lim\limits_{x\to a}\frac{x^1 - a^1}{x-a} =\lim\limits_{x\to a} 1 = 1 = 1a^{1-1}$
    \item If $n=k$, then assume $f'(a)= ka^{k-1}$. Therefore, by Example \ref{12.7}:
    \[\lim\limits_{x-a\to 0}\frac{x^{k} - a^{k}}{x-a} =
    \lim\limits_{x\to a}\frac{(x-a)(\sum_{i=0}^{k-1}(x)^{k-1-i}a^i)}{h} =  \lim\limits_{x\to a}(\sum_{i=0}^{k-1}(x)^{k-1-i}a^i)= ka^{k-1}\] Thus, $\lim\limits_{x\to a} x^{k-1-i}a^i = a^{k-1}$
    \item If $n=k+1$, then since $f'(a) = \lim\limits_{x\to a}\frac{x^{k+1} - a^{k+1}}{x-a}$. By Example \ref{12.7}:
    \[\lim\limits_{x\to a}\frac{x^{k+1} - a^{k+1}}{x-a} =
    \lim\limits_{x\to a}\frac{(x-a)(\sum_{i=0}^k x^{k-i}a^i)}{x-a}\]
    \[\lim\limits_{x\to a} \sum_{i=0}^k x^{k-i}a^i = \lim\limits_{x\to a} [x\sum_{i=0}^{
    k}(x)^{k-1-i}a^i] = \lim\limits_{x\to a} [x(\sum_{i=0}^{k-1}(x^{k-1-i}a^i) + a^{k-1})]\]
    By the inductive step and Theorem 11.9:
    \[ a(ka^{k-1} + a^{k-1}) = ka^k+a^k = (k+1)a^k\]
\end{enumerate}
\end{proof} \vspace{4pt}     \hrule   \vspace{4pt}
\item[b)] Let $k\in\bbR.$ Prove that if $f:\bbR\to\bbR$ is given by $f(x)=k,$ then $f'(a)=0,$ for all $a\in \bbR.$ 
\end{enumerate}
\vspace{4pt}     \hrule   \vspace{4pt} \begin{proof}
Let $a\in \bbR$. Consider the expression:
\[\lim\limits_{x\to a} \frac{f(x) - f(a)}{x-a} = \lim\limits_{x\to a} \frac{k - k}{x-a} = \lim\limits_{x\to a} 0 = 0\] Thus, $f'(a) = 0$ exists and therefore $f'(a) = 0$ for all $a\in \bbR$
\end{proof} \vspace{4pt}     \hrule   \vspace{4pt}
\end{exmp}

\section*{Example 12.9}
\begin{exmp} 
\label{12.9}
	Suppose that $f\colon A\to\bbR$ and $g\colon A\to\bbR$ are differentiable at $a\in A$. 
	\begin{enumerate}[(a)]
		\item Prove that $(f+g)$ is differentiable at $a$ and compute $(f + g)'(a)$ in terms of $f'(a)$ and $g'(a)$. 
\vspace{4pt}     \hrule   \vspace{4pt}  \begin{proof}:\\
\begin{itemize}
    \item Because $f$ is differentiable at $a$, then by Theorem \ref{12.4}, $f'(a) = \lim\limits_{h\to 0} \frac{f(a+h) - f(h)}{h}$ and by Theorem \ref{12.5}, $f$ is continuous at $a$. 
    \item Because $g$ is differentiable at $a$, then by Theorem \ref{12.4}, $g'(a) = \lim\limits_{h\to 0} \frac{g(a+h) - g(a)}{h}$ and by Theorem \ref{12.5}, $g$ is continuous at $a$.
\end{itemize} 
Therefore, consider:
\[\lim\limits_{h\to 0} \frac{(f+g)(a+h) - (f+g)(a)}{h} = \lim\limits_{h\to 0} \frac{f(a+h) + g(a+h) - (f(a) + g(a))}{h} \]
\[\lim\limits_{h\to 0} \frac{f(a+h) + g(a+h) - (f(a) + g(a))}{h} = \lim\limits_{h\to 0} \frac{(f(a+h) - f(a)) +(g(a+h) - g(a))}{h}\]
\[\lim\limits_{h\to 0} \frac{(f(a+h) - f(a)) +(g(a+h) - g(a))}{h} = \lim\limits_{h\to 0 } (\frac{f(a+h) - f(a)}{h} + \frac{g(a+h) - g(a)}{h})\]

Thus, because both $f$ and $g$ are continuous at $a$, then $f(a+h)$, $g(a+h)$ $f(a)$ $g(a)$, and $h$ are all continuous at $a$. Thus, by Theorem 11.9:\[(f+g)'(a) = \lim\limits_{h\to 0 } \frac{f(a+h) - f(a)}{h} + \lim\limits_{h\to 0 }\frac{g(a+h) - g(a)}{h}\] Because both components on the right hand side exist and equal the derivatives, then $(f+g)'(a) = f'(a) + g'(a)$ exists.
  \end{proof}\vspace{4pt}     \hrule   \vspace{4pt}
		\item Prove that $(fg)$ is differentiable at $a$ and  compute $(fg)'(a)$ in terms of $f(a)$, $g(a)$, $f'(a)$ and $g'(a)$. 
\vspace{4pt}     \hrule   \vspace{4pt}  \begin{proof}:\\
\begin{itemize}
    \item Because $f$ is differentiable at $a$, then by Theorem \ref{12.4}, $f'(a) = \lim\limits_{h\to 0} \frac{f(a+h) - f(h)}{h}$ and by Theorem \ref{12.5}, $f$ is continuous at $a$. 
    \item Because $g$ is differentiable at $a$, then by Theorem \ref{12.4}, $g'(a) = \lim\limits_{h\to 0} \frac{g(a+h) - g(a)}{h}$ and by Theorem \ref{12.5}, $g$ is continuous at $a$.
\end{itemize} 
Therefore, consider:
\[\lim\limits_{h\to 0}\frac{(fg)(a+h) - (fg)(a)}{h}\]
  \begin{align*}
    &=\lim\limits_{h\to 0}\frac{f(a+h)g(a+h) - f(a)g(a)}{h}\\
    &=\lim\limits_{h\to 0}\frac{f(a+h)g(a+h) - f(a)g(a) + (f(a)g(a+h) - f(a)g(a+h))}{h}\\
    &=\lim\limits_{h\to 0}\frac{(f(a+h)g(a+h) - f(a)g(a+h)) + (f(a)g(a+h) - f(a)g(a))}{h}\\
    &=\lim\limits_{h\to 0}\frac{g(a+h)(f(a+h) -f(a)) + f(a)(g(a+h) - g(a))}{h}\\
  \end{align*}
Thus, because both $f$ and $g$ are continuous at $a$, then $f(a+h)$, $g(a+h)$ $f(a)$ $g(a)$, and $h$ are all continuous at $a$. Thus, by Theorem 11.9:
  \begin{align*}
    &=\lim\limits_{h\to 0}\frac{g(a+h)(f(a+h) -f(a))}{h} + \lim\limits_{h\to 0}\frac{f(a)(g(a+h) - g(a))}{h}\\
    &=\lim\limits_{h\to 0}g(a+h)\lim\limits_{h\to 0}\frac{(f(a+h) -f(a))}{h} + \lim\limits_{h\to 0}\frac{f(a)(g(a+h) - g(a))}{h}
  \end{align*}
  By Lemma \ref{12.3}, $\lim\limits_{h\to 0}g(a+h) = \lim\limits_{x\to a}g(x)$. Since $g$ is continuous at $a$, then $\lim\limits_{x\to a}g(x) = g(a)$. Thus:
  \begin{align*}
  &=g(a)\lim\limits_{h\to 0}\frac{(f(a+h) -f(a))}{h} + f(a)\lim\limits_{h\to 0}\frac{(g(a+h) - g(a))}{h}\\
  &=g(a)f'(a)+f(a)g'(a)
  \end{align*}
  Thus, because $g(a), f'(a), f(a)$ and $g'(a)$ exist, then $(fg)'(a)$ exists and equals the above equation.
  \end{proof}\vspace{4pt}     \hrule   \vspace{4pt}
		\item Prove that $\frac{1}{g}$ is differentiable at $a$ (under an appropriate assumption) and compute $(\frac{1}{g})'(a)$ in terms of $g'(a)$ and $g(a)$. What assumption do you need to make? 
\vspace{4pt}     \hrule   \vspace{4pt}  \begin{proof}
By the same reasoning as above (and using 11.9.c for the 4th line below) and considering the case when $g(a) \neq 0$, consider the expression:
\begin{align*}
    &\lim\limits_{h\to 0} \frac{\frac{1}{g(a+h)} - \frac{1}{g(a)}}{h}\\
    &= \lim\limits_{h\to 0} (\frac{\frac{1}{g(a+h)} - \frac{1}{g(a)}}{h}) \frac{g(a+h)g(a)}{g(a+h)g(a)}\\
    &=\lim\limits_{h\to 0} \frac{g(a) - g(a+h)}{h(g(a+h))g(a)}\\
    &=-\lim\limits_{h\to 0} \frac{g(a+h) - g(a))}{h} \cdot \lim\limits_{h\to 0} \frac{1}{g(a+h)g(a)}
\end{align*}
  Note that that last step is valid because $-1$ is a constant and continuous and thus its limit equals itself. By Lemma \ref{12.3}, $\lim\limits_{h\to 0}g(a+h) = \lim\limits_{x\to a}g(x)$. Since $g$ is continuous at $a$, then $\lim\limits_{x\to a}g(x) = g(a)$. Thus:
  \begin{align*}
     &=  -\lim\limits_{h\to 0} \frac{g(a+h) - g(a))}{h} \cdot \frac{1}{g(a)g(a)}\\
      &= \frac{-g'(a)}{(g(a))^2}
  \end{align*}
   If $g(a)\neq 0$, then $(\frac{1}{g})'(a)$ exists. Thus, because $g(a)$ and $g'(a)$ exist, then $(\frac{1}{g})'(a)$ exists and equals the above equation.
\end{proof}\vspace{4pt}     \hrule   \vspace{4pt}
		\item Prove that $\frac{f}{g}$ is differentiable at $a$ (under an appropriate assumption) and compute $(\frac{f}{g})'(a)$ in terms of $f(a)$, $g(a)$, $f'(a)$ and $g'(a)$. What assumption do you need to make?
\vspace{4pt}     \hrule   \vspace{4pt}  \begin{proof}:\\
Same reasoning and considerations as above (same theorems are used to prove that we can "split the limits").
\begin{align*}
&\lim\limits_{h\to 0}\frac{\frac{f(a+h)}{g(a+h)} - \frac{f(a)}{g(a)}}{h}\\
&= \lim\limits_{h\to 0}(\frac{\frac{f(a+h)}{g(a+h)} - \frac{f(a)}{g(a)}}{h})\frac{g(a+h)g(a)}{g(a+h)g(a)}\\
&= \lim\limits_{h\to 0}\frac{f(a+h)g(a) - f(a)g(a+h)}{h(g(a+h)g(a))}\\
&= \lim\limits_{h\to 0}\frac{f(a+h)g(a) - f(a+h)g(a+h) + f(a+h)g(a+h)-f(a)g(a+h)}{h(g(a+h)g(a))}\\
&= \lim\limits_{h\to 0}\frac{f(a+h)(g(a) - g(a+h)) + g(a+h)(f(a+h)-f(a))}{h} \lim\limits_{h\to 0}\frac{1}{g(a+h)g(a)}\\
&= (-f(a)\lim\limits_{h\to 0}\frac{g(a+h) - g(a)}{h} + g(a)\lim\limits_{h\to 0}\frac{(f(a+h)-f(a))}{h}) \lim\limits_{h\to 0}\frac{1}{g(a+h)g(a)}\\
&= (-f(a)g'(a) + g(a)f'(a)) \frac{1}{g(a)g(a)}\\
&= \frac{f'(a)g(a) - f(a)g'(a)}{(g(a))^2}
\end{align*}
\end{proof}\vspace{4pt}     \hrule   \vspace{4pt}
\vspace{4pt}     \hrule   \vspace{4pt}  \begin{proof}
Alternate Proof (The lame non-algebra way):\\ Same considerations as above:
\begin{align*}
(\frac{f}{g})'(a) &= (\frac{1}{g})'(a)f(a) + (\frac{1}{g})(a)f'(a)\\
(\frac{f}{g})'(a) &=  (\frac{-g'(a)}{(g(a))^2})f(a) + \frac{f'(a)}{g(a)}\\
(\frac{f}{g})'(a) &=  (\frac{-g'(a)f(a)}{(g(a))^2}) + \frac{f'(a)g(a)}{g(a)^2}\\
(\frac{f}{g})'(a) &=  (\frac{f'(a)g(a)-g'(a)f(a)}{(g(a))^2})
\end{align*}
\end{proof}\vspace{4pt}     \hrule   \vspace{4pt}
	\end{enumerate}
\end{exmp}

\section*{Theorem 12.10: Chain Rule}
\begin{thm} \label{12.10}

	Let $a \in A$, $g\colon A\to\bbR$ and $f\colon I\to \bbR,$ where $I$ is an interval containing $g(A).$  Suppose that $g$ is differentiable at $a$ and $f$ is differentiable at $g(a)$.
	Then $f \circ g$ is differentiable at $a$ and:
	\[
		(f \circ g)'(a) = f'(g(a)) \cdot g'(a).
	\]
\end{thm}
\vspace{4pt}     \hrule   \vspace{4pt}\begin{proof}
Define $\varphi: I \to \bbR$ such that \[
		\phi(y) =
		\begin{cases}
			\dfrac{f(y) - f(g(a))}{y - g(a)} \quad &\text{if $y\neq g(a)$, } \\
			f'(g(a)) \quad &\text{if $y = g(a)$.} 
		\end{cases}
	\]  Consider \[\lim\limits_{x\to a}\frac{f(g(x)) - f(g(a))}{x-a}\] Either $g(x) = g(a)$ or $g(x) \neq g(a)$.
\begin{enumerate} [i]
    \item If $g(x) = g(a)$, then \[\lim\limits_{x\to a}\frac{f(g(a)) - f(g(a))}{x-a} = \lim\limits_{x\to a}\frac{0}{x-a} = 0 = \lim\limits_{x\to a}\varphi(g(x)\cdot \frac{g(x) - g(a)}{x-a}\] Note that this last equality is valid because both expressions are defined by the construction of $\varphi$
    \item  If $g(x) \neq g(a)$, then \[\lim\limits_{x\to a}\frac{f(g(x)) - f(g(a))}{x-a} = \lim\limits_{x\to a}[\frac{f(g(x)) - f(g(a))}{g(x) - g(a)}\cdot \frac{g(x) - g(a)}{x-a}] = \lim\limits_{x\to a}\varphi(g(x)) \cdot \frac{g(x) - g(a)}{x-a}\]
\end{enumerate}
Thus, in any case, \[\lim\limits_{x\to a}\frac{f(g(a)) - f(g(a))}{x-a} = \lim\limits_{x\to a} (\varphi(g(x))) \cdot \frac{g(x) - g(a)}{x-a})\] Because $g(x)$ is differentiable and continuous, then this is equivalent to $\lim\limits_{x\to a}\varphi(g(x)) \cdot g'(a)$. Since $f$ is differentiable at $g(a)$. Thus, \[f'(g(a)) = \lim\limits_{y\to g(a)}\frac{f(y) - f(g(a))}{y-g(a)} = \lim\limits_{y\to g(a)}\varphi(y) = \varphi(g(a))\] Thus, $\varphi$ is continuous at $g(a)$.
Therefore, by Theorem 11.13, because $g$ is continuous at $a$: \[\lim\limits_{x\to a}\varphi(g(x)) = \varphi(g(a)) = f'(g(a))\] And thus:
\[\lim\limits_{x\to a}\varphi(g(x)) \cdot g'(a) = f'(g(a)) \cdot g'(a)\] exists and is equal to $(f\circ g)'(a)$
\end{proof}\vspace{4pt}     \hrule   \vspace{4pt}

We now come to the most important theorem in differential calculus, Corollary~\ref{12.15}.

\section*{Definition 12.11}
\begin{defn} \label{12.11}
	Let $f\colon A \to \bbR$ be a function. If $f(a)$ is the last point of $f(A)$, then $f(a)$ is called the \emph{maximum value} of $f$. If $f(a)$ is the first point of $f(A)$, then $f(a)$ is the \emph{minimum value} of $f$. We say that $f(a)$ is a \emph{local maximum value} of $f$ if there exists a region $R$ containing $a$ such that $f(a)$ is the last point of $f(A \cap R)$. We say that $f(a)$ is a \emph{local minimum value} of $f$ if there exists a region $R$ containing $a$ such that $f(a)$ is the first point of $f(A \cap R)$.
\end{defn}

\section*{Remark 12.12}
\begin{rem}
	Equivalently, $f(a)$ is a \emph{local maximum} (resp.\ \emph{minimum}) \emph{value} of $f$ if there exists $U$ open \emph{in $A$} such that $f(a)$ is the last (resp.\ first) point of $f(U)$.
 \end{rem}

\section*{Theorem 12.13}
\begin{thm}
\label{12.13}
Let $f\colon A \to \bbR$ be differentiable at $a$. Suppose that $f(a)$ is a local maximum or local minimum value of $f$. Then $f'(a) = 0$.
\end{thm}

\vspace{4pt}     \hrule   \vspace{4pt}\begin{proof}
\textbf{Lemma: 12.13 Supplement} If $f: A \to \bbR$ is differentiable at $a$, then if $R$ is a region containing $a$, then $f|_{A\cap R}: A\cap R \to \bbR$ is differentiable at $a$. 
\begin{proof}
    Since $f$ is differentiable at $a$, then $\lim\limits_{x\to a}\frac{f(x)-f(a)}{x-a} = f'(a)$. Thus, $a\in LP(A)$ and for all $\epsilon>0$, there exists a $\delta>0$ such that if $x\in A$ and $0<|x-a|<\delta$, then $|\frac{f(x) - f(a)}{x-a} - f'(a)|< \epsilon$. Since $a\in LP(A)$ and $a\in R$, then $a\in LP(R)$ since $R$ is a region. Thus, as a corollary to Theorem 3.25, $a\in LP(A\cap R)$. Moreover, if $x\in (A\cap R)$, then $f|_{A\cap R}(x) = f(x)$. Thus, if $x\in A\cap R$, then $f'|_{A\cap R}(x) = \lim\limits_{x\to a}\frac{f|_{A\cap R}(x) - f|_{A\cap R}(a)}{x-a} = \lim\limits_{x\to a}\frac{f(x) - f(a)}{x-a} = f'(a)$
\end{proof}
\textit{Unfortunately, I didn't actually end up using this Lemma for this proof, but it's still a cool argument that comes up later!}\\\\
Assume, for the sake of contradiction, that if $f(a)$ is a local maximum or local minimum value of $f$, then $f'(a) \neq 0$:
\begin{enumerate}
    \item If $f(a)$ is a local maximum of $f$, then:
    \begin{enumerate}
        \item Consider the case when $f'(a)>0$. Because $f'(a)$ exists, then $f'(a) = \lim\limits_{x\to a^+}\frac{f(x) - f(a)}{x-a}$. Therefore, there exists a $\delta>0$ such that if $x\in A$ and $0<x-a<\delta$, then $|\frac{f(x)- f(a)}{x-a} - f'(a)| < f'(a)$. Therefore, it follows that $\frac{f(x) - f(a)}{x-a} >0$ and $x-a>0$. Thus, it must hold that $f(x) - f(a) > 0$ and so $f(x)>f(a)$. However, because $a\in A$ and $A$ is open, then $A\cap R$ is nonempty (Theorem 3.25), so for some $x\in (A\cap R)$, because $f(a)$ is a local maximum, then $f(x) \leq f(a)$, which is a contradiction. 
        \item Consider the case when $f'(a)< 0$. Because $f'(a)$ exists, then $f'(a) =  \lim\limits_{x\to a^+}\frac{f(x) - f(a)}{x-a}$. Therefore, there exists a $\delta>0$ such that if $x\in A$ and $0<x-a<\delta$, then $|\frac{f(x)- f(a)}{x-a} - f'(a)| < -f'(a)$. Thus, it must be true that $\frac{f(x) - f(a)}{x-a}>0$ and $x-a>0$, and so it must hold that $f(x) - f(a) > 0$. Thus, $f(x) > f(a)$ for some $x\in A$. However, because $a\in A$ and $A$ is open, then $A\cap R$ is nonempty (Theorem 3.25), so for some $x\in (A\cap R)$, because $f(a)$ is a local maximum, then $f(x) \leq f(a)$, which is a contradiction.  
    \end{enumerate}
    Thus, $f'(a) = 0$ if $f(a)$ is a local maximum of $f$.
    \item If $f(a)$ is a local minimum of $f$, then similar logic using the left hand limit applies. 
    \end{enumerate}
\end{proof}\vspace{4pt}     \hrule   \vspace{4pt}

\section*{Theorem 12.14: Rolle Theorem}
\begin{thm}\label{12.14}
	Suppose that $f\colon [a, b] \to \bbR$ is continuous, differentiable on $(a, b)$ and that $f(a) = f(b)$. Then there exists a point $\lambda \in (a, b)$ such that $f'(\lambda) = 0$.
\end{thm}
\vspace{4pt}     \hrule   \vspace{4pt} \begin{proof}:\\
By Example 10.21, there exists a point $p \in [a,b]$ such that $f(p)\geq f(x)$ for all $x\in [a,b]$. Either $p \in (a,b)$ or $p = a$ or $p = b$. Note that if $p=a$ or $p=b$, then since $f(p) = f(a) = f(b)$, then $f(a) = f(b) \geq f(x)$ for all $x\in [a,b]$. Similarly, note that there exists a point $c\in [a,b]$ such that $f(c)\leq f(x)$ and similar logic as the maximum applies.
\begin{enumerate}
    \item If $p \in (a,b)$, then because $f(p) \geq f(x)$ for all $x\in [a,b]$, then since $(a,b)\subset [a,b]$, then $f(p) \geq f(x)$ for all $x\in (a,b)$. Thus, since $(a,b) = (a,b) \cap [a,b]$, and $f(p)$ is the last point of $f((a,b) \cap [a,b])$, then $f(p)$ is a local maximum and thus by Theorem \ref{12.13}, $f'(p) = 0$. Thus, let $\lambda = p$ and therefore $f'(\lambda) = 0$ for some $\lambda \in (a,b)$.
    \item If $c \in (a,b)$, then by similar logic, $f(c)$ is a local minimum and so if we let $\lambda = c$, then $f
    (\lambda) = 0$ for some $\lambda \in (a,b)$.
    \item If $p,c \notin (a,b)$, then since $f(a) = f(b)\leq f(x)$ and $f(a) = f(b) \geq f(x)$ for all $x\in [a,b]$, then $f(a) = f(b) = f(x)$ for all $x\in [a,b]$. Thus, for any $x \in [a,b]$, $f(x) = f(\lambda)$, where $\lambda \in (a,b)$. Thus, $f(x) - f(\lambda) = 0$, and so $f'(\lambda) = \lim\limits_{x\to \lambda} \frac{f(x) - f(\lambda)}{x-\lambda} = \lim\limits_{x\to \lambda}\frac{0}{x-\lambda} = 0$. 
\end{enumerate}
Thus, in any case, there exists a $\lambda\in [a,b]$ such that $f'(\lambda) = 0$. 
\end{proof}\vspace{4pt}     \hrule   \vspace{4pt}

\section*{Corollary 12.15: Mean Value Theorem}
\begin{cor}
	\label{12.15}
	Suppose that $f\colon [a, b] \to \bbR$ is continuous on $[a, b]$ and differentiable on $(a, b)$. Then there exists a point $\lambda \in (a, b)$ such that:
	\[
		f(b) - f(a) = f'(\lambda) (b - a).
	\]
\end{cor}
\vspace{4pt}     \hrule   \vspace{4pt} \begin{proof}:\\
Consider that the slope between $a,b$ is:
\[\frac{f(b) - f(a)}{b-a}\] Thus, the equation of the line connect $f(a), f(b)$ is:
\[g(x) = \frac{f(b)-f(a)}{b-a}(x-a) + f(a)\] Note that $g(x)$ is continuous on $[a,b]$ because $f$ is continuous at $a,b$ and by Corollary 11.12, polynomials are continuous. Thus, if \[h(x) = (f(x) - \frac{f(b)-f(a)}{b-a}(x-a) + f(a)) = f(x) - g(x)\] then $h(a) = f(a) - f(a) = 0$ and $h(b) = f(b) - (f(b) - f(a) + f(a)) = 0$. Thus, $h(a) = h(b)$. Moreover, because $f,g$ are continuous on $[a,b]$, then by Corollary 11.10, $h$ is continuous on $[a,b]$. Because $f$ is differentiable on $(a,b)$, then by Example \ref{12.9}, $h$ is differentiable on $(a,b)$. Thus, by Rolle's theorem, there exists a $\lambda \in (a,b)$ such that $h'(\lambda) = 0$. Thus, by Example \ref{12.9}, since \[h'(x) = f'(x) - g'(x)\] then $h'(\lambda) = 0 = f'(\lambda) - \frac{f(b) - f(a)}{b-a}$ implies that \[f'(\lambda) = \frac{f(b) - f(a)}{b-a}\]
\end{proof}\vspace{4pt}     \hrule   \vspace{4pt}

\section*{Corollary 12.16}
\begin{cor}
	\label{12.16}
	Suppose that $f\colon [a, b] \to \bbR$ is continuous on $[a, b]$ and differentiable on $(a, b)$. Then,
	\begin{enumerate}[(a)]
		\item If $f'(x)>0$ for all $x \in (a, b), $ then $f$ is strictly increasing on $[a, b]$.
		\item If $f'(x)<0$ for all $x \in (a, b), $ then $f$ is strictly decreasing on $[a, b]$.
		\item If $f'(x) = 0$ for all $x \in (a, b), $ then $f$ is constant on $[a, b]$.
	\end{enumerate}
\end{cor}
\vspace{4pt}     \hrule   \vspace{4pt} \begin{proof}:\\
If $[y,z]\subset [a,b]$, then because $f|_{[y,z]}:[y,z]\to \bbR$ is the restriction of $f$ to $[y,z]$, then by Proposition 9.7, because $f$ is continuous, then $f|_{[y,z]}$ is continuous on $[y,z]$. Moreover, because $f$ is differentiable on $(a,b)$ (that is, for all $c\in (a,b)$, $f'(c)$ exists), and $(y,z)\subset (a,b)$, then $f$ is differentiable on $(y,z)$ (Lemma 12.13.Supplement). Thus, assume, for the sake of contradiction, that: 
\begin{enumerate}
    \item If $f'(x)>0$ for all $x\in (a,b)$, then $f$ is not strictly increasing. Thus, if $y,z\in (a,b)$ with $y<z$, then $f(y)\geq f(z)$. Thus, because $f$ is continuous on $[y,z]$ and differentiable on $(y,z)$, then there exists some $\lambda \in (y,z)$ such that $\frac{f(z) - f(y)}{z-a} = f'(\lambda)$. However, because $f(z) - f(y)\leq 0$ and $(z-y)>0$, then \[\frac{f(z) - f(y)}{z-y}  = f'(\lambda)\leq 0\] which is a contradiction, since $f'(\lambda)>0$ for all $\lambda \in (a,b)$ 
    \item If $f'(x)<0$ for all $x\in (a,b)$, then $f$ is not strictly decreasing. Thus, if $y,z\in (a,b)$ with $y<z$, then $f(y)\leq f(z)$. Thus, because $f$ is continuous on $[y,z]$ and differentiable on $(y,z)$, then there exists some $\lambda \in (y,z)$ such that $\frac{f(z) - f(y)}{z-a} = f'(\lambda)$. However, because $f(z) - f(y)\geq 0$ and $(z-y)>0$, then \[\frac{f(z) - f(y)}{z-a}  = f'(\lambda)\geq 0\] which is a contradiction, since $f'(\lambda)<0$ for all $\lambda \in (a,b)$ 
    \item If $f'(x) = 0$ for all $x\in (a,b)$, then $f$ is not constant, Thus, if $y,z\in (a,b)$ with $y<z$, then $f(y)\neq f(z)$. Thus, because $f$ is continuous on $[y,z]$ and differentiable on $(y,z)$, then there exists some $\lambda \in (y,z)$ such that $\frac{f(z) - f(y)}{z-a} = f'(\lambda)$. However, because $f(z) - f(y)\neq 0$ and $(z-y)>0$, then \[\frac{f(z) - f(y)}{z-a}  = f'(\lambda)\neq 0\] which is a contradiction, since $f'(\lambda)=0$ for all $\lambda \in (a,b)$ 
\end{enumerate}
\end{proof}\vspace{4pt}     \hrule   \vspace{4pt}

\section*{Remark 12.17}
\begin{rem}\label{12.17}
	Corollary \ref{12.16} also holds if instead of $[a, b]$ we have an arbitrary interval $I$, and instead of $(a, b)$ we have the interior of $I$.\end{rem} 

 \section*{Corollary 12.18}
\begin{cor}
\label{12.18}
	Suppose that $f\colon [a, b] \to \bbR$ and $g\colon [a, b] \to \bbR$ are continuous on $[a, b]$, differentiable on $(a, b)$, and $f'(x) = g'(x)$ for all $x \in (a, b)$. Then there is some $c\in\bbR$ such that for all $x \in [a, b]$, we have $f(x) = g(x) + c$.
\end{cor} 
\vspace{4pt}     \hrule   \vspace{4pt} \begin{proof}:\\
Because $f$ and $g$ are continuous on $[x,b]$ where $x\in [a,b)$ and $f$ and $g$ are differentiable on $(x,b)$, then by Corollary \ref{12.15}, there exists a $\lambda\in (x,b)$ such that \[f'(\lambda) = \frac{f(b) - f(x)}{b-x}\] \[g'(\lambda) = \frac{g(b) - g(x)}{b-x}\] Note that because $(x,b)\subset (a,b)$, and for all $\lambda \in (a,b)$, $f'(\lambda) = g'(\lambda)$, then:
\[\frac{f(b) - f(x)}{b-x} = \frac{g(b) - g(x)}{b-x}\]
\[f(b) - f(x) = g(b) - g(x)\]
\[(f(b) - g(b)) + g(x) = f(x)\]
Thus, if $c=f(b) - g(b)$, then $f(x) = g(x) + c$\\
\textbf{Although this proof is incorrect, I will keep it in here, as it is a good reminder of what MVT isn't (Hint: Are those $\lambda$ truly the same in any case?). Also note that a cool corollary to see would be the converse of the MVT with a few modifications. Look below at "Correct 12.18" Proof below.}
\end{proof}\vspace{4pt}     \hrule   \vspace{4pt}
\vspace{4pt}     \hrule   \vspace{4pt} \begin{proof} Correct Proof\\
Becuase $g'(x) = f'(x)$, then by Theorem 11.10, because $f$ and $g$ are continuous on $[a,b]$, then $h(x) = f(x) - g(x)$ is continuous on $[a,b]$. Similarly, by Theorem \ref{12.9}, because $f$ and $g$ are differentiable on $(a,b)$, then $h'(x) = f'(x) - g'(x) = 0$ is differentiable on $(a,b)$. Thus, because $h'(x) = 0$, then $h$ is constant on $[a,b]$, and therefore, $h(x) = c$ for all $x \in [a,b]$. Therefore, $c = f(x) - g(x)$ implies that $f(x) = g(x) + c$.
\end{proof}\vspace{4pt}     \hrule   \vspace{4pt}

\section*{Corollary 12.19}
\begin{cor}	
\label{12.19}
	Suppose that $f\colon [a, b] \to \bbR$ and $g\colon[a, b] \to \bbR$ are continuous on $[a, b]$ and differentiable on $(a, b)$. Then there is a point $\lambda \in (a, b)$ such that:
	\[
		(f(b)-f(a))g'(\lambda) = (g(b)-g(a))f'(\lambda).
	\]
\end{cor}
\vspace{4pt}     \hrule   \vspace{4pt} \begin{proof}:\\
Consider \[\pi(x) := f(x)(g(b) - g(a)) - g(x)(f(b) - f(a))\]
It follows that \begin{align*}
\pi(a) &= f(a)(g(b) - g(a)) - g(a)(f(b) - f(a))\\
\pi(a) &= f(a)g(b) - f(a)g(a) - g(a)f(b) + g(a)f(a)\\
\pi(a) &= f(a)g(b) - f(b)g(a)
\end{align*}
and 
\begin{align*}
\pi(b) &= f(b)(g(b) - g(a)) - g(b)(f(b) - f(a))\\
\pi(b) &= f(b)g(b) - f(b)g(a) - g(b)f(b) + g(b)f(a)\\
\pi(b) &= f(a)g(b) - f(b)g(a)
\end{align*}
Thus, because $\pi(a) = \pi(b)$, then since $f$ and $g$ are continuous, then $\pi$ is continuous and so by Theorem \ref{12.14}, there exists a $\lambda\in (a,b)$ such that \begin{align*}
    \pi'(\lambda) &= 0\\
    \pi(\lambda) &= f'(\lambda)(g(b) - f(a)) - g'(\lambda)(f(b) - f(a))\\
    f'(\lambda)(g(b) - f(a)) &= g'(\lambda)(f(b) - f(a))
\end{align*}
\end{proof}\vspace{4pt}     \hrule   \vspace{4pt}

Finally, we prove another very important theorem that tells us about inverse functions and their derivatives. 

\section*{Theorem 12.20}
\begin{thm}\label{12.20}
Suppose that $f:(a,b)\to\bbR$ is differentiable and that the derivative $f':(a,b)\to\bbR$ is continuous. Also suppose that there is a point $p\in(a,b)$ such that $f'(p)\neq 0$. 
Then there exists a region $R\subset (a,b)$ such that $p\in R$ and $f$ with domain restricted to $R$ is injective. 
Furthermore, $f^{-1}:f(R)\to R$ is differentiable at the point $f(p)$ and
$$ (f^{-1})'(f(p))=\frac{1}{f'(p)} .$$
\end{thm} 
\vspace{4pt}     \hrule   \vspace{4pt} \begin{proof}:\\
\begin{itemize}
    \item WLOG, let $f'(p)>0$, then because $f'(p)$ exists, then by Lemma 11.8, there exists a region $R = (a,b) \cap (p-\delta, p+\delta)$ such that for all $x\in R$, $f'(x) > \frac{f'(p)}{2} >0$. Thus, by Corollary, 12.16, because $f'$ is continuous and $f'(x)>0$ for all $x\in R$, then $f|_{R'\cap (a,b)}$ is strictly increasing on $R$ and thus injective. It follows that by Theorem 9.15 $f^{-1}: f(R) \to R$ is continuous.
    \item Define $\varphi: R \to f(R)$ such that \[
		\phi(x) =
		\begin{cases}
			\frac{x-p}{f(x) - f(p)} \quad &\text{if $x\neq p$, } \\
			\frac{1}{f'(p)} \quad &\text{if $x = p$.} 
		\end{cases}
	\]
 Note that because $f$ is injective, then if $f(x) = f(p)$, then $x = p$. 
    Thus, because $f'(p)$ exists, then \[\lim\limits_{x\to p}\varphi(x) = \lim\limits_{x\to p}\frac{x-p}{f(x) - f(p)} = \lim\limits_{x\to p}\frac{1}{\frac{f(x) - f(p)}{x - p}} = \frac{1}{f'(p)}\] Thus, because \[\lim\limits_{x\to p}\varphi(x) = \varphi(p)\] then $\varphi(x)$ is continuous at $f(p)$. Thus, because $f^{-1}$ is continuous, then because $\lim\limits_{y\to f(p)} f^{-1}(x) = p$ and $\varphi$ is continuous at $p$, then by Remark 11.14:
	\[\lim\limits_{y\to f(p)} \frac{f^{-1}(y) - f^{-1}(f(p))}{f(f^{-1}(y)) - f(p)} = \lim\limits_{y\to f(p)}\varphi(f^{-1}(y)) = \varphi(\lim\limits_{y\to f(p)}f^{-1}(y) = \varphi(f^{-1}(f(p))) = \frac{1}{f'(p)}\] 
Thus, because $f'(p) \neq 0$, then $\lim\limits_{y\to f(p)} \frac{f^{-1}(y) - f^{-1}(f(p))}{f(f^{-1}(y)) - f(p)} =\lim\limits_{y\to f(p)} \frac{f^{-1}(y) - f^{-1}(f(p))}{y
- f(p)}$ exists and equals $f^{-1}'(x)$
\end{itemize}
\end{proof}\vspace{4pt}     \hrule   \vspace{4pt}

\section*{Exercise 12.21}
\begin{exmp} \label{12.21}
Consider the function $f(x)=x^n$ for a fixed $n\in\bbN$.  Show that if $n$ is even, then $f$ is strictly increasing
on the set of non-negative real numbers and that if $n$ is odd, then $f$ is strictly increasing on all of $\bbR$.
For a given $n$, let $A$ be the aforementioned set on which $f$ is strictly increasing.  Define the
inverse function $f^{-1}:f(A)\rightarrow A$ by $f^{-1}(x)=\sqrt[n]{x}$, which we sometimes also
denote $f^{-1}(x)=x^{1/n}$.  Use Theorem\ \ref{12.20} to find the points $y\in f(A)$ at
which $f^{-1}$ is differentiable, and determine $(f^{-1})'(y)$ at these points.
\end{exmp}
\vspace{4pt}     \hrule   \vspace{4pt} \begin{proof}:\\
Because $f$ is a polynomial, then by Corollary 11.12, $f$ is continuous. Also, by Example \ref{12.8}, $f'(x) = nx^{n-1}$ exists and since $n-1\geq 0$, then $f'(x)$ is continuous as well. 
\begin{itemize}
    \item If $n$ is even, then $n=2k$ for some $k\in \bbN$:
    \begin{enumerate}
        \item If $x>0$, then $f'(x) = nx^{n-1}$, then because $n, x^{n-1}>0$, then $f'(x)>0$ for all $x\in \bbR_+$.
    \end{enumerate}
    Thus, because $f'(x)>0$ for all $x\in (0, \infty)$, then by Corollary \ref{12.16}, $f$ is strictly increasing on $[0,\infty)$
    \item If $n$ is odd, then $n = 2k-1$ for some $k\in \bbN$:
    \begin{enumerate}
        \item If $x>0$, then by the same reason as above, $f'(x)>0$.
        \item If $x<0$, then because $f'(x) = (2k-1)x^{2k-2} = (2k-1)(\frac{x^{2k}}{x^2})$. Thus, because $2k=n$ is even then $x^n$ and $x^2$ are positive. Thus, $f'(x)>0$. 
    \end{enumerate}
    Thus, because $f'(x)>0$ for all $x\in \bbR\setminus\{0\}$, then by Corollary \ref{12.16}, $f$ is strictly increasing on $(-\infty, 0]$ and $[0,\infty)$.
    Assume, for the sake of contradiction, that $f$ is not strictly increasing at $x = 0$. Thus, if $x, 0 \in \bbR$, such that $x<0$, then $f(x)\geq f(0)$. Also, if $0,y \in \bbR$ with $0<y$, then $f(0) \geq f(y)$. Thus, $f(x) \geq f(y)$, which is a contradiction, since $x,y \in \bbR\setminus\{0\}$ with $x<y$ and thus since $f$ is strictly increasing on $\bbR\setminus\{0\}$, then $f(x) < f(y)$. Thus, $f$ is strictly increasing on all $\bbR$.   
\end{itemize}
\begin{itemize}
    \item If $n$ is even, then $A = [0,\infty)$, and thus, because $f'(x) >0$ for all $x\in A\setminus\{0\}$, then $f'(x) \neq 0$. Moreover, if $y\in f(A\setminus\{0\})$, then $y\neq 0$ for all $y\in f(A)$, and $f^{-1}(y)$ is differentiable on $A\setminus\{0\}$ and thus, by Theorem \ref{12.20},
\[(f^{-1})'(y) = \frac{1}{f'((f^{-1})(y))} = \frac{1}{\frac{1}{n}(y^{\frac{1}{n}})^{{n-1}})} = \frac{1}{n}y^{\frac{1-n}{n}}\]
\item If $n$ is odd, then $A = \bbR \setminus\{0\}$ and the same result as above appears.
\end{itemize}
Note: I am not including as a separate case when $n-1 = 0$, as $0^0$ might be undefined or might be equal to $1$. If it's equal to $1$, then for $n=1$, one does not have to subtract $0$ from $A$, since $f'(x) \neq 0$ for any $x\in A$. 
\end{proof}\vspace{4pt}     \hrule   \vspace{4pt}

\section*{Acknowledgments} 
Thanks, as always, to Professor Oron Propp for being a great mentor in both Office Hours and during class. Thanks also to Victor Hugo Almendra Hernández for being an amazing TA and resource. Also big preash to Jake Zumo for being a cool TA. Also thanks to all my great peers for presenting proofs during class.
\begin{thebibliography}{9}




\end{thebibliography}

\end{document}

