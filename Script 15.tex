
\documentclass[openany, amssymb, psamsfonts]{amsart}
\usepackage{mathrsfs,comment}
\usepackage[usenames,dvipsnames]{color}
\usepackage[normalem]{ulem}
\usepackage{url}
\usepackage{tikz}
\usepackage{tkz-euclide}
\usepackage{lipsum}
\usepackage{marvosym}
\usepackage[all,arc,2cell]{xy}
\UseAllTwocells
\usepackage{enumerate}
\newcommand{\bA}{\mathbf{A}}
\newcommand{\bB}{\mathbf{B}}
\newcommand{\bC}{\mathbf{C}}
\newcommand{\bD}{\mathbf{D}}
\newcommand{\bE}{\mathbf{E}}
\newcommand{\bF}{\mathbf{F}}
\newcommand{\bG}{\mathbf{G}}
\newcommand{\bH}{\mathbf{H}}
\newcommand{\bI}{\mathbf{I}}
\newcommand{\bJ}{\mathbf{J}}
\newcommand{\bK}{\mathbf{K}}
\newcommand{\bL}{\mathbf{L}}
\newcommand{\bM}{\mathbf{M}}
\newcommand{\bN}{\mathbf{N}}
\newcommand{\bO}{\mathbf{O}}
\newcommand{\bP}{\mathbf{P}}
\newcommand{\bQ}{\mathbf{Q}}
\newcommand{\bR}{\mathbf{R}}
\newcommand{\bS}{\mathbf{S}}
\newcommand{\bT}{\mathbf{T}}
\newcommand{\bU}{\mathbf{U}}
\newcommand{\bV}{\mathbf{V}}
\newcommand{\bW}{\mathbf{W}}
\newcommand{\bX}{\mathbf{X}}
\newcommand{\bY}{\mathbf{Y}}
\newcommand{\bZ}{\mathbf{Z}}

%% blackboard bold math capitals
\newcommand{\bbA}{\mathbb{A}}
\newcommand{\bbB}{\mathbb{B}}
\newcommand{\bbC}{\mathbb{C}}
\newcommand{\bbD}{\mathbb{D}}
\newcommand{\bbE}{\mathbb{E}}
\newcommand{\bbF}{\mathbb{F}}
\newcommand{\bbG}{\mathbb{G}}
\newcommand{\bbH}{\mathbb{H}}
\newcommand{\bbI}{\mathbb{I}}
\newcommand{\bbJ}{\mathbb{J}}
\newcommand{\bbK}{\mathbb{K}}
\newcommand{\bbL}{\mathbb{L}}
\newcommand{\bbM}{\mathbb{M}}
\newcommand{\bbN}{\mathbb{N}}
\newcommand{\bbO}{\mathbb{O}}
\newcommand{\bbP}{\mathbb{P}}
\newcommand{\bbQ}{\mathbb{Q}}
\newcommand{\bbR}{\mathbb{R}}
\newcommand{\bbS}{\mathbb{S}}
\newcommand{\bbT}{\mathbb{T}}
\newcommand{\bbU}{\mathbb{U}}
\newcommand{\bbV}{\mathbb{V}}
\newcommand{\bbW}{\mathbb{W}}
\newcommand{\bbX}{\mathbb{X}}
\newcommand{\bbY}{\mathbb{Y}}
\newcommand{\bbZ}{\mathbb{Z}}

%% script math capitals
\newcommand{\sA}{\mathscr{A}}
\newcommand{\sB}{\mathscr{B}}
\newcommand{\sC}{\mathscr{C}}
\newcommand{\sD}{\mathscr{D}}
\newcommand{\sE}{\mathscr{E}}
\newcommand{\sF}{\mathscr{F}}
\newcommand{\sG}{\mathscr{G}}
\newcommand{\sH}{\mathscr{H}}
\newcommand{\sI}{\mathscr{I}}
\newcommand{\sJ}{\mathscr{J}}
\newcommand{\sK}{\mathscr{K}}
\newcommand{\sL}{\mathscr{L}}
\newcommand{\sM}{\mathscr{M}}
\newcommand{\sN}{\mathscr{N}}
\newcommand{\sO}{\mathscr{O}}
\newcommand{\sP}{\mathscr{P}}
\newcommand{\sQ}{\mathscr{Q}}
\newcommand{\sR}{\mathscr{R}}
\newcommand{\sS}{\mathscr{S}}
\newcommand{\sT}{\mathscr{T}}
\newcommand{\sU}{\mathscr{U}}
\newcommand{\sV}{\mathscr{V}}
\newcommand{\sW}{\mathscr{W}}
\newcommand{\sX}{\mathscr{X}}
\newcommand{\sY}{\mathscr{Y}}
\newcommand{\sZ}{\mathscr{Z}}


\renewcommand{\phi}{\varphi}
\renewcommand{\emptyset}{\O}

\newcommand{\abs}[1]{\lvert #1 \rvert}
\newcommand{\norm}[1]{\lVert #1 \rVert}
\newcommand{\sm}{\setminus}


\newcommand{\sarr}{\rightarrow}
\newcommand{\arr}{\longrightarrow}

\newcommand{\hide}[1]{{\color{red} #1}} % for instructor version
%\newcommand{\hide}[1]{} % for student version
\newcommand{\com}[1]{{\color{blue} #1}} % for instructor version
%\newcommand{\com}[1]{} % for student version
\newcommand{\meta}[1]{{\color{green} #1}} % for making notes about the script that are not intended to end up in the script
%\newcommand{\meta}[1]{} % for removing meta comments in the script

\DeclareMathOperator{\ext}{ext}
\DeclareMathOperator{\ho}{hole}
%%% hyperref stuff is taken from AGT style file
\usepackage{hyperref}  
\hypersetup{%
  bookmarksnumbered=true,%
  bookmarks=true,%
  colorlinks=true,%
  linkcolor=blue,%
  citecolor=blue,%
  filecolor=blue,%
  menucolor=blue,%
  pagecolor=blue,%
  urlcolor=blue,%
  pdfnewwindow=true,%
  pdfstartview=FitBH}   
  
\let\fullref\autoref
%
%  \autoref is very crude.  It uses counters to distinguish environments
%  so that if say {lemma} uses the {theorem} counter, then autrorefs
%  which should come out Lemma X.Y in fact come out Theorem X.Y.  To
%  correct this give each its own counter eg:
%                 \newtheorem{theorem}{Theorem}[section]
%                 \newtheorem{lemma}{Lemma}[section]
%  and then equate the counters by commands like:
%                 \makeatletter
%                   \let\c@lemma\c@theorem
%                  \makeatother
%
%  To work correctly the environment name must have a corrresponding 
%  \XXXautorefname defined.  The following command does the job:
%
\def\makeautorefname#1#2{\expandafter\def\csname#1autorefname\endcsname{#2}}
%
%  Some standard autorefnames.  If the environment name for an autoref 
%  you need is not listed below, add a similar line to your TeX file:
%  
%\makeautorefname{equation}{Equation}%
\def\equationautorefname~#1\null{(#1)\null}
\makeautorefname{footnote}{footnote}%
\makeautorefname{item}{item}%
\makeautorefname{figure}{Figure}%
\makeautorefname{table}{Table}%
\makeautorefname{part}{Part}%
\makeautorefname{appendix}{Appendix}%
\makeautorefname{chapter}{Chapter}%
\makeautorefname{section}{Section}%
\makeautorefname{subsection}{Section}%
\makeautorefname{subsubsection}{Section}%
\makeautorefname{theorem}{Theorem}%
\makeautorefname{thm}{Theorem}%
\makeautorefname{excercise}{Exercise}%
\makeautorefname{cor}{Corollary}%
\makeautorefname{lem}{Lemma}%
\makeautorefname{prop}{Proposition}%
\makeautorefname{pro}{Property}
\makeautorefname{conj}{Conjecture}%
\makeautorefname{defn}{Definition}%
\makeautorefname{notn}{Notation}
\makeautorefname{notns}{Notations}
\makeautorefname{rem}{Remark}%
\makeautorefname{quest}{Question}%
\makeautorefname{exmp}{Example}%
\makeautorefname{ax}{Axiom}%
\makeautorefname{claim}{Claim}%
\makeautorefname{ass}{Assumption}%
\makeautorefname{asss}{Assumptions}%
\makeautorefname{con}{Construction}%
\makeautorefname{prob}{Problem}%
\makeautorefname{warn}{Warning}%
\makeautorefname{obs}{Observation}%
\makeautorefname{conv}{Convention}%


%
%                  *** End of hyperref stuff ***

%theoremstyle{plain} --- default
\newtheorem{thm}{Theorem}[section]
\newtheorem{cor}{Corollary}[section]
\newtheorem{exercise}{Exercise}
\newtheorem{prop}{Proposition}[section]
\newtheorem{lem}{Lemma}[section]
\newtheorem{prob}{Problem}[section]
\newtheorem{conj}{Conjecture}[section]
%\newtheorem{ass}{Assumption}[section]
%\newtheorem{asses}{Assumptions}[section]

\theoremstyle{definition}
\newtheorem{defn}{Definition}[section]
\newtheorem{ass}{Assumption}[section]
\newtheorem{asss}{Assumptions}[section]
\newtheorem{ax}{Axiom}[section]
\newtheorem{con}{Construction}[section]
\newtheorem{exmp}{Example}[section]
\newtheorem{notn}{Notation}[section]
\newtheorem{notns}{Notations}[section]
\newtheorem{pro}{Property}[section]
\newtheorem{quest}{Question}[section]
\newtheorem{rem}{Remark}[section]
\newtheorem{warn}{Warning}[section]
\newtheorem{sch}{Scholium}[section]
\newtheorem{obs}{Observation}[section]
\newtheorem{conv}{Convention}[section]

%%%% hack to get fullref working correctly
\makeatletter
\let\c@obs=\c@thm
\let\c@cor=\c@thm
\let\c@prop=\c@thm
\let\c@lem=\c@thm
\let\c@prob=\c@thm
\let\c@con=\c@thm
\let\c@conj=\c@thm
\let\c@defn=\c@thm
\let\c@notn=\c@thm
\let\c@notns=\c@thm
\let\c@exmp=\c@thm
\let\c@ax=\c@thm
\let\c@pro=\c@thm
\let\c@ass=\c@thm
\let\c@warn=\c@thm
\let\c@rem=\c@thm
\let\c@sch=\c@thm
\let\c@equation\c@thm
\numberwithin{equation}{section}
\makeatother

\bibliographystyle{plain}

%--------Meta Data: Fill in your info------
\title{University of Chicago Calculus IBL Course}

\author{Agustin Esteva}

\date{Apr 12. 2024}

\begin{document}

\begin{abstract}

16310's Script 15.\\ Let me know if you see any errors! Contact me at aesteva@uchicago.edu.


\end{abstract}

\maketitle

\tableofcontents

\setcounter{section}{15}

\section*{Definition 15.1}
\begin{defn}
\label{15.1}
	A \emph{sequence} (of real numbers) is a function $a\colon \bbN \to \bbR$.
\end{defn}

By setting $a_n = a(n)$, we can think of a sequence $a$ as a list $a_1, a_2, a_3, \dotsc$ of real numbers.  We use the notation $(a_n)_{n=1}^{\infty}$ for such a sequence, or if there is no possibility of confusion, we sometimes abbreviate this and write simply $(a_n)$.  More generally, we also use the term sequence to refer to a function defined on $\{n \in \bbN\cup\{0\} \mid n\geq n_0\}$ for any fixed $n_0 \in \bbN\cup\{0\}$.  We write $(a_n)_{n = n_{0}}^{\infty}$ for such a sequence.

\section*{Definition 15.2}
\begin{defn}  
\label{15.2}
	We say that a sequence $(a_n)$ \emph{converges} to a point $p \in \bbR$ if, for every open interval $I$ containing $p$, there exists $N\in \bbN$ such that if $n\geq N$, then $a_n\in I$. If a sequence converges to some point, we say it is \emph{convergent}. If $(a_n)$ does not converge to any point, we say that the sequence \emph{diverges} or is \emph{divergent}.
\end{defn}

\textbf{Notes on Notation:}
\vspace{4pt}     \hrule   \vspace{4pt}
For the rest of the script, I will adopt the convention that $\{a_n | n\in \bbN\}$ is the image of the entire sequence. Thus, if $(a_n) = (1,1,\dots,$ then $\{a_n | n\in \bbN\} = \{1\}.$ If I want to refer to a sub-list of the sequence (remember that a sequence is just a list of numbers), then I will use the notation $(a_r)$ for some $r\in \bbN.$ Thus, back to our example, if I say $(a_r)_{r\leq 3}$ then $(a_r)_{r\leq 3}= 1,1,1.$ This notation will come useful later on. Note that for these finite sub-lists, I will exclusively use the $r$ subscript, as to not get them confused with sub-sequences. 
\vspace{4pt}     \hrule   \vspace{4pt}

\section*{Example 15.3}
\begin{exmp}
\label{15.3}
	Show that a sequence $(a_n)$ converges to $p$ if and only if any region containing $p$ contains all but finitely many terms of the sequence.
\end{exmp}
\vspace{4pt}     \hrule   \vspace{4pt} \begin{proof}:\\
\begin{enumerate}
\item ($\implies$:) If $(a_n)$ converges to $p,$ then for every open interval $I$ containing $p,$ there exists $N \in \bbN$ such that if $n\geq N,$ then $a_n \in I.$ Thus, if $R$ is an open set containing $p,$ then because $R$ is an open interval, then $R$ contains every $a_n$ such that $n\geq N.$ Let $(a_r)_{r<N}$ be the list of terms that may not be contained in $R,$\footnote{look at the note above on notation. As an example, if $(a_n) = \frac{1}{n},$ and $R = (-\frac{1}{2}, \frac{1}{2}),$ then $R$ there exists $N = 3$ such that if $n\geq N,$ then $(a_n) \in R.$ In this case, $(a_r)_{r<3} = a_1, a_2 = 1,\frac{1}{2}.$} then because $|(a_r)_{r<N}| = N-1,$ then $(a_r)_{r<N}$ is finite. Therefore, because the list of terms of $a_n$ that are not contained in $R$ has cardinality less than or equal to $n-1,$ then $R$ contains all but finitely many points of the sequence. \footnote{Note that the edge case, regions such that there exists no terms in the sequence outside of the region, are contained in this proof. The cardinality of $(a_r)_{r<N} = (a_r)_{r<1}$ would just be zero, which is still a finite number of terms.}
\item ($\impliedby$:) Let $R$ be a region containing $p$ and $(a_n)$ be a sequence. If $R$ contains all but finitely many points of the sequence, then because a sequence has infinite amount of terms, then $R$ contains an infinite number of $(a_n).$ Let $N\in \bbR$ correspond to the highest index of $(a_N)\notin R.$\footnote{ i.e, if $(a_n) = 1,-1,-2,1,-1,1,1,1,\dots$ and $R = (\frac{1}{2}, \frac{1}{2}),$ then $N = 5,$ as $5$ corresponds to the highest index for a point in $(a_n)\notin R.$} Thus, if $n\geq N,$ then $(a_n)\in R.$ Such $(a_n)$ must exist because $(a_n)$ is infinite. Therefore, $(a_n)$ converges to $p.$
\end{enumerate}
\footnote{This proof wasn't the most rigorous or had the best notation, but this Example is mostly just used as an insight about sequences.}
\end{proof}\vspace{4pt}     \hrule   \vspace{4pt} 

\section*{Theorem 15.4}
\begin{thm}
\label{15.4}
	Suppose that $(a_n)$ converges both to $p$ and to $p'$.
	Then $p = p'$.
\end{thm}
\vspace{4pt}     \hrule   \vspace{4pt} \begin{proof}:\\
Assume, for the sake of contradiction, that $p \neq p'.$ Without loss of generality, let $p<p'.$ Therefore, there exists some $q \in \bbR$ such that $p< q < p'$ and thus $(-\infty,q) \cap (q, \infty) = \emptyset.$ Because $(-\infty,q)$ is an open interval containing $p,$ and $(a_n)$ converges to $p,$ then then there exists some $N_1\in \bbN$ such that if $n\geq \bbN$, then $a_{n} \in (\infty,q).$ Similarly, because $(q,\infty)$ is an open interval containing $p',$ then there exists some $N_2\in \bbN$ such that if $n\geq N,$ then $a_{n}\in (q,\infty).$ Let $N = \max(N_1, N_2),$ then for all $n\geq N,$ $a_n \in (-\infty, q)$ and $a_n \in (q, \infty),$ which is a contradiction, since the sets are disjoint. 
\end{proof} \vspace{4pt}     \hrule   \vspace{4pt} 

\section*{Definition 15.5}
\begin{defn}
\label{15.5}
	If a sequence $(a_n)$ converges to $p \in \bbR$, we call $p$ the \emph{limit} of $(a_n)$ and write
	\[
		\lim_{n \to \infty} a_n = p.
	\]
\end{defn}

\section*{Example 15.6}
\begin{exmp} 
	\label{15.6}
	Which of the following sequences $(a_n)$ converge?  Which diverge? For each that converges, what is the limit? Give proofs for your answers.
		\begin{enumerate}[(a)]
			\item $a_n = 5$.
   \vspace{4pt}     \hrule   \vspace{4pt} \begin{proof}:\\
   Let $I$ be an interval containing $5.$ If $n \geq 1,$ then $a_n = 5$ and thus $a_n \in I.$ Therefore, because $a_n \in I$ for any $I$ containing $5,$ then $\displaystyle\lim_{n \to \infty} = 5.$
   \end{proof}\vspace{4pt}     \hrule   \vspace{4pt}
			\item $a_n = n$.
   \vspace{4pt}     \hrule   \vspace{4pt} \begin{proof}:\\
   Assume, for the sake of contradiction, that $\displaystyle\lim_{n \to \infty} = p.$ Therefore, if $R = (p-\frac{1}{4}, p + \frac{1}{4})$ is a region containing $p,$ then there exists some $N\in \bbN$ such that if $n\geq N,$ then $a_n \in R.$ Thus, since $a_{N} =N \in R,$ then consider $n = N+1.$ Because $a_{N+1} = N+1,$ then $a_{N+1}\notin R,$ which is a contradiction. Thus, $\displaystyle\lim_{n\to \infty}a_n$ does not exist. 
   \end{proof}\vspace{4pt}     \hrule   \vspace{4pt}
			\item $a_n = \frac{1}{n}$.
    \vspace{4pt}     \hrule   \vspace{4pt} \begin{proof}:\\
    Consider that for any open interval $I$ containing $0,$ there exists some region $(a,b)\subset I$ containing $0.$ Therefore, by the Archimidean property, there exists some $N\in \bbN$ such that $\frac{1}{N}< b$ and thus, $a_N \in (a,b)$ and so $a_N \in I.$ Therefore, if $N\leq n,$ then $\frac{1}{n} \leq \frac{1}{N},$ and so $a_n = \frac{1}{n} \in (a,b).$ Thus, $\displaystyle\lim_{n \to \infty}a_n = 0.$
    \end{proof}\vspace{4pt}     \hrule   \vspace{4pt}
			\item $a_n = (-1)^n$.
		\end{enumerate}
\end{exmp}
\vspace{4pt}     \hrule   \vspace{4pt}\begin{proof}:\\
\footnote{We claim that \[(-1)^n = \begin{cases} 1 \quad \text{n is even} \\ -1 \quad \text{n is odd}
\end{cases}\]
\begin{enumerate}
    \item If $n = 1,$ then $(-1)^1 = -1.$
    \item If $n = 2k+1,$ where $k\in \bbN,$ then assume $(-1)^n = (-1)^{2k+1} = -1.$
    \item If $n = 2(k+1) +1,$ then $n = 2k+3.$ Thus, consider that $(-1)^n = (-1)^{2k+3} = (-1)^{2k+1}(-1)(-1) = (-1)^{2k+1},$ by our inductive hypothesis, $(-1)^{2k+1} = -1.$
\end{enumerate}
If $n$ is even, then $(-1)^n = (-1)^2k = (-1)^{2k+1 - 1} = (-1)^{2k-1}(-1).$ By the proof above, $(-1)^{2k+1} = (-1)$ and $(-1)^{-1} = -1.$ Thus, $(-1)^2k = -1.$}
Assume, for the sake of contradiction, that $\displaystyle\lim_{n\to \infty} a_n = p.$ Therefore, for all regions containing $p,$ there exists some $N \in \bbN$ such that if $N\leq n,$ then $a_n \in R.$ However, consider that $(p-1, p+ 1)$ is a region containing $p.$ Note that either $a_n = 1$ or $a_n = -1.$ Assume that $-1$ and $1$ are both in $(p-1, p+1).$ Thus
\[p-1 < -1 < p+1; \qquad -2<p<0\] and \[p-1 < 1 < p+1; \qquad 0<p<2\] which is a contradiction, since $0<p<0.$ Therefore, if $a_N \in R,$ then $a_{N+1}\notin R,$ which is a contradiction.
\end{proof}\vspace{4pt}     \hrule   \vspace{4pt}

\section*{Theorem 15.7}
\begin{thm}
	\label{15.7}
	A sequence $(a_n)$ converges to $p\in\bbR$ if, and only if, for all $\epsilon>0$, there is some $N\in\bbN$ such that for all $n \geq N$ we have $\abs{a_n - p} < \epsilon$. 
\end{thm}
\vspace{4pt}     \hrule   \vspace{4pt}\begin{proof}:\\
\begin{enumerate}
    \item ($\implies$:) Consider that because $(a_n)$ converges to $p,$ then for any $\epsilon>0,$ the region $(p-\epsilon, p+ \epsilon)$ contains every $a_n$ such that $n\geq N.$ Because $p-\epsilon< a_n < p + \epsilon,$ then $|a_n - p|< \epsilon.$
    \item ($\impliedby$:) If $I$ is an open interval containing $p,$ then there exists some $\epsilon>0$ such that $(p- \epsilon, p+ \epsilon)\subset I.$ Therefore, there exists some $N \in \bbN$ such that for all $n\geq N,$ $|a_n - p|< \epsilon,$ and therefore $a_n \in I,$ Therefor, $(a_n)$ converges to $p.$
\end{enumerate}
\end{proof}      \vspace{4pt}     \hrule   \vspace{4pt}

\textbf{Bernoulli's Inequality: If $x>-1,$ then $(1 + x)^n \geq 1 + nx$ for any $n \in \bbN$}
\vspace{4pt}     \hrule   \vspace{4pt}\begin{proof}:\\
\begin{enumerate}
\item If $n = 0,$ then $1=(1+x)^0 = 1 + (0)x.$ 
\item If $n =k,$ where $k\in \bbN,$ then assume $(1+x)^k \geq 1 + kx.$ Note that $1+kx> 0$ since $x>-1$
\item If $n = k+1,$ then by the inductive hypothesis;
\begin{align*}
(1+x)^{k+1} &= (1+x)(1+x)^k\\
&\geq (1+x)(1+kx)\\
&= 1 + x + kx + kx^2\\
\tag{$kx^2\geq 0$}&\geq 1 + x + kx\\
&= 1 + x(k+1)
\end{align*}
\end{enumerate}
Therefore, if $1+x = |y|,$ then $|y|^n \geq 1 + n(|y|-1).$
\end{proof}\vspace{4pt}     \hrule   \vspace{4pt}

\section*{Example 15.8}
\begin{exmp} :\\
\label{15.8}
	\begin{enumerate}[(a)]
		\item Prove that $\displaystyle \lim_{n \to \infty} \frac{(-1)^n}{n} = 0$.
  \vspace{4pt}     \hrule   \vspace{4pt}\begin{proof}:\\
  Let $\epsilon>0.$ By the Archimidean property, there exists some $N$ such that  $\frac{1}{N}< \epsilon.$ Let $n\geq N,$ it follows that $\frac{1}{n}\leq \frac{1}{N}< \epsilon.$ Moreover, note that $|(-1)^n| = 1$ for any $n\in \bbN.$ Therefore:
  \begin{align*}
  |a_n - 0| &= |\frac{(-1)^n}{n} - 0|\\
  &= |\frac{(-1)^n}{n}|\\
  &= |\frac{1}{n}||(-1)^n|\\
  &\leq \frac{1}{n}\\
  &< \epsilon *1
  \end{align*}
  \end{proof}  \vspace{4pt}     \hrule   \vspace{4pt}
		\item Let $x \in \bbR$ with $\abs{x} < 1$. Prove that $\displaystyle \lim_{n \to \infty} x^n = 0$. 
  \vspace{4pt}     \hrule   \vspace{4pt} \begin{proof}:\\
\begin{enumerate}
\item If $x = 0,$ then Figure 2 (not shown here) makes it clear that $\displaystyle\lim_{x\to \infty}x^n = 0.$ \footnote{How Agustin felt after this joke:\newline \url{https://th.bing.com/th/id/OIP.Yr9jOj9dcErjBcKd3CN-QgHaFj?rs=1&pid=ImgDetMain}}
\item If $x\neq 0,$ then since $|x|<1,$ then $\frac{1}{|x|}>1.$ Thus, by Bernoulli's inequality, $\frac{1}{|x|^n} \geq n(\frac{1}{|x|}-1) + 1.$ Thus, $|x^n|\leq \frac{1}{n(\frac{1}{|x|}-1) + 1}<\frac{1}{n(\frac{1}{|x|}-1)}$ For all $\epsilon>0,$ there exists an $N\in \bbN$ such that $\frac{1}{N}\leq \epsilon(\frac{1}{|x|}-1),$\footnote{This is a positive real number for any $|x|<1$} and thus, if $n\geq N,$ then 
\begin{align*}
|a_n - 0| &= |x^n|\\
&< \frac{1}{n(\frac{1}{|x|}-1)}\\
&< \frac{\epsilon(\frac{1}{|x|}-1)}{(\frac{1}{|x|}-1)}\\
&= \epsilon
\end{align*}
\end{enumerate}

  \end{proof}   \vspace{4pt}     \hrule   \vspace{4pt}
		\item Let $x\in\bbR$ with $\abs{x}>1$. Prove that $(x^n)$ is divergent.
		{\it Hint for b) and c): Use Bernoulli's inequality (Sheet 0, exmp 3) to show that if $|y|>1$, then $|y|^n\geq n(|y|-1)+1$. }
\vspace{4pt}     \hrule   \vspace{4pt} \begin{proof}:\\
Let $\epsilon = 1.$ Let $N\in \bbN$ and $p\in \bbR.$ By the Archimidean property, there exists an $n' \in \bbN$ such that $n'>\frac{|p|}{|x|-1}.$ Because $|x|>1,$ then there exists some $n'' \in \bbN$ such that $|x^{n''}|>|p|.$ Therefore, if $n \geq \max(N,n', n''),$ then $|x-1|> \frac{|p|}{n},$ and so:
\begin{align*}
|x^n - p|&\geq ||x^n| - |p||\\
&= ||x^n| - |p||\\
\tag{$|x^{n''}|>|p|$}&= |x^n| - |p|\\
\tag{$x^n>0$}&= x^n - |p|\\
&\geq n(|x|-1) +1 - |p|\\
&> n\frac{|p|}{n} +1 - |p||\\
&= |p| +1 - |p||\\
&= 1
\end{align*}
Therefore, because there exists some $\epsilon >0$ such that for all $N \in \bbN,$ if $n\geq N,$ $|x^n - p|\not < \epsilon,$ then $(x^n)$ is divergent.
\end{proof}\vspace{4pt}     \hrule   \vspace{4pt} 
	\end{enumerate}
\end{exmp}

\section*{Theorem 15.9}
\begin{thm}
\label{15.9}
	If $\displaystyle \lim_{n\to \infty}a_n=a$ and $\displaystyle \lim_{n\to\infty} b_n=b$ both exist, then 
	\begin{enumerate}[(a)]
		\item \quad $\displaystyle\lim_{n\to \infty}(a_n+b_n) = \lim_{n\to\infty} a_n+\lim_{n\to\infty} b_n$.
  \vspace{4pt}     \hrule   \vspace{4pt}\begin{proof}:\\
  Let $\epsilon>0.$ Because $\lim_{n\to \infty} a_n = a$ and $\lim_{n\to \infty}b_n = b,$ then:
  \begin{enumerate}
  \item There exists some $N_a \in \bbN$ such that if $n>N_a,$ $|a_n - a|< \frac{\epsilon}{2}.$
  \item There exists some $N_b \in \bbN$ such that if $n>N_b,$ $|b_n - b|< \frac{\epsilon}{2}.$
  \end{enumerate}
  Therefore, let $N = \max(N_a, N_b),$ and thus for all $\epsilon>0:$
  \begin{align*}
  |a_n + b_n  - (a+b)|&= |a_n - a + b_n - b|\\
  &\leq |a_n - a| + |b_n - b|\\
  &<\frac{\epsilon}{2} + \frac{\epsilon}{2}\\
  &= \epsilon
  \end{align*}
  And thus, $\lim_{n\to \infty}(a_n + b_n) = \lim_{n\to \infty}a_n + \lim_{n\to \infty}b_n$
  \end{proof} \vspace{4pt}     \hrule   \vspace{4pt}
		\item \quad $\displaystyle\lim_{n\to \infty} (a_n\cdot b_n)  =  \left( \lim_{n\to \infty} a_n \right)\cdot \left(\lim_{n\to \infty} b_n \right)$.
 \vspace{4pt}     \hrule   \vspace{4pt} \begin{proof}:\\
  Let $\epsilon>0.$ Because $\lim_{n\to \infty} a_n = a$ and $\lim_{n\to \infty}b_n = b,$ then:
  \begin{enumerate}
  \item There exists some $N_a \in \bbN$ such that if $n>N_a,$ $|a_n - a|< \frac{\epsilon}{|b| + |a|+1}.$
  \item There exists some $N_b \in \bbN$ such that if $n>N_b,$ $|b_n - b|< \frac{\epsilon}{|b| + |a|+1}.$
  \item There exists some $N_0 \in \bbN$ such that if $n>N_0,$ $|a_n - a|< 1.$\footnote{If $|a_n - a|<1,$ then $|a_n|< |a|+1$ \begin{proof} Because $|a_n - a|< 1,$ then $a -1< a_n < a+1.$ If $|a_n|<0,$ then $|a_n|< |a-1|\leq |a| + 1.$ If $|a_n\geq 0,$ then $|a_n|< |a+1|= |a| + 1.$\end{proof}}
  \end{enumerate}
  Therefore, let $N = \max(N_a, N_b, N_0),$ and thus for all $\epsilon>0:$
  \begin{align*}
  |a_nb_n - ab| &= |a_nb_n + a_nb - a_nb -ab|\\
  &= |a_n(b_n - b) + b(a_n - a)|\\
  &\leq |a_n(b_n - b)| + |b(a_n - a)|\\
  &= |a_n||(b_n - b)| + |b||(a_n - a)|\\
  &< |a_n|\frac{\epsilon}{|b| + |a|+1} + |b|\frac{\epsilon}{|b| + |a|+1}\\
  &<\epsilon(\frac{|a_n| + |b|}{|b|+|a|+1})\\
\tag{footnote 7}  &< \epsilon(\frac{|b| + |a| + 1}{|b| + |a| + 1})\\
  &= \epsilon
  \end{align*}
  \end{proof}\vspace{4pt}     \hrule   \vspace{4pt}
	\item Moreover, if $\displaystyle\lim_{n\to \infty} b_n\neq 0$, $\quad$ $\displaystyle\lim_{n\to \infty}\displaystyle\frac{a_n}{b_n}= \frac{\displaystyle\lim_{n\to \infty}a_n}{\displaystyle\lim_{n\to \infty} b_n}$.
 \vspace{4pt}     \hrule   \vspace{4pt} \begin{proof}:\\
\textbf{Claim: if $\displaystyle\lim_{n\to \infty}(b_n) = b\neq 0$, then $\displaystyle\lim_{n\to \infty}\frac{1}{b_n} = \displaystyle\frac{1}{b}$}
\begin{proof}:\\
Let $\epsilon>0.$  Because $(b_n)$ converges to $b,$ then there exists some $N \in \bbN$ such that for all $n\geq N.$ then $|b_n - b|< \min(\frac{|b|}{2}, \frac{b^2\epsilon}{2}).$ Therefore, 
\begin{align*}
\frac{|b|}{2} > |b_n - b|\geq ||b_n| - |b|| 
\end{align*}
Thus, $|b_n|> \frac{|b|}{2}>0,$ and thus $\frac{1}{|b_n|}< \frac{2}{|b|}.$
Let $\epsilon>0.$ Because $\displaystyle\lim_{n\to \infty} a_n = a$ and $\displaystyle\lim_{n\to \infty}b_n = b,$ then:
\begin{align*}
|\frac{1}{b_n}  - \frac{1}{b}| &= |\frac{b_n - b}{b_nb}|\\
&= |\frac{1}{b_nb}||b_n-b|\\
&<|\frac{2}{b^2}|\frac{b^2 \epsilon}{2}\\
&= \frac{2}{b^2}\frac{b^2 \epsilon}{2}\\
&= \epsilon
\end{align*}
\end{proof}
Thus, by part b, \[\displaystyle\lim_{n\to \infty}\frac{a_n}{b_n} = \displaystyle\lim_{n\to \infty}a_n \cdot \displaystyle\lim_{n\to \infty}\frac{1}{b_n} = \frac{a}{b} = \frac{\displaystyle\lim_{n\to \infty}a_n}{\displaystyle\lim_{n\to \infty}b_n}\]
 \end{proof}\vspace{4pt}     \hrule   \vspace{4pt}
\end{enumerate}

\end{thm}

\section*{Example 15.10}
\begin{exmp}\label{15.10}
	Which of the following sequences $(a_n)$ converge?  Which diverge? For each that converges, what is the limit? Give proofs for your answers.
		\begin{enumerate}[(a)]
			\item $a_n = (-1)^n \cdot n$.
\vspace{4pt}     \hrule   \vspace{4pt} \begin{proof}:\\
Assume, for the sake of contradiction, that $(a_n)$ converges. Thus, because $(b_n) = \frac{1}{n}$ converges, then $(a_n)(b_n)$ converges. However, because $(a_n)(b_n) = (-1)^nn\frac{1}{n} = (-1)^n,$ then it does not converge, which is a contradiction.
\end{proof} \vspace{4pt}     \hrule   \vspace{4pt}
			\item $\! a_n = \frac{1}{n^2 + 1}(2+\frac{1}{n})$
\vspace{4pt}     \hrule   \vspace{4pt} \begin{proof}:\\
Let $b_n = \frac{1}{n^2 +1}.$ For all $\epsilon>0,$ there exists an $N\in \bbN$ such that $\frac{1}{N}< \epsilon.$ Thus, if $n\geq N$, then:
\begin{align*}
|\frac{1}{n^2 +1} - 0| &= |\frac{1}{n^2 +1}|\\
&< \frac{1}{n}\\
&< \epsilon
\end{align*}
Thus, $\lim_{n\to \infty}b_n = 0.$\newline Let $c_n = (2+ \frac{1}{n}).$ For all $\epsilon>0, $ there exists an $N\in \bbN$ such that $\frac{1}{N}< \epsilon.$ Thus, if $n\geq N:$
\begin{align*}
|2+ \frac{1}{n} - 2|&= |\frac{1}{n}|\\
&< \epsilon
\end{align*}
Therefore, $\lim_{n\to \infty}c_n = 2.$ Therefore, by Theorem \ref{15.9}, $\lim_{n\to\infty}(a_n) = \lim_{n\to\infty}(b_nc_n) = \lim_{n\to\infty}(b_n) \dot \lim_{n\to\infty}(c_n) = 0.$
\end{proof}
\begin{proof} \textbf{Alternate Proof}:\\
Once it is proved that $(b_n) = \frac{1}{n^2+1}$ converges to $0$ (which can be shown quickly by proving that $\frac{\frac{1}{n^2}}{\frac{1}{n^2} + 1}$ converges) then consider that because $(c_n) = 2$ and $(d_n) = \frac{1}{n}$ both converge to $2$ and $0,$ respectively, then: \[\displaystyle\lim_{n\to \infty}(2+\frac{1}{n}) =\lim_{n\to \infty}2 + \lim_{n\to \infty}\frac{1}{n} = 2 \]
\end{proof}\vspace{4pt}     \hrule   \vspace{4pt}

			\item $a_n=\frac{5n+1}{2n+3}$.
\vspace{4pt}     \hrule   \vspace{4pt} \begin{proof}:\\
For all $\epsilon>0,$ there exists an $N \in \bbN$ such that $\frac{1}{N}<\frac{\epsilon}{8}$ Thus, if $n\geq N,$ then:
\begin{align*}
|a_n - \frac{5}{2}| &= |\frac{5n+1}{2n+2}- \frac{5}{2}|\\
&=|\frac{10n+2}{4n+4}- \frac{10n+10}{4n+4}|\\
&= |\frac{-8}{4n+4}|\\
&\leq |\frac{-8}{n}|\\
&= \frac{8}{n}\\
&< 8 * \frac{\epsilon}{8}\\
&= \epsilon
\end{align*}
\end{proof}
\begin{proof}\textbf{Alternate Proof}:\\
    Consider that $(a_n) = \frac{5 + \frac{1}{n}}{2 + \frac{3}{n}}.$ Thus, if $(b_n) = 5 + \frac{1}{n},$ then as shown in the alternate proof above, $\displaystyle\lim_{n\to \infty} = 5.$ Similarly, $\displaystyle\lim_{n\to \infty}(c_n) = \displaystyle\lim_{n\to\infty}(2 + \frac{3}{n}) =2.$ Thus, since $(c_n)$ doesn't converge to $0,$ then \[\lim_{n\to \infty}(a_n) = \frac{\lim_{n\to \infty}(b_n)}{\lim_{n\to \infty}(c_n)} = \frac{5}{2}.\]
\end{proof}\vspace{4pt}     \hrule   \vspace{4pt}
\begin{proof}:\\
$\frac{5 + \frac{2}{n}}{2 + \frac{2}{n}}$
\end{proof}
			\item $a_n=\frac{(-1)^n+1}{n}$.
		\end{enumerate}
\end{exmp}
\vspace{4pt}     \hrule   \vspace{4pt}\begin{proof}:\\
For all $\epsilon>0,$ there exists an $N\in \bbN$ such that $\frac{1}{N}<\frac{\epsilon}{2}$ Thus, if $n\geq N,$ then:
\begin{align*}
|a_n - 0| &= |\frac{(-1)^n+1}{n}|\\
&= |\frac{1}{n}||(-1)^n+1|\\
&= 2\frac{1}{n}\\
&< 2* \frac{\epsilon}{2}\\
&= \epsilon
\end{align*}
\end{proof}\begin{proof}\textbf{Alternate Proof}:\\
    Consider that $(a_n) = \frac{(-1)^n  +1}{n} = \frac{(-1)^n}{n} + \frac{1}{n}.$ It has been shown that $\displaystyle_{n\to \infty}\frac{(-1)^n}{n} = 0$ and $\displaystyle\lim_{n\to\infty}\frac{1}{n} = 0.$ Therefore, $\displaystyle\lim_{n\to \infty}(a_n) = 0.$
\end{proof}\vspace{4pt}     \hrule   \vspace{4pt}


We've used the word ``limit'' in two contexts now: the limit points of a set, and the limit of a sequence. The defns of these two terms may seem similar. Is there a formal connection? thm~\ref{thm:closure sequence} alludes to an answer. 

\section*{Theorem 15.11}
\begin{thm}
	\label{15.11}
	Let $A \subset \bbR$. Then $p \in \overline{A}$ if and only if there exists a sequence $(a_n)$, with each $a_n \in A$, that converges to $p$.
\end{thm}
\vspace{4pt}     \hrule   \vspace{4pt} \begin{proof}:\\
\begin{itemize}
\item ($\implies$:) If $p \in \overline{A},$ then either $p\in A$ or $p\in LP(A):$
\begin{itemize}
\item If $p\in A,$ then there exists the sequence $(a_n) = p.$ Therefore, all $a_n \in A$ and moreover, for all regions containing $p,$ there exists a $N = 1\in \bbN$ such that if $n\geq 1,$ then $a_n = p \in R.$ Thus, $\lim_{n\to \infty}(a_n) = p.$
\item If $p\in LP(A),$ then for all regions, $R,$ containing $p,$ $R\cap A\setminus\{p\}\neq \emptyset.$ Thus, if $R_n = (p-\frac{1}{n}, p+\frac{1}{n})$ is a region containing $p,$ where $n\in \bbN$, then define $a: \bbN \to \bbR$ such that $a_n: = \max(R_n \cap A \setminus\{p\}).$ Note that because for every $n\in \bbN,$ the region $R$ contains some $x\in A$ such that $x\neq p,$ then $a$ is well defined. Let $(a,b)$ be a region containing $p.$ Thus, there exists some $\delta\in \bbR$ such that $(p-\delta, p+ \delta) \subset (a,b).$ Thus, there exists a $N \in \bbN$ such that $\frac{1}{N} < \delta.$ Thus, if $n\geq N,$ then $\frac{1}{n}\leq \frac{1}{N}$ and so $(p-\frac{1}{n}, p+\frac{1}{n})\subset (a,b).$ Therefore, because $a_n = \max((p-\frac{1}{n}, p+\frac{1}{n}) \cap A\setminus\{p\}),$\footnote{Ultimately, I decided on this $\max$ function to make my life a bit easier. It is not entirely correct to use this, as there are infinite $a_n$ in this region, but this idea is useful, as it creates one output for each input. Alternatively, the Axiom of Choice could have been invoked in choosing these $a_n,$ but the Axiom of Choice has not been defined in this class.} then $a_n \in (a,b).$ Thus, all $a_n \in A$ and $\lim_{n\to \infty}(a_n) = p.$
\end{itemize}
\item ($\impliedby$:) If there exists a sequence $(a_n),$ with each $a_n \in A$ that converge to $p,$ then assume $p\notin \overline{A}.$ Thus,  $p\notin A$ and $p\notin LP(A).$ If $p\notin LP(A),$ then there exists a region, $R,$ containing $p,$ such that $R \cap A\setminus\{p\} = \emptyset.$ However, because $R$ is a region containing $p,$ then because $\lim_{n\to \infty}(a_n) = p,$ then there must exist some $N \in \bbN$ such that if $n\geq N$, then $a_n \in R.$ However, because $a_n \in A,$ then $R \cap A\setminus\{p\} \neq \emptyset,$ which is a contradiction. Therefore, $p\in \overline{A}.$
\end{itemize}
\vspace{4pt}     \hrule   \vspace{4pt} \end{proof}

\section*{Definition 15.12}
\begin{defn}
\label{15.12}
	A sequence $(a_n)$ is \emph{bounded} if its image $\{a_n \mid n \in \bbN\}$ is bounded.
\end{defn}

\section*{Theorem 15.13}
\begin{thm}
\label{15.13}
	Every convergent sequence is bounded.
\end{thm}
\vspace{4pt}     \hrule   \vspace{4pt} \begin{proof}:\\
Let $(a_n)$ be a sequence that converges to some $p\in \bbR$. Thus, there exists a $N\in \bbN$ such that for all $n\geq N$ $|a_n - p|< \epsilon.$ Therefore, if $n\geq N,$ then $p-\epsilon<a_n < p + \epsilon.$ By Example \ref{15.3}, $(a_r)_{r\leq N}$ is a finite list of terms in $a_n$ that may not be contained in $(p-\epsilon, p+\epsilon).$ Because $(a_r)_{r\leq N}$ is finite, then $A = \{a_n|r<N\}$ is finite, as $A$ can have at most $N-1$ elements. Let $m = \text{lower bound}(A)<p-\epsilon$ and $M = \text{upper bound}(A)>p+\epsilon$ Thus, $(a_n)$ is bounded below by $m$ and is bounded above by $M.$
\end{proof}\vspace{4pt}     \hrule   \vspace{4pt}


The converse is not true\footnote{$a_n = (-1)^n$}, but there are two important partial converses. For the first, Theorem~\ref{thm:increasing converges}, we recall defn 8.16 along with defn 15.1, which say that $(a_n)$ is an increasing sequence if $a_n\leq a_{n+1},$ for all $n\in \bbN,$ and  $(a_n)$ is decreasing if
$a_n\geq a_{n+1},$ for all $n\in \bbN . $ The defns for strictly increasing/strictly decreasing are similar but with strict inequalities.

\section*{Definition 15.14}
\begin{thm}
\label{15.14}
	\label{thm:increasing converges}
	Every bounded increasing sequence converges to the supremum of its image. Every bounded decreasing sequence converges to the infimum of its image.
\end{thm}
\vspace{4pt}     \hrule   \vspace{4pt}\begin{proof}:\\
Let $A = \{a_n | n \in \bbN\}$. Obviously, $A \neq \emptyset.$ Thus, because $A$ is bounded, then $u = \sup(A)$ exists. Let $\epsilon>0.$ Assume that for all $n\in \bbN$, $|a_n - u|\not < \epsilon.$ Therefore, for all $a_n,$ $a_n< u-\epsilon.$ Thus, $u-\epsilon$ is an upper bound of $a_n,$ which is a contradiction, since $u-\epsilon< u.$ Thus, for all $\epsilon>0,$ there exists some $N\in \bbN$ such that $|a_N - u|< \epsilon.$ Thus, if $n\geq N,$ then because $(a_n)$ is increasing, then $a_N \leq a_n< p,$ and thus $|a_n - u|< \epsilon.$
\end{proof}\vspace{4pt}     \hrule   \vspace{4pt}
To discuss the second partial converse, Theorem~\ref{15.18}, we need another defn.

\section*{Definition 15.15}
\begin{defn}
\label{15.15}
	Let $(a_n)$ be a sequence.  A \emph{subsequence} of $(a_n)$ is a sequence $b\colon \bbN \to \bbR$ defined by the composition $b = a \circ i$, where $i\colon \bbN \to \bbN$ is a strictly increasing function. If $(a_n)$ has a subsequence with limit $p$, we call $p$ a \emph{subsequential limit} of $(a_n)$.
\end{defn}

We can write $b_k = a(i(k)) = a_{i(k)} = a_{i_k}$, so that $(b_k)$ is the sequence $b_1, b_2, b_3, \dotsc$, which is equal to the sequence $a_{i_1}, a_{i_2}, a_{i_3}, \dotsc, $ where $i_1 < i_2 < i_3 < \dotsb$. 


\section*{Theorem 15.16}
\begin{thm}
	If $(a_n)$ converges to $p$, then so do all of its subsequences.
\end{thm}
\vspace{4pt}     \hrule   \vspace{4pt}\begin{proof}:\\
If $(a_n)$ converges to $p,$ then for all $\epsilon>0,$ there exists some $N \in \bbN$ such that if $n\geq N,$ then $|a_n -p|< \epsilon.$ Thus, let $(b_k)$ be a subsequence of $(a_n)$ such that $b_k = a(i(k)).$ Because $i$ is strictly increasing over $\bbN,$ then $N \leq i(N).$ \footnote{\begin{proof}:\\ Let $N \in \bbN$
    \begin{enumerate}
        \item If $N = 1,$ then $i(N)\geq N,$ as otherwise, $i(N)\notin \bbN.$
        \item If $N = m,$ where $m \in \bbN,$ then assume $m \leq i(m).$
        \item If $N = m+1,$ then because $i$ is strictly increasing, then $i(m+1)> i(m).$ Therefore, $i(m+1)\geq i(m) + 1,$ as otherwise, $i(m+1)\notin \bbN.$ Thus, by the inductive hypothesis, $i(m+1)\geq m+1.$
    \end{enumerate}
\end{proof}} Therefore, because $i(N)\geq N$ and $i(N)\in \bbN,$ then $|a(i(N)) - p|< \epsilon,$ and so if $n\geq i(N),$ then $i(n)\geq i(N),$ and thus $|a(i(n)) - p|< \epsilon.$ Therefore, $|b_n - p|< \epsilon.$
\end{proof}\vspace{4pt}     \hrule   \vspace{4pt}

\section*{Example 15.17}
\begin{exmp}
	Construct a sequence with two subsequential limits. Construct a sequence with infinitely many subsequential limits.
\end{exmp}
\vspace{4pt}     \hrule   \vspace{4pt}\begin{proof}:\\
\begin{enumerate}
    \item Let $k\in \bbN.$ Define $a: \bbN \to \bbR$ such that \[(a_n) :=
\begin{cases} 
9 \qquad n = 2k\\
6 \qquad n = 2k-1
\end{cases}\] (i.e, $(a_n) = 6,9,6,9,\dots$) Define $b: \bbN \to \bbR$ defined by the composition $b:= a \circ i_b,$ where $i_b:\bbN \to \bbN$ is defined by $i_b:= 2k-1.$ Thus, $(b_n) = 6,6,6,\dots.$ Similarly, define $c:\bbN \to \bbN$ by the composition $c:=a \circ i_c,$ where $i_c: \bbN \to \bbN$ is defined by $i_c:= 2k.$ Thus, $(c_n) = 9,9,9,\dots.$ You know it. I know it. Everybody knows it\footnote{Agustin gets political!}: $\lim_{n\to \infty}(b_n) = 6$ and $\lim_{n\to\infty}(c_n) = 9.$
\item Define $a: \bbN \to \bbN$ such that 
\[a_n:= 1, 1,2, 1,2,3, \dots\] Therefore, let $i_n: \bbN \to \bbN$ be strictly increased and be defined such that given some $n\in \bbN,$ $i_n$ is the position of such $n$ in $(a_n).$ Therefore, if $b: \bbN \to \bbR$ is defined such that $b := a \circ i_n,$ then:
\begin{align*}
    (b_1) = (a(i_1)) &=  a(1), a(2), a(4), \dots = 1,1,1,\dots\\
    (b_2) = (a(i_2)) &= a(3), a(5), \dots, a(i_{1_2} +1) = 2,2,2,\dots\\
    (b_3) = (a(i_3)) &= a(6), \dots, a(i_{n_2} +1) = 3,3,3,3\dots\\
    &\vdots\\
\end{align*}
Thus, there are infinite subsequences of $a_n$ that converge to a distinct limit. Note that $i_{n_k}$ refers to the position some $n\in \bbN$ that has been repeated $k$ times. 
\end{enumerate}

\end{proof}\vspace{4pt}     \hrule   \vspace{4pt}

\section*{Theorem 15.18}
\begin{thm}
	\label{15.18}
	Every bounded sequence has a convergent subsequence. 
\end{thm}
\vspace{4pt}     \hrule   \vspace{4pt} \begin{proof}:\\
Let $(a_n)$ be a sequence:
\begin{enumerate}
    \item If $(a_n)$ converges to some $p,$ then the statement is trivially true, as $(a_n)$ is a subsequence of itself.
    \item If $(a_n)$ diverges, then define some set $B := \{a_i \in a(\bbN) | a_i\geq a_n | i<n\}.$ 
    \begin{enumerate}
        \item If $B$ is finite, then let $a_N = \inf(B).$ Because $B$ is finite, then $B$ is closed, and so $\inf(B)\in B.$ Note that $a_{N+1}< a_N,$ as otherwise, $a_N\notin B.$ There must exist $r\in \bbN$ such that $N+1 <r$ and $a_{N+1}<a_r<a_N,$ as otherwise, $a_{N+1}\in b_i,$ which is a contradiction, since $b_N = \inf(B).$ Define $A_1 := \{a_n | a_{N+1}\leq a_n\}$. Let $b: \bbN \to \bbR$ be defined as $b: = a \circ i,$ where $i: \bbN \to \bbN$ strictly increasing is defined inductively. 
        \begin{enumerate}
            \item If $n = 1,$ then let \[\inf(A_1) = a_{N+1}= a(N+1) =: a(i(1))\] Therefore, $i(1) = N+1.$ Moreover, create $A_2:=A_1\setminus\{a_{N+1} = A_1\setminus\{b_1\}\}$
            \item If $n = k,$ where $k \in \bbN,$ then if $N+1 < p \in \bbN,$ then assume $\inf(A_k) := A_1\setminus\{b_1, b_2, \dots, b_k\} = a_p=: a(i(k)).$ Therefore, $i(k) = p$ and $b_1< b_k < a_N.$
            \item If $n = k+1,$ then assume that there does not exist an $a_n$ such that $a_p< a_n < a_N.$ Therefore, $a_j \in B,$ which is a contradiction. Thus, because $\inf(A_k)<\inf(A_{k+1}),$ then $i(k+1)> i(k)$ and by the inductive step, $b_1 < b_k <b_{k+1} = \inf(A_{k+1})< a_N.$
        \end{enumerate}
        Therefore, $(b_n)$ is a strictly increasing (sub)sequence that is bounded, and thus, by Theorem \ref{15.14}, it converges.
        \item If $B$ is infinite, then if $a_N \in B,$ there must exist some $n\in \bbN$ such that $N<n$ and $a_N \geq a_n$ as otherwise, $a_N$ would be the last element in $B$ which is a contradiction. Thus, define $b_n$ inductively:
        \begin{enumerate}
            \item Let $a_{r_1}\in B.$ If $n=1,$ then let $i(1) = r_1.$ Therefore, $b(1) = a(i(1)) = a_{r_1}.$ Define $B \subset A_1:= \{a_n| a_{r_1}\geq a_n\}.$ Note that because $A \subset B,$ then $n<r.$ Also note that $\sup(A_1) = a_{r_1} = b_1.$ 
            \item If $n = k,$ then assume $i(k) = j$ such that $b(j)= \sup(A\setminus\{b(1), b(2), \dots, b(j-1)\}).$ Therefore, $b(1)\geq b(j)$\\
            \item If $n = k+1,$ then  then if $i(k+1) = j+1,$ where $a(i(j+1)) = b(j+1)=\sup(A\setminus \{b(1), b(2), \dots, b(j-1), b(j)\}),$ then by the inductive step, $b_1\leq b_j \leq b_{j+1}.$
        \end{enumerate}
        Therefore, $(b_j)$ is a decreasing (sub)sequence that is bounded, and thus, by Theorem \ref{15.14}, it converges. 
    \end{enumerate}
\end{enumerate}
\end{proof}
\begin{proof}The above proof is cumbersome and hard to read; for a less rigorous approach:\\
Construct the same set $B$ as above. 
\begin{itemize}
    \item If $B$ is finite, then there must be some $a_m$ be the last point of $B.$ Because $a_m$ is the last point, then for any $N >m,$ there must exist some $N_1\in \bbN$ such that $N_1 >N$ and $a_{N}< a_{N_1}<a_m.$ Thus, let $N_1<N_2<\dots$ such that $a_{N_1}< a_{N_2}< \dots$ and define $b_k$ such that $(b_k):= a_{N_1}, a_{N_2}, \dots.$ Note that $(b_k)$ is constantly increasing and bounded (by $a_m$.) Therefore, by Theorem \ref{15.14}, it converges. 
    \item If $B$ is infinite: Then let $n_1< n_2< \dots$ be the $n\in \bbN$ such that $a_{n_1}\geq a_{n_2}\geq \dots $ are all in $B.$ Thus, if $b_k$ is the subsequence such that $(b_k) = a_{n_1}, a_{n_2}, \dots$ then $b_k$ is a decreasing subsequence. Note that because $(a_n)$ is bounded and $(b_k)$ is made of only terms in $(a_n),$ then $(b_k)$ is bounded as well. Thus, by Theorem \ref{15.14}, it converges.
\end{itemize}
    
\end{proof}


\vspace{4pt}     \hrule   \vspace{4pt}

We are now able to prove a useful characterization of convergent sequences.



\section*{Theorem 15.19: Augustin Cauchy Sequence}
\begin{thm} \label{15.19}  A sequence $(a_n)$ of real numbers converges if, and only if, for all $\epsilon>0,$ there is some $N\in\bbN$ such that $|a_n-a_m|<\epsilon,\forall n\geq N, m\geq N.$
\end{thm}
\vspace{4pt}     \hrule   \vspace{4pt}\begin{proof}:\\
\begin{itemize}
    \item ($\implies$:) Let $\epsilon>0,$ If $(a_n)$ is a sequence that converges to $p\in \bbR$ then there exists some $N \in \bbN$ such that if $n\geq N$, then $|a_n - p|< \frac{\epsilon}{2}$ Similarly, if $m\geq N,$ then $|a_m - p|<\frac{\epsilon}{2}.$ 
    \begin{align*}
        \epsilon &= \frac{\epsilon}{2}+ \frac{\epsilon}{2}\\
        &> |a_m - p| + |p-a_n|\\
        &\geq |a_m - a_n|
    \end{align*}
    \item ($\impliedby$:) Let $\epsilon = 1,$ then there exists some $N \in \bbN$ such that if $n,m\geq N,$ then 
    \[|a_n - a_m|< 1\] 
    Thus, for all $n\geq N,$ \[|a_n - a_N|<1\] and so $\{a_n|n\geq N\}$ is bounded by $a_N - 1$ and $a_N + 1.$ Thus, because there exists a finite 
    amount of $a_n$ outside of $(a_N-1, a_N+1),$ then there exists a lower and upper bound for $(a_n)$ and so $a_n$ is bounded. Therefore, by Theorem \ref{15.18}, $(a_n)$ has a convergent subsequence, $(b_k).$ Let $\epsilon>0.$
    \begin{enumerate}
        \item There exists an $N_1$ such that if $m,n \geq N,$ then $|a_n - a_m|< \frac{\epsilon}{2}$
        \item There exists some $N_2$ such that if $k\geq N_2,$ then $|b_k - p|< \frac{\epsilon}{2}$
    \end{enumerate}
    Let $M = \max(N_1, N_2).$ Thus, if $n\geq M,$ then:
    \begin{align*}
        |a_n - p| &\leq |a_n- b_n| + |b_n - p|\\
        &< \frac{\epsilon}{2} + \frac{\epsilon}{2}
    \end{align*}
\end{itemize}
\end{proof}\vspace{4pt}     \hrule   \vspace{4pt}

\section*{Remark 15.20: Building the Reals, The Real Way}
\vspace{4pt}     \hrule   \vspace{4pt}
The following is an alternate construction for $\bbR:$\footnote{Aknowledgment: Terry Tao}\newline\newline
\begin{defn}
    A \textit{real number} is defined to be an
 object of the form $\displaystyle\lim_{n\to \infty}(a_n),$ where $(a_n)$ is a Cauchy sequence of rational numbers. Two real numbers $\displaystyle\lim_{n\to \infty}(a_n)$ and $\displaystyle\lim_{n\to \infty}(b_n)$ are
 said to be equal iff $(a_n)$ and $(b_n)$
  are equivalent Cauchy sequences.
 The set of all real numbers is denoted $\bbR$
\end{defn}
\vspace{4pt}     \hrule   \vspace{4pt}


\section*{Example 15.21}
\begin{exmp} 
Use Theorem~\ref{15.19}  to show that the sequence in exmp~\ref{15.10}a does not converge.
\end{exmp}
\vspace{4pt}     \hrule   \vspace{4pt}\begin{proof}:\\
    Let $\epsilon = 1.$ For all $N\in \bbN,$ there exists an $m=2N>N$ and $n = 2N+1 >N$ such that:
    \begin{align*}
        |(-1)^{n}n - (-1)^mm|&= |(-1)^{2N+1}n - (-1)^{2N}m|\\
        &= |-2N - 2N+1|\\
        &= |-4N+1|\\
        &>1
    \end{align*}
And thus, because for some $\epsilon>0,$ if $N \in \bbN,$ then there exists some $n,m\geq N$ such that $|a_n - a_m|\not < \epsilon,$ then by Theorem \ref{15.19}, $(a_n)$ does not converge.
\end{proof}\vspace{4pt}     \hrule   \vspace{4pt}



\begin{center}
{\em Additional Exercises}
\end{center}

\begin{enumerate}

\item
\begin{enumerate}
\item[i)] Suppose that for two convergent sequences of real numbers, $(a_n)$ and $(b_n),$ 
$  a_n \leq b_n, \text{ for all } n\in \bbN. $
 Prove that $ \lim_{n\to\infty} a_n \leq \lim_{n\to\infty}b_n.$
\item[ii)] Suppose that $0\leq a_n\leq b_n,$ for all $n\in\bbN.$ Suppose $\lim_{n\to\infty} b_n=0.$ Show that $\lim_{n\to\infty} a_n=0$ also. 
\item[iii)] Suppose that $(a_n), (b_n), (c_n)$ are sequences of real numbers such that $a_n\leq b_n\leq c_n,$ for all $n\in \bbN.$ Suppose also that $\lim_{n\to\infty} a_n = \lim_{n\to\infty} c_n = L.$ Prove that $\lim_{n\to\infty} b_n = L$ also.
\item[iv)] Let $a\in\bbR$ with $a>1.$ Prove that $\displaystyle \lim_{n\to\infty} a^{\frac{1}{n}}=1.$ Does this result hold for $0<a\leq 1$ also? 
{\em Hint: Use iii) and Bernoulli's inequality (Script 0, Exercise 3) with an appropriate choice of $x.$}
\item[v)] Justify the following inequalities:
$$\left(1+\frac{1}{\sqrt{n}}\right)^n\geq 1+\sqrt{n}\geq \sqrt{n}.$$
Hence prove that $\lim_{n\to \infty} n^{\frac{1}{n}}=1.$

\end{enumerate}

\item
Suppose that the sequence $(a_n)$ converges to $0$  and the sequence $(b_n)$ is bounded. Prove that 
$$ 
\lim_{n\to\infty} (a_n b_n)=0.
$$ 



\item Suppose that $(a_n)$ is a sequence of real numbers. Prove that $(a_n)$ has either an increasing subsequence or a decreasing subsequence (or both).


\item For $n\in\bbN,$ define $a_{n+1}=2+a_n/3$ and let $a_0=100.$ Show that $(a_n)$ is convergent.

\item Find the limit of each of the following sequences:
\begin{enumerate}
\item $a_n=\sqrt{n+1}-\sqrt{n-1};$
\item $a_n=n-\sqrt{n+1}\sqrt{n+2}.$
\end{enumerate}

\item
 Let $a>0$ and consider a sequence 
$$
x_n=\sqrt{a+\sqrt{a+...\sqrt{a}}},
$$
where $n$ is the number of roots. Prove that $x_n$ converges and find its limit. You may assume that square roots are well-defined, i.e. there is a bijective, continuous, strictly increasing 
function $g:\bbR^{+}\to \bbR^{+}$ such that $g(x)^2=x$ for all $x\in \bbR^{+}.$

{\em Here $\bbR^{+}$ denotes the set of non-negative real numbers.}





\item Let $f:[a,b]\to\bbR$ be a bounded function and $\Omega\in\bbR$. Prove that $\int_a^b f$ exists and equals $\Omega$ if and only if there  exists a sequence of partitions $(P_n)_{n\in\bbN}$ such that
$$\lim_{n\to\infty} U(P_n,f)=\lim_{n\to\infty} L(P_n,f)=\Omega.$$


\item
\begin{enumerate}
\item[a)] 

Let $f:[a,b]\longrightarrow \bbR$ be a bounded function. Let $P_n$ be a partition of $[a,b]$ into $n$ equal subintervals. Prove that
if $\displaystyle \lim_{n\longrightarrow\infty} U(f,P_n)=\displaystyle\lim_{n\longrightarrow\infty} L(f,P_n)=L,$ then
$f$ is integrable on $[a,b]$ and $\int _a^b f=L.$


\item[b)] Use a) to prove that $f(x)=x^2$ is integrable on $[a,b]$ and find $\int_a^b f.$ 
{\em Hint: Sheet 0 Exercise 2 should be useful.}
 \end{enumerate}

 
 \section*{Additional Exercise 9: Sequences and Continuity}
 \item Let $A$ be an open set and $f\colon A\to \bbR.$ 
\begin{enumerate}
\item[a)] Write down the negation of the statement:
``Given $\epsilon>0,$ there is some $\delta>0$ such that $|x-a|<\delta\Longrightarrow |f(x)-f(a)|<\epsilon.$''
\vspace{4pt}     \hrule   \vspace{4pt}\begin{proof}:\\
There exists an  $\epsilon > 0$ such that for every  $\delta > 0$, there exists an $ x\in A$ such that $|x - a| < \delta$ and $|f(x) - f(a)| \geq \epsilon$.
\end{proof}\vspace{4pt}     \hrule   \vspace{4pt}


\item[b)] Prove that if $f\colon A\longrightarrow \bbR$ is not continuous at $a\in A,$ then there is some $\epsilon>0$ and a sequence $(x_n)$ in $A$ such that 
$|x_n-a|<\frac{1}{n}$ and $|f(x_n)-f(a)|\geq \epsilon.$
\vspace{4pt}     \hrule   \vspace{4pt}\begin{proof}:\\
    If $f$ is not continuous at $a,$ then there exists an $\epsilon>0$ such that for all $\delta>0,$ if $x\in A$ and $|x-a|< \delta,$ then $f(x) - f(a)\geq \epsilon.$ Because $A$ is open and $a \in A,$ then $a\in LP(A).$ Thus, for any $n\in \bbN,$ if $(a-\frac{1}{n}, a + \frac{1}{n})$ is a region containing $a,$ there must exist some $x_n \in A \cap (a-\frac{1}{n}, a+ \frac{1}{n}).$ By the axiom of choice, define the sequence $(x_n)$ by choosing such $x_n \in (a-\frac{1}{n}, a+ \frac{1}{n})$ for any $n \in \bbN.$ Therefore, if $\frac{1}{N}< \delta$ and $n\geq N,$ then $|x_n - a|< \frac{1}{n}< \delta,$ and thus $|f(x_n) - f(a)|\geq \epsilon.$
\end{proof}\vspace{4pt}     \hrule   \vspace{4pt}

\item[c)]
Prove that a function $f\colon A\longrightarrow \bbR$ is continuous  at $a\in A$ if, and only if, $\displaystyle \lim_{n\longrightarrow \infty}f(x_n)=f(a)$ for every sequence $(x_n)$ in $A$ such that
$\displaystyle \lim_{n\longrightarrow \infty} x_n=a.$
\vspace{4pt}     \hrule   \vspace{4pt} \begin{proof}:\\
    \begin{itemize}
        \item ($\implies$:) If $f$ is continuous at $a,$ then for all $\epsilon>0,$ there exists a $\delta>0$ such that if $x\in A$ and $|x-a|<\delta,$ then $|f(x) - f(a)|< \delta.$ Thus, let $(a_n)$ be a sequence converging to $a$ (we know that at least one exists by Theorem \ref{15.11}). Therefore, there exists an $N \in \bbN$ such that if $n\geq N,$ then $|x_n - a|< \delta,$ and thus $|f(x_n) - f(a)|< \epsilon.$ 
        \item ($\impliedby$:) This follows from the contrapositive of part b.
    \end{itemize}
\end{proof}\vspace{4pt}     \hrule   \vspace{4pt}

\item[d)] Prove that a function $f\colon A\longrightarrow \bbR$ is uniformly continuous if, and only if, for every pair of sequences $(x_n),(y_n)$ in $A$ such that $(x_n-y_n)$ converges to 0,  $(f(x_n)-f(y_n))$ converges to 0.
\vspace{4pt}     \hrule   \vspace{4pt} \begin{proof}:\\
    \begin{itemize}
        \item ($\implies$:) If $f$ is uniformly continuous, then for all $\epsilon>0,$ there exists a $\delta>0$ such that if $x,y \in A,$ and $|x-y|< \delta,$ then $|f(x) - f(y)|< \epsilon.$ Thus let $(x_n), (y_n) \in A$ be pairs of sequences such that $(x_n) - (y_n)$ converge to $0.$ Therefore, there exists an $N\in \bbN$ such that if $n\geq N,$ then $|x_n - y_n|< \delta.$ Because $|x_n - y_n|< \delta,$ then $|f(x_n) - f(y_n)|< \epsilon,$ and thus $(f(x_n) - f(y_n))$ converge to $0.$
        \item ($\impliedby$:) Assume, for the sake of contradiction, that $f$ is not uniformly continuous. Therefore, there exists some $\epsilon>0$ such that for all $\delta>0,$ if $x,y \in A$ and $|x-y|< \delta,$ then $|f(x) - f(y)|\geq \epsilon.$ Therefore, define $(x_n), (y_n)$ similarly as to how $(x_n)$ was defined in part $(b).$\footnote{In reality, if they were both defined as $(x_n),$ then they would always be the same sequence. In order to sidestep this issue, simply define $y_n$ such that $y_n \in (a-\frac{1}{2n}, a+ \frac{1}{2n})$ for all $n\in \bbN.$ This doesn't impact the proof however.} Therefore, if $N\in \bbN$ such that $\frac{1}{N}< \delta,$ then for all $n\geq N:$
        \begin{align*}
            |x_n - y_n|&\leq |x_n - a| + |a - y_n|\\
            &< \frac{1}{n} + \frac{1}{n}\\
            &= \frac{1}{2n}\\
            &< \frac{1}{n}\\
            &< \delta
        \end{align*}
        and therefore $|f(x_n) - f(y_n)|\geq \epsilon,$ and so $(f(x_n)) - (f(y_n))$ does not converge to $0,$ which is a contradiction.
    \end{itemize}
\end{proof}\vspace{4pt}     \hrule   \vspace{4pt} 
\end{enumerate}

\item We say that a sequence is {\em a close sequence} if it satisfies the condition in thm~\ref{CS}, i.e. if, for all $\epsilon >0,$ there is some $N\in \bbN$ such that $|a_n-a_m|<\epsilon,$ for all $n\geq N, m\geq N.$ Suppose that $f\colon A\longrightarrow \bbR$ is uniformly continuous and $(x_n)$ is a close sequence in $A.$ Prove that $(f(x_n))$ is a close sequence in $\bbR.$

\item Suppose that $f: \bbR\setminus\{0\}\to\bbR$ is a continuous function.
\begin{enumerate}
\item Show that if $f$ is uniformly continuous, then $\displaystyle\lim_{x\to 0}f(x)$ exists.
\item Give an example to show that $\displaystyle\lim_{x\to 0}f(x)$ need not exist if $f$ is only continuous (but not uniformly).
\end{enumerate}

\item
Suppose that $A \subset B$ and that $A$ is dense in $B$.  Suppose $f: A \to \bbR$ is uniformly continuous. Prove that there exists a unique continuous function $\overline f : B \to \bbR$ such that $f(x) = \overline f(x)$ for all $x \in A$.  Moreover, $\overline f$ is uniformly continuous.

\item
\begin{enumerate}
\item
A {\em close sequence} (of rational numbers) is a sequence $(a_n)$ with $a_n\in\bbQ$ such that, given an $\epsilon\in\bbQ$ with $\epsilon>0,$ there is some $N\in\bbN$ such that for all $n,m\geq N$ we have $|a_n-a_m|<\epsilon.$ (This is the condition we see in Theorem~\ref{CS}.) Prove that 
\begin{enumerate}
\item[i)] Any sequence of rationals that converges in $\bbQ$ is a close sequence. 
\item[ii)] Close sequences are bounded. 
\item[iii)] If $(x_n)$ and $(y_n)$ are close sequences then so are $(x_n+y_n)$ and $(x_n y_n).$ 
\item[iv)] The set of close sequences of rationals has additive and multiplicative identities, additive inverses, but many close sequences fail to have multiplicative inverses.
\end{enumerate}





\item  Define $X$ to be the set of all close sequences of rational numbers. Then we define an equivalence relation on $X$ by 
$(a_n)\sim (b_n)$ if $\displaystyle \lim_{n\longrightarrow \infty}|a_n-b_n|=0$.  Prove that this is indeed an equivalence relation.





We let ${\mathcal R}$ denote the set of equivalences classes of close sequences of rationals and define addition and multiplication in ${\mathcal R}$ by

\begin{eqnarray*}
[(a_n)]+[(b_n)] & = & [(a_n+b_n)] \\
{ } [(a_n)] [(b_n)] & = & [(a_n b_n)].
\end{eqnarray*}



Note that we can view $\bbQ$ as a subset of ${\mathcal R}$ by identifying a rational number $q$ with the equivalence class of the constant sequence, $[(q,q,q,.....)].$

\item Prove that addition and multiplication are well-defined.

\item
Prove that ${\mathcal R}$ is a field.

{\it Note: The field axioms should all be very routine, except for FA8.}
\item
We say that an equivalence class $[(a_n)]$ is {\em positive} and write $[(a_n)]>0,$ or $0<[(a_n)],$  if there is some $N\in\bbN$ and $\epsilon\in\bbQ,\epsilon>0,$ such that $a_n>\epsilon$ for all $n\geq N.$ We say that $[(a_n)]>[(b_n)]$ (or $[(b_n)]<[(a_n)]$),  if $[(a_n-b_n)]>0.$  Prove that $<$ is well-defined and that ${\mathcal R}$ is an ordered field with this ordering.

\item 
\begin{enumerate}
\item[i)] Prove that ${\mathcal R}$ satisfies Axioms 1-3. 
\item[ii)] Prove that if $x,y\in {\mathcal R}, x<y,$ there is some $z\in {\mathcal R}$ such that $x<z<y.$ 
\item[iii)] Prove every nonempty bounded subset of ${\mathcal R}$ has a supremum.
\item[iv)] Does ${\mathcal R}$ satisfy Axiom 4?
\item[v)] Does ${\mathcal R}$ satisfy Axiom 5?
\end{enumerate}

{\em Hint for iii): Let $X$ be a bounded subset of ${\mathcal R}.$ Consider 
$$Y=\{q\in\bbQ\mid [(q,q,q,\cdots)]\text{ is not an upper bound of }X\}.$$ 
Construct a suitable sequence $(q_n)$ of points in $Y.$  }


\item 
What can you say about ${\mathcal R}$?
\end{enumerate}
\end{enumerate}




\end{document}

\section*{Acknowledgments} 
Thanks, as always, to Professor Oron Propp for being a great mentor in both Office Hours and during class. Thank you to Richard Gale for showing me a smart way of doing 13.4 (I included both his (first one) and my proof (second)). Thanks also to Lina Piao for working with me to figure out a couple of proofs, such as 13.19, 13.20, and 13.29.
\begin{thebibliography}{9}




\end{thebibliography}

\end{document}

