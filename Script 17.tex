
\documentclass[openany, amssymb, psamsfonts]{amsart}
\usepackage{mathrsfs,comment}
\usepackage[usenames,dvipsnames]{color}
\usepackage[normalem]{ulem}
\usepackage{url}
\usepackage{tikz}
\usepackage{tkz-euclide}
\usepackage{lipsum}
\usepackage{marvosym}
\usepackage[all,arc,2cell]{xy}
\UseAllTwocells
\usepackage{enumerate}
\newcommand{\bA}{\mathbf{A}}
\newcommand{\bB}{\mathbf{B}}
\newcommand{\bC}{\mathbf{C}}
\newcommand{\bD}{\mathbf{D}}
\newcommand{\bE}{\mathbf{E}}
\newcommand{\bF}{\mathbf{F}}
\newcommand{\bG}{\mathbf{G}}
\newcommand{\bH}{\mathbf{H}}
\newcommand{\bI}{\mathbf{I}}
\newcommand{\bJ}{\mathbf{J}}
\newcommand{\bK}{\mathbf{K}}
\newcommand{\bL}{\mathbf{L}}
\newcommand{\bM}{\mathbf{M}}
\newcommand{\bN}{\mathbf{N}}
\newcommand{\bO}{\mathbf{O}}
\newcommand{\bP}{\mathbf{P}}
\newcommand{\bQ}{\mathbf{Q}}
\newcommand{\bR}{\mathbf{R}}
\newcommand{\bS}{\mathbf{S}}
\newcommand{\bT}{\mathbf{T}}
\newcommand{\bU}{\mathbf{U}}
\newcommand{\bV}{\mathbf{V}}
\newcommand{\bW}{\mathbf{W}}
\newcommand{\bX}{\mathbf{X}}
\newcommand{\bY}{\mathbf{Y}}
\newcommand{\bZ}{\mathbf{Z}}

%% blackboard bold math capitals
\newcommand{\bbA}{\mathbb{A}}
\newcommand{\bbB}{\mathbb{B}}
\newcommand{\bbC}{\mathbb{C}}
\newcommand{\bbD}{\mathbb{D}}
\newcommand{\bbE}{\mathbb{E}}
\newcommand{\bbF}{\mathbb{F}}
\newcommand{\bbG}{\mathbb{G}}
\newcommand{\bbH}{\mathbb{H}}
\newcommand{\bbI}{\mathbb{I}}
\newcommand{\bbJ}{\mathbb{J}}
\newcommand{\bbK}{\mathbb{K}}
\newcommand{\bbL}{\mathbb{L}}
\newcommand{\bbM}{\mathbb{M}}
\newcommand{\bbN}{\mathbb{N}}
\newcommand{\bbO}{\mathbb{O}}
\newcommand{\bbP}{\mathbb{P}}
\newcommand{\bbQ}{\mathbb{Q}}
\newcommand{\bbR}{\mathbb{R}}
\newcommand{\bbS}{\mathbb{S}}
\newcommand{\bbT}{\mathbb{T}}
\newcommand{\bbU}{\mathbb{U}}
\newcommand{\bbV}{\mathbb{V}}
\newcommand{\bbW}{\mathbb{W}}
\newcommand{\bbX}{\mathbb{X}}
\newcommand{\bbY}{\mathbb{Y}}
\newcommand{\bbZ}{\mathbb{Z}}

%% script math capitals
\newcommand{\sA}{\mathscr{A}}
\newcommand{\sB}{\mathscr{B}}
\newcommand{\sC}{\mathscr{C}}
\newcommand{\sD}{\mathscr{D}}
\newcommand{\sE}{\mathscr{E}}
\newcommand{\sF}{\mathscr{F}}
\newcommand{\sG}{\mathscr{G}}
\newcommand{\sH}{\mathscr{H}}
\newcommand{\sI}{\mathscr{I}}
\newcommand{\sJ}{\mathscr{J}}
\newcommand{\sK}{\mathscr{K}}
\newcommand{\sL}{\mathscr{L}}
\newcommand{\sM}{\mathscr{M}}
\newcommand{\sN}{\mathscr{N}}
\newcommand{\sO}{\mathscr{O}}
\newcommand{\sP}{\mathscr{P}}
\newcommand{\sQ}{\mathscr{Q}}
\newcommand{\sR}{\mathscr{R}}
\newcommand{\sS}{\mathscr{S}}
\newcommand{\sT}{\mathscr{T}}
\newcommand{\sU}{\mathscr{U}}
\newcommand{\sV}{\mathscr{V}}
\newcommand{\sW}{\mathscr{W}}
\newcommand{\sX}{\mathscr{X}}
\newcommand{\sY}{\mathscr{Y}}
\newcommand{\sZ}{\mathscr{Z}}


\renewcommand{\phi}{\varphi}
\renewcommand{\emptyset}{\O}

\newcommand{\abs}[1]{\lvert #1 \rvert}
\newcommand{\norm}[1]{\lVert #1 \rVert}
\newcommand{\sm}{\setminus}


\newcommand{\sarr}{\rightarrow}
\newcommand{\arr}{\longrightarrow}

\newcommand{\hide}[1]{{\color{red} #1}} % for instructor version
%\newcommand{\hide}[1]{} % for student version
\newcommand{\com}[1]{{\color{blue} #1}} % for instructor version
%\newcommand{\com}[1]{} % for student version
\newcommand{\meta}[1]{{\color{green} #1}} % for making notes about the script that are not intended to end up in the script
%\newcommand{\meta}[1]{} % for removing meta comments in the script

\DeclareMathOperator{\ext}{ext}
\DeclareMathOperator{\ho}{hole}
%%% hyperref stuff is taken from AGT style file
\usepackage{hyperref}  
\hypersetup{%
  bookmarksnumbered=true,%
  bookmarks=true,%
  colorlinks=true,%
  linkcolor=blue,%
  citecolor=blue,%
  filecolor=blue,%
  menucolor=blue,%
  pagecolor=blue,%
  urlcolor=blue,%
  pdfnewwindow=true,%
  pdfstartview=FitBH}   
  
\let\fullref\autoref
%
%  \autoref is very crude.  It uses counters to distinguish environments
%  so that if say {lemma} uses the {theorem} counter, then autrorefs
%  which should come out Lemma X.Y in fact come out Theorem X.Y.  To
%  correct this give each its own counter eg:
%                 \newtheorem{theorem}{Theorem}[section]
%                 \newtheorem{lemma}{Lemma}[section]
%  and then equate the counters by commands like:
%                 \makeatletter
%                   \let\c@lemma\c@theorem
%                  \makeatother
%
%  To work correctly the environment name must have a corrresponding 
%  \XXXautorefname defined.  The following command does the job:
%
\def\makeautorefname#1#2{\expandafter\def\csname#1autorefname\endcsname{#2}}
%
%  Some standard autorefnames.  If the environment name for an autoref 
%  you need is not listed below, add a similar line to your TeX file:
%  
%\makeautorefname{equation}{Equation}%
\def\equationautorefname~#1\null{(#1)\null}
\makeautorefname{footnote}{footnote}%
\makeautorefname{item}{item}%
\makeautorefname{figure}{Figure}%
\makeautorefname{table}{Table}%
\makeautorefname{part}{Part}%
\makeautorefname{appendix}{Appendix}%
\makeautorefname{chapter}{Chapter}%
\makeautorefname{section}{Section}%
\makeautorefname{subsection}{Section}%
\makeautorefname{subsubsection}{Section}%
\makeautorefname{theorem}{Theorem}%
\makeautorefname{thm}{Theorem}%
\makeautorefname{excercise}{Exercise}%
\makeautorefname{cor}{Corollary}%
\makeautorefname{lem}{Lemma}%
\makeautorefname{prop}{Proposition}%
\makeautorefname{pro}{Property}
\makeautorefname{conj}{Conjecture}%
\makeautorefname{defn}{Definition}%
\makeautorefname{notn}{Notation}
\makeautorefname{notns}{Notations}
\makeautorefname{rem}{Remark}%
\makeautorefname{quest}{Question}%
\makeautorefname{exmp}{Example}%
\makeautorefname{ax}{Axiom}%
\makeautorefname{claim}{Claim}%
\makeautorefname{ass}{Assumption}%
\makeautorefname{asss}{Assumptions}%
\makeautorefname{con}{Construction}%
\makeautorefname{prob}{Problem}%
\makeautorefname{warn}{Warning}%
\makeautorefname{obs}{Observation}%
\makeautorefname{conv}{Convention}%


%
%                  *** End of hyperref stuff ***

%theoremstyle{plain} --- default
\newtheorem{thm}{Theorem}[section]
\newtheorem{cor}{Corollary}[section]
\newtheorem{exercise}{Exercise}
\newtheorem{prop}{Proposition}[section]
\newtheorem{lem}{Lemma}[section]
\newtheorem{prob}{Problem}[section]
\newtheorem{conj}{Conjecture}[section]
%\newtheorem{ass}{Assumption}[section]
%\newtheorem{asses}{Assumptions}[section]

\theoremstyle{definition}
\newtheorem{defn}{Definition}[section]
\newtheorem{ass}{Assumption}[section]
\newtheorem{asss}{Assumptions}[section]
\newtheorem{ax}{Axiom}[section]
\newtheorem{con}{Construction}[section]
\newtheorem{exmp}{Example}[section]
\newtheorem{notn}{Notation}[section]
\newtheorem{notns}{Notations}[section]
\newtheorem{pro}{Property}[section]
\newtheorem{quest}{Question}[section]
\newtheorem{rem}{Remark}[section]
\newtheorem{warn}{Warning}[section]
\newtheorem{sch}{Scholium}[section]
\newtheorem{obs}{Observation}[section]
\newtheorem{conv}{Convention}[section]

%%%% hack to get fullref working correctly
\makeatletter
\let\c@obs=\c@thm
\let\c@cor=\c@thm
\let\c@prop=\c@thm
\let\c@lem=\c@thm
\let\c@prob=\c@thm
\let\c@con=\c@thm
\let\c@conj=\c@thm
\let\c@defn=\c@thm
\let\c@notn=\c@thm
\let\c@notns=\c@thm
\let\c@exmp=\c@thm
\let\c@ax=\c@thm
\let\c@pro=\c@thm
\let\c@ass=\c@thm
\let\c@warn=\c@thm
\let\c@rem=\c@thm
\let\c@sch=\c@thm
\let\c@equation\c@thm
\numberwithin{equation}{section}
\makeatother

\bibliographystyle{plain}

%--------Meta Data: Fill in your info------
\title{University of Chicago Calculus IBL Course}

\author{Agustin Esteva}

\date{Apr 12. 2024}

\begin{document}

\begin{abstract}

16310's Script 16.\\ Let me know if you see any errors! Contact me at aesteva@uchicago.edu.


\end{abstract}

\maketitle

\tableofcontents

\setcounter{section}{17}

\section*{Definition 17.1: Sequence of Functions}
\begin{defn}
\label{17.1}
	Let $A\subset\bbR$, and consider $X = \{f\colon A \to \bbR \}$, the collection of real-valued functions on $A$. A \emph{sequence of functions (on $A$)} is an ordered list $(f_1, f_2, f_3, \ldots)$ which we will denote $(f_n)$, 
	where each $f_n \in X$. (More formally, we can think of the sequence as a function $F\colon \bbN \to X$, where $f_n = F(n)$, for each $n \in \bbN$, but this degree of formality is not particularly helpful.) 
	
	We can take the sequence to start at any $n_0 \in \bbZ$ and not just at $1$, just like we did for sequences of real numbers.
\end{defn}

\section*{Definition 17.2: Pointwise Convergence of Functions}
\begin{defn}
\label{17.2}
	The sequence $(f_n)$ \emph{converges pointwise} to a function $f\colon A \to \bbR$ if for all $x \in A$ and $\epsilon > 0$, there exists $N \in \bbN$ such that:
	\begin{center}
		if $ n \geq N$, \quad then $\abs{f_n(x) - f(x)} < \epsilon$.
	\end{center}
	In other words, we have that for all $x \in A$, $\lim\limits_{n \to \infty} f_n(x) = f(x)$.
\end{defn}

\section*{Definition 17.3: Uniform Convergence of Functions}
\begin{defn}
\label{17.3}
	The sequence $(f_n)$ \emph{converges uniformly} to a function $f\colon A \to \bbR$ if for all $\epsilon > 0$, there exists $N \in \bbN$ such that:
	\begin{center}
		if $ n \geq N$, \quad then $\abs{f_n(x) - f(x)} < \epsilon$ \quad for every $x \in A$.
	\end{center}
Equivalently, the sequence $(f_n)$ \emph{converges uniformly} to a function $f\colon A \to \bbR$ if for all $\epsilon > 0$, there exists $N \in \bbN$ such that:
	\begin{center}
		if $ n \geq N$, \quad then $\displaystyle \sup_{x\in A} \abs{f_n(x) - f(x)} < \epsilon$.
	\end{center}
\end{defn}

\section*{Example 17.4}
\begin{exmp} 
\label{17.4}
Suppose that a sequence $(f_n)$ converges pointwise to a function $f.$ Prove that if $(f_n)$ converges uniformly to a function $g$, then  $f=g.$
\end{exmp}
\vspace{4pt}     \hrule   \vspace{4pt}\begin{proof}:\\
If $(f_n)$ converges uniformly to a function $g,$ then for all $\epsilon>0,$ there exists some $N \in \bbN$ such that if $n\geq N,$ then $|f_n(x) - g(x)|< \epsilon$ for all $x \in A.$ Thus, if $x\in A,$ there exists such $N,$ such that if $n\geq N,$ then $|f_n(x) - g(x)|< \epsilon.$\footnote{This proves that if $(f_n)$ converges uniformly, then it converges pointwise} Therefore, because $\displaystyle\lim_{n\to \infty}(f_n(x)) = g(x)$ and $\displaystyle\lim_{n\to \infty }(f_n(x)) = f(x),$\footnote{This is from the fact that $(f_n)$ converges pointwise to $f$} then by Theorem 15.4, because convergent points are unique, then $f(x) = g(x)$ for all $x\in A$ and so $f = g.$
\end{proof}\vspace{4pt}     \hrule   \vspace{4pt}

\section*{Example 17.5}
\begin{exmp} 
\label{17.5}
	For each of the following sequences of functions, determine what function the sequence $(f_n)$ converges to pointwise. Does the sequence converge uniformly to this function?
\end{exmp}
\begin{enumerate}
\item[i)]  For $n \in \bbN$, let $f_n\colon [0, 1] \to \bbR$ be given by $f_n(x) = x^n$.	
\vspace{4pt}     \hrule   \vspace{4pt}\begin{proof}:\\
\begin{itemize}
\item 
\begin{enumerate}
    \item If $x \in [0,1),$ then by Example 15.8, $\displaystyle\lim_{n\to \infty}x^n = 0.$
    \item If $x = 1,$ then $\displaystyle\lim_{n\to \infty} =1.$
\end{enumerate}
Thus, for all $x\in [0,1],$  $\displaystyle\lim_{n \to \infty}x^n = 
\begin{cases}
0,\qquad x\in [0,1)\\
1,\qquad x = 1
\end{cases},$ and so $x^n$ converges pointwise.
\item Assume, for the sake of contradiction, that $(f_n(x))$ converges uniformly to $f(x).$ Therefore, let $\epsilon = \frac{1}{2}.$ For all $N \in \bbN,$ if $n\geq N,$ then if $x = (\frac{4}{5})^\frac{1}{N},$ then $x\in [0,1).$ However, since $|f_n(x) - f(x)| = |x^n - 0| = |\frac{4}{5}|> \frac{1}{2},$ then $|f_n(x) - f(x)|>\epsilon$ for some $x\in A,$ which is a contradiction.
\end{itemize}
\end{proof}\vspace{4pt}     \hrule   \vspace{4pt}
\item[ii)] For $n\in \bbN,$ let $f_n\colon \bbR\to \bbR$ be given by $\displaystyle f_n(x)=\frac{\sin (nx)}{n}.$ (For the purposes of this example you may assume basic knowledge of $\sin$.)
\vspace{4pt}     \hrule   \vspace{4pt}\begin{proof}:\\
\begin{itemize}
\item For all $\epsilon>0,$ there exists an $N \in \bbN$ such that $\frac{1}{N}< \epsilon.$ Therefore, if $n\geq N,$ then since $|\sin(nx)|<1:$
\begin{align*}
|f_n(x) - f(x)| &= \left|\frac{\sin(nx)}{n} - 0\right|\\
&= \left| \frac{1}{n}\right| |\sin(nx)|\\
&\leq \epsilon(1)\\
&= \epsilon
\end{align*}
Therefore, $f_n(x)$ converges uniformly to $f(x) = 0.$.
\item Because $f_n(x)$ converges uniformly, then it converges pointwise to $f(x) = 0.$
\end{itemize}
\end{proof}
\item[iii)] For $n\in\bbN,$ let $f_n\colon [0,1]\to \bbR$ be given by $f_n(x)=\begin{cases} n^2 x & 0\leq x\leq \frac{1}{n}\\ 
n(2-nx) & \frac{1}{n}\leq x\leq \frac{2}{n}\\ 0 & \frac{2}{n}\leq x\leq 1. 
\end{cases} $ 
\vspace{4pt}     \hrule   \vspace{4pt}\begin{proof}:\\
\begin{itemize}
\item For all $\epsilon>0,$ there exists an $N \in \bbN$ such that if $N$ and $n\geq N,$ then for all $x\in (0,1],$ there exists some $N_x$ such that $\frac{2}{N_x}< x,$ and thus $f_n(x) = 0.$ Therefore, for all $x\in (0,1],$ if $n\geq N_x,$ then $|0-0| = 0 < \epsilon.$ Note that if $x=0,$ then the case is reduced to $f_n(x) = 0,$ and so evidently, for any $N\in \bbN,$ $\displaystyle\lim_{n\to \infty}f_n(0) = 0.$ Thus, $(f_n)$ converges pointwise to $f(x) = 0.$
\item Let $\epsilon = 1.$ Let $N \in \bbN$ and $n\geq N.$ If $x = \frac{1}{N},$ then:
\begin{align*}
|f_n(\frac{1}{N}) - 0| &= |n^2\frac{1}{n}|\\
&= |n|\\
&\geq 1
\end{align*}
Thus, $(f_n)$ is not uniformly convergent. 
\end{itemize}
\end{proof}\vspace{4pt}     \hrule   \vspace{4pt}
\end{enumerate}



\section*{Theorem 17.6}
\begin{thm} 
\label{17.6}
	Let $(f_n)$ be a sequence of functions, and suppose that each $f_n\colon A \to \bbR$ is continuous. If $(f_n)$ converges uniformly to $f\colon A \to \bbR$, then $f$ is continuous.
\end{thm}
\vspace{4pt}     \hrule   \vspace{4pt}\begin{proof}:\\
Let $\epsilon>0.$ Let $N \in \bbN$ such that if $n\geq N,$ then since $(f_n)$ converges uniformly to $f$, and so $|f_N(x) - f(x)|< \frac{\epsilon}{3}$ for all $x\in A.$ Because each $f_n(x)$ is continuous, then $f_N(x)$ is continuous and so if $x\in A$ and $|x-a|< \delta,$ where $\delta>0,$ then $|f_N(x) - f(a)|< \frac{\epsilon}{3}.$ Therefore, because (1) holds for all $x\in A,$ then $|f_N(a) - f(a)|< \frac{\epsilon}{3}.$ Therefore for all $a\in A,$ if $|x-a|< \delta,$ then:
\begin{align*}
|f(x) - f(a)|&\leq |f(x) - f_N(x)| + |f_N(x) - f(a)|\\
&\leq |f(x) - f_N(x)| + |f_N(x) - f_N(a)| + |f_N(a) - f(a)|\\
&< \frac{\epsilon}{3} + \frac{\epsilon}{3} + \frac{\epsilon}{3}\\
&= \epsilon
\end{align*}
\end{proof}\vspace{4pt}     \hrule   \vspace{4pt}

\section*{Theorem 17.7}
\begin{thm}
	\label{17.7}
	Suppose that $(f_n)$ is a sequence of integrable functions on $[a, b]$ and suppose that $(f_n)$ converges uniformly to $f\colon [a, b] \to \bbR$. Then
	\[
		\int_{a}^{b} f = \lim_{n \to \infty} \int_{a}^{b} f_n.
	\]
\end{thm}
\vspace{4pt}     \hrule   \vspace{4pt}\begin{proof}:\\
Let $\epsilon>0.$ Because $(f_n)$ converges uniformly to $f$, then there exists some $N \in \bbN$ such that for all $n\geq N,$ $|f_n(x) - f(x)|< \frac{\epsilon}{4(b-a)}$ for all $x\in A.$\footnote{The for all $x\in A$ is important here, as it allows us to look at the behavior of $f(x)$ when $x$ is restricted to the subintervals of a partition.} Because each $f_n$ is integrable, then then there exists a partition $P$ such that $U(f_N, P) - L(f_N, P)< \frac{\epsilon}{2}.$ Note that for all $x\in [t_{i-1}, t_{i}],$ $f(x) \leq f_N(x) + \frac{\epsilon}{4(b-a)}\leq M_i(f_N) +\frac{\epsilon}{4(b-a)},$ and thus. $M_i(f)\leq M_i(f_N) +\frac{\epsilon}{4(b-a)}.$ Similarly, $m_i(f)\geq m_i(f_N) - \frac{\epsilon}{4(b-a)}$ for any $i \in [n].$ Therefore, $M_i(f) - m_i(f)\leq M_i(f_N) - m_i(f_N) + 2(\frac{\epsilon}{4(b-a)}).$ Therefore:
\begin{align*}
U(f,P) - L(f,P) &\leq \displaystyle\sum_{i=1}^n(M_i(f_N) - m_i(f_N) + (\frac{\epsilon}{(b-a)}))(t_i-t_{i-1})\\
&= U(f_N, P) - L(f_N,P) + \frac{\epsilon}{2(b-a)}(b-a)\\
&< \frac{\epsilon}{2} + \frac{\epsilon}{2}\\
&= \epsilon
\end{align*}
Therefore, $f$ is integrable. Moreover, for all $\epsilon>0,$ there exists an $N \in \bbN$ such that if $n\geq N,$ then since $|f_n(x) - f(x)|\leq \frac{\epsilon}{2(b-a)}$ for all $x\in A.$ Thus, by Theorem 13.28:
\[|\int_a^bf_n - \int_a^bf| = |\int_a^bf_n -f|\leq \int_a^b|f_n -f|\leq \frac{\epsilon}{2(b-a)}(b-a) = \frac{\epsilon}{2}< \epsilon\]
Therefore, \[\int_a^bf = \lim_{n\to \infty}\int_a^b f_n\]
\end{proof}\vspace{4pt}     \hrule   \vspace{4pt}


\section*{Theorem 17.8}
\begin{thm}\label{17.8}
	Let $(f_n)$ be a sequence of functions defined on an open interval containing $[a, b]$ such that each $f_n$ is differentiable on $[a,b]$ and $f'_n$ is integrable on $[a, b]$. Suppose further that $(f_n)$ converges pointwise to $f$ on $[a,b]$ and that $(f'_n)$ converges uniformly to a continuous function $g$ on $[a,b].$ Then $f$ is differentiable at every $x\in [a,b],$ and
	\[
		f'(x) = \lim_{n \to \infty} f'_n(x).
	\]
\end{thm}
\vspace{4pt}     \hrule   \vspace{4pt}\begin{proof}:\\
By Theorem 17.7, because $(f_n')$ are all integrable and uniformly converge to some continuous function $g$ on $[a,b],$ then $g$ is integrable and \[\int_a^xg = \displaystyle\lim_{n\to \infty}\int_a^xf_n'\] By the Second Fundamental Theorem of Calculus, because $f_n'$ is integrable for all $n$ and there exists functions $f_n$ such that $f_n' = f_n',$ then \[\displaystyle\lim_{n\to \infty}\int_a^xf_n' = \lim_{n\to \infty}(f_n(x) - f_n(a)).\] Because $(f_n)$ converges pointwise to $f,$ then by Theorem 15.8, \[\displaystyle\lim_{n\to \infty}(f_n(x) - f_n(a)) = f(x) - f(a).\] Define $F(x) := \int_a^xg.$ By the work above, this implies that $F(x) = f(x) - f(a).$ Because $g$ is integrable on $[a,b]$, then by Theorem 13.29, $F(x)$ is continuous on $[a,b].$ Therefore, by the First Fundamental Theorem of Calculus, because $g$ is continuous on $(a,b),$ then $F$ is differentiable on $[a,b]$\footnote{I am misusing notation here, since it is technically only differentiable on $(a,b).$} and $F'(x) = g(x) = f'(x) - f'(a)$ for all $x\in [a,b].$ Therefore, because $f'(a) = 0$ since it is a constant, then $f'(x) = g(x).$
\end{proof}\vspace{4pt}     \hrule   \vspace{4pt}

\section*{Theorem 17.9}
\begin{thm}
\label{17.9}
	Let $(f_n)$ be a sequence of functions defined on a set $A$. Then the following are equivalent: 
	\begin{enumerate}[(a)]
		\item There is some function $f$ such that $(f_n)$ converges uniformly to $f$ on $A$.
		\item For all $\epsilon > 0$, there exists an $N \in \bbN$ such that when $m, n \geq N$,
		\[
			\abs{f_n(x) - f_m(x)} < \epsilon, \ \text{for every } x \in A.
		\]

	\end{enumerate}
\end{thm}
\vspace{4pt}     \hrule   \vspace{4pt}\begin{proof}:\\
Look at the footnote for $b\to a$ for a more rigorous proof, but if intuition is all you need, then the one on the document suffices.
\begin{itemize}
    \item ($a\to b$): Because $(f_n)$ converges uniformly to $f,$ then there exists some $N \in \bbN$ such that if $n>m\geq N,$ then $|f_n(x) - f(a)|< \frac{\epsilon}{2}$ and $|f_m(x) - f(a)|< \frac{\epsilon}{2}.$ Thus:
    \begin{align*}
        |f_n(x) - f_m(x)|&\leq |f_n(x) - f(a)| + |f_m(x) - f(a)|\\
        &< \frac{\epsilon}{2} + \frac{\epsilon}{2}\\
        &= \epsilon
    \end{align*}
    \item ($b\to a$): Because $(f_n)$ is Cauchy, then it converges pointwize. Thus, define $f(x):= \displaystyle\lim_{n\to \infty}(f_n(x)).$ Therefore, for all $x\in A,$ there exists an $N' \in \bbN$ such that if $m\geq N',$ then $|f_m(x) - f(a)|< \frac{\epsilon}{2}.$ Moreover, there exists an $N'' \in \bbN$ such that if $n>m\geq N'',$ then $|f_n(x) - f_m(x)|< \frac{\epsilon}{2}$ for all $x\in A.$ Thus, let $N = \max(N',N''),$ and so if $n>m\geq N,$ then for all $x\in A$\footnote{This method is almost naive since it has a problem of the for all $x\in A$ part, as $N'$ is very much $x-$ dependent. However, a fix would be to evaluate $|f_m(x) - f_n(a)|<\frac{\epsilon}{2}$ in the $m-$  limit. By Theorem 15.8, we can distribute this limit into the absolute value (seeing as how absolute value is continuous), and so because $\displaystyle\lim_{m\to \infty}f_m(x) = f(x),$ then $\displaystyle\lim_{m\to \infty}|f_m(x) - f_n(a)|< \displaystyle\lim_{m\to \infty}\epsilon \implies |f(x) - f_n(a)| \leq \frac{\epsilon}{2}<\epsilon.$}: 
    \begin{align*}
        |f_n(x) - f(a)|&\leq |f_n(x) - f_m(a)| + |f_m(a) - f(a)|\\
        &\leq \frac{\epsilon}{2} + \frac{\epsilon}{2}\\
        &= \epsilon
    \end{align*}
    
\end{itemize}
\end{proof}\vspace{4pt}     \hrule   \vspace{4pt}

\section*{Definition 17.10: Series of Functions}
\begin{definition}
	We define series of functions the same way we defined series of numbers. That is, given a sequence $(f_n)$, define the sequence of partial sums $(p_n)$ by
	\[
		p_n(x) = f_1(x) + \dots + f_n(x)
	\]
	and say that $\sum\limits_{n = 1}^{\infty} f_n$ converges pointwise or converges uniformly to $f$ if the sequence $(p_n)$ does.
\end{definition}

\section*{Theorem 17.11: Weierstrass - M Test}
\begin{thm}\label{17.11}
	Suppose that $f_n\colon A \to \bbR$ is a sequence of functions and that there exists a sequence of positive real numbers $(M_n)$ such that for all $x \in A$, we have $\abs{f_n(x)} \leq M_n$.
	If $\sum\limits_{n = 1}^{\infty} M_n$ converges, then for each $x \in A$, the series of numbers $\sum\limits_{n = 1}^{\infty} f_n(x)$ converges absolutely. Furthermore, $\sum\limits_{n = 1}^{\infty} f_n$ converges uniformly to the function $f$ defined by:
	\[
		f(x) = \sum_{n = 1}^{\infty} f_n(x).
	\]
\end{thm}
\vspace{4pt}     \hrule   \vspace{4pt}\begin{proof}
    Because for all $x\in A,$ $|f_n(x)|\leq M_n,$ then by the Convergence Test, because $\displaystyle\sum_{n=1}^\infty M_n$ converges, then $\displaystyle\sum_{i=1}^\infty |f_n(x)|$ converges. Therefore, for all $x\in A,$ the series converges absolutely. Therefore, by Theorem 16.5, if $k>m \geq N,$ then $|\displaystyle\sum_{k=m+1}^n M_n|< \epsilon.$ Thus, because $|f_n(x)|\leq M_n,$ then \[\displaystyle\sum_{k=m+1}^n |f_n(x)|\leq  |\displaystyle\sum_{k=m+1}^n M_n| < \epsilon\] for all $x\in A.$ Thus, by Theorem 16.11, $|\displaystyle\sum_{k=m+1}^n f_k(x)|\leq \displaystyle\sum_{k=m+1}^n |f_k(x)| < \epsilon,$ and so $|\displaystyle\sum_{k=1}^nf_k(x) - \displaystyle\sum_{k=1}^mf_k(x)|<\epsilon$ for all $x\in A.$ Thus, by Theorem \ref{17.9}, $\displaystyle\sum_{k=1}^nf_k(x)$ converges uniformly and thus, by definition, $\displaystyle\sum_{n=1}^\infty f_n(x)$ converges uniformly and so define $f(x):=\displaystyle\sum_{n=1}^\infty f_n(x)$ 
\end{proof}\vspace{4pt}     \hrule   \vspace{4pt}

The previous theorem raises the question of what functions defined as series look like.

\section*{Power Series}

\begin{defn}
	A function of the form
	\[
		f(x) = \sum_{n = 0}^{\infty} c_n (x - a)^n, \qquad c_n \in \bbR
	\]
	is called a \emph{power series}. The power series is \emph{centered} at $a$ and the numbers $c_n$ are called the coefficients.
\end{defn} 

The next theorem gives a way to determine where a given power series is well-defined.

\begin{thm}
	Let 
	\[
		f(x) = \sum_{n = 0}^{\infty} c_n x^n
	\]
	be a power series centered at $0$. Suppose that $x_0$ is a real number such that the series
	\[
		f(x_0) = \sum_{n = 0}^{\infty} c_n x_{0}^n
	\]
	converges. Let $r$ be any number such that $0 < r < \abs{x_0}$. Then the following series of functions converge uniformly on $[-r, r]$ (and absolutely for each $x \in [-r, r]$):
	\begin{eqnarray*}
		f(x) & = &  \sum_{n = 0}^{\infty} c_n x^n\\
		g(x) & = &  \sum_{n = 1}^{\infty} n c_n x^{n - 1}\\
		\qquad \text{and} \qquad h(x) & = & \sum_{n=0}^\infty c_n\frac{x^{n+1}}{n+1}
	\end{eqnarray*}
	
	
	Furthermore, $f$ is differentiable on $[-r, r] $ and $f' = g$. Also $h$ is differentiable on $[-r,r]$ and $h'=f.$ 
	\hint[Hints]{(i) There is some $M$ such that $\abs{c_n x_0^n} \leq M,$ for all $n$. (ii) If $x \in [-r, r]$, $\abs{c_n x^n} \leq M \left(\frac{r}{\abs{x_0}}\right)^n, $ for all $ n$. (iii) Use Theorem~\ref{uniformconvseries}.} 
\end{thm}

We may paraphrase this theorem as follows: if a (zero-centered) power series converges at $x_0,$ then it may be differentiated and anti-differentiated term-by-term on $(-|x_0|,|x_0|)$ to obtain power series representations of the derivative and antiderivative of $f.$  

\begin{cor}
If $f(x)=\displaystyle \sum_{n=0}^\infty c_n x^n$ converges for all $x$ in some interval $(-R,R),$ then $f'(x)=\displaystyle\sum_{n=1}^\infty nc_n x^{n-1},$ for all $x\in (-R,R).$ 
\end{cor} 


\begin{exmp}
Let $p_n\colon\bbR\to\bbR$ be given by $p_n(x)=c_0+c_1 x+c_2x^2 +\cdots c_n x^n.$  Prove that, for all $k=0,\cdots, n,$

$$c_k = \frac{p_n^{(k) } (0)}{k!}$$ and deduce that, for all $x\in \bbR,$ 

$$p_n(x) = \sum_{k=0}^n \frac{p_n^{(k) } (0)}{k!}x^k.$$ 

\end{exmp}

\begin{defn} 
Suppose that $f:A\to\bbR$ has derivatives of all orders at $a\in A.$ 
Then we define $$T_{f,a}(x)= \sum_{k=0}^{\infty} \frac{f^{(k)} (a)}{k!} (x-a)^k.$$

\end{defn} 


\begin{exmp}\label{TSex}
\begin{enumerate}
\item[a)] Find $T_{f,0}$ for each of the following functions:
\begin{enumerate}
\item[i)] $f:\bbR\to\bbR$ given by $f(x)=x^3-3x^2+1$ 
\item[ii)] $f:\bbR\setminus\{1\}\to \bbR$ given by $f(x) = \frac{1}{1-x}$ 
\end{enumerate}
\item[b)] For each of the examples in a), based on prior knowledge, for which values of $x$ does $T_{f,0}(x)=f(x)$?  

\end{enumerate}
\end{exmp}




\begin{thm} \label{thm1}
Let $A\subset \bbR$ be an open interval containing $a$. Let $f\colon A\to\bbR$ be $(n+1)-$times differentiable (so $f',f'',f''',\cdots, f^{(n+1)}$ all exist) and let $f^{(n+1)}$ be continuous. Let $x\in A.$  Then
$$f(x)=p_n(x)+R_n(x) $$
where
$$p_n(x)=\sum_{i=0}^n \frac{(x-a)^i}{i!} f^{(i)} (a)$$
and

$$R_n(x)=\int_{a}^x f^{(n+1)}(t)\frac{(x-t)^n}{n!} dt.$$
\end{thm}


\begin{prop}\label{prop2}  With the same notation as in Theorem~\ref{thm1},

$$R_n(x) = \frac{f^{(n+1)} (c)}{(n+1)!} (x-a)^{n+1},$$
for some $c$ between $x$ and $a.$
\end{prop}



\begin{exmp}


Let $A\subset \bbR$ be an open set, $f:A\to\bbR$ be $(n+1)-$times differentiable (so $f',f'',f''',\cdots, f^{(n+1)}$ all exist) and let $f^{(n+1)}$ be continuous. Let $a\in A.$ 
Suppose that $f'(a)=f''(a)=\cdots f^{(n)}(a)=0$ but $f^{(n+1)}(a)\neq 0.$ Prove that 
\begin{enumerate}
\item[a)] If $n$ is odd and $f^{(n+1)}(a)>0$ then $f$ has a local minimum at $a.$
\item[b)] If $n$ is odd and $f^{(n+1)}(a)<0$ then $f$ has a local maximum at $a.$ 
\item[c)] What happens if $n$ is even?
\end{enumerate}
\end{exmp}

\begin{prop}

Let $A\subset \bbR$ be an open interval containing $a$. Suppose that $f:A\to\bbR$ has derivatives of all orders at $a\in A.$ Let $x\in A.$  
Then 
$f(x)=\displaystyle \sum_{n=0}^\infty \frac{f^{k}(a)}{k!} (x-a)^k$ if, and only if, $\displaystyle \lim_{n\to\infty} R_n(x) = 0.$ 
\end{prop} 


\begin{exmp}
 Using the remainder formula in either Theorem~\ref{thm1} or Proposition~\ref{prop2}, prove directly that $\lim_{n\to\infty} R_n(x)=0$ for each of the functions in Exercise~\ref{TSex}, for the range of $x-$values identified in Exercise~\ref{TSex}b).

 \end{exmp} 








 



\begin{exmp}
Suppose that $f$ is given by a power series centered at $a$. So, for some $R,$ and coefficients $c_n,$
$$f(x) = \sum_{n=0}^\infty c_n (x-a)^n,\quad \text{ for } |x-a|<R.$$

Then
 $T_{f,a}(x)=f(x),$ for all $x$ with $|x-a|<R.$ 


\end{exmp} 
\bigskip




\begin{center}
{\em Additional Exercises}
\end{center}


We sometimes use Leibniz notation for derivatives:
\[
\frac d{dx} f(x) = f'(x).
\]
This allows us to differentiate functions that haven't been named. 



\begin{enumerate}



\item

Let $n!$ (read as ``$n$ factorial'') denote the product of all the natural numbers between $1$ and $n$ (inclusive).
Thus $n!=n\,(n-1)!$, and we define $0!=1$ so that this holds for all $n\ge0$.

Define the following functions by power series:
\[
\cos x=\sum_{n=0}^\infty (-1)^n\frac{x^{2n}}{(2n)!},\qquad
\sin x=\sum_{n=0}^\infty (-1)^n\frac{x^{2n+1}}{(2n+1)!}.
\]

Note that we interpret $0^0$ as 1 in these formulae.

\begin{enumerate}

\item
Show that these series converge pointwise for all real $x$ and uniformly on $[a,b],$ for any $a,b\in \bbR.$ 



\item
Show that  $\sin'=\cos$ and $\cos'=-\sin$.
Also find $\sin 0$, $\cos 0$.
\end{enumerate}


\item 

Define the function $\displaystyle E(x)=\sum_{n=0}^\infty \frac{x^n}{n!}.$ (Note that we interpret $0^0=1$ in this formula.)
\begin{enumerate}
\item[a)] Show that the series  $E(x)$ converges pointwise for all real $x$ and uniformly on $[a,b],$ for any $a,b\in \bbR.$ 
\item[b)] Prove that $E'(x)=E(x),$ for all $x\in \bbR,$ and $E(0)=1.$
\item[c)] Prove that the function $E$ satisfying the properties in b) is unique. {\em Hint: Proposition~\ref{prop2} should be helpful here}
\item[d)] 

\begin{enumerate} 
\item[i)] Prove that $\frac{d}{dx} (E(x)E(-x)) =0.$ Deduce that $E(-x)E(x)=1,$ for all $x\in\bbR.$ 
\item[ii)] Prove that $E(x)\neq 0,$ for all $x\in \bbR.$
\item[iii)]
Prove that $E(x)E(y)=E(x+y)$ for all $x,y\in\bbR$.

\item[iiv)]
Prove that $E(r)=E(1)^r,$ for all rationals $r.$
\end{enumerate} 
\end{enumerate}

 















\item

\begin{enumerate}

\item
Let $\exp (x)=E(x)$ where $E$ is as defined in the previous exercise. Prove that $\exp$ has an inverse function (call it $\log$) and find its domain and range.


\item
Show that $\log'(x)=1/x$. (Recall Theorem 11.23 concerning derivatives of inverse functions.)

\item
Recall that 
\[
\frac1{1-x}=\sum_{n=0}^\infty x^n,\qquad|x|<1.
\]
Use this series to find a power series  representation for $f(x)=\log(1-x)$, $|x|<1$. Be sure to justify your answer.


\end{enumerate}


\item 
Consider the sequence $(f_n)$ of functions $f_n \colon \mathbb{R} \to \mathbb{R}$ given by
$$f_n (x) = \frac{x}{1+nx^2}.$$
\begin{enumerate}
\item[a)] Find the pointwise limit $f$ of $(f_n)$.
\vspace{4pt}    \hrule   \vspace{4pt}\begin{proof}:\\
For all $\epsilon>0:$
\begin{enumerate}
    \item If $x \neq 0,$ then there exists an $N\in \bbN$ such that $\frac{1}{\epsilon}< \frac{1}{x} + Nx.$ Thus, if $n\geq N,$ then:
\begin{align*}
    |f_n(x) - 0| &= |f_n(x)|\\
    &= |\frac{x}{1+nx^2}|\\
    &=|\frac{1}{\frac{1}{x} + nx}|\\
    &< \epsilon
\end{align*}
And so $(f_n)(x)$ converges pointwise to $0.$
\item If $x = 0,$ then for any $n\in \bbN,$ because $f_n(0) = 0$ and  $\displaystyle\lim_{n\to \infty}(0) = 0,$ then $(f_n)(x)$ converges pointwise to $0.$
\end{enumerate}
\end{proof}\vspace{4pt}    \hrule   \vspace{4pt}
\item[b)] Show that the convergence $f_n \to f$ is uniform.
\vspace{4pt}    \hrule   \vspace{4pt}\begin{proof}:\\
By Professor Propp's hint, we were able to find the maximum and minimum of $(f_n)(x)$ using a graphical method. There exists an $N\in \bbN$ such that then $\frac{1}{\epsilon}<2\sqrt{N}\leq 2\sqrt{n}\leq |\frac{1}{x} + nx|$ for any $n\geq N,$ $x\in \bbR.$ Thus, if $n\geq N,$ then \footnote{If $x = 0,$ then we could still use the same $N,$ but it's reduced to the trivial case of $f_n(0) =0.$} Therefore:
\begin{align*}
    |f_n(x) -0|&= |f_n(x)|\\
    &= |\frac{x}{1+nx^2}|\\
    &= |\frac{1}{\frac{1}{x}+nx}|\\
    &\leq \frac{1}{2\sqrt{N}}\\
    &<\epsilon.
\end{align*}
While one could reach the fact that $2\sqrt{N}$ is the magical bounds for any epsilon tube around $f = 0$ by a longer route using differentiation, but as stated, graphing is sufficient.
\end{proof}\vspace{4pt}    \hrule   \vspace{4pt}
\item[c)] Compute $f_n'$ for each $n$.
\vspace{4pt}    \hrule   \vspace{4pt}\begin{proof}:\\
Define $f(x)= x$ and $g(x) = 1+nx^2.$ Note that both $f$ and $g$ are polynomials and so differentiable for all $x\in \bbR.$ Moreover, note that $f'(x)= 1$ and $g'(x) = 2xn.$ Thus, by Example 12.9, \[(\frac{f}{g})'(x) = \frac{f'(x)g(x) - g'(x)f(x)}{f^2(x)} = \frac{1+nx^2 - 2x^2n}{(1+nx^2)^2} = \frac{1-nx^2}{(1+nx^2)^2}.\]
\end{proof}\vspace{4pt}    \hrule   \vspace{4pt}
\item[d)]  Find the pointwise limit $g$ of $(f_n')$. Is the convergence uniform?
\vspace{4pt}    \hrule   \vspace{4pt}\begin{proof}:\\
    For all $\epsilon>0,$ 
    \begin{enumerate}
        \item If $x \neq 0,$ then there exists an $N\in \bbN$ such that $\frac{1}{1+Nx^2}<\epsilon$\footnote{By manipulation of the Archimidean property, this holds true} Thus, if $n\geq N,$ then $\frac{1}{1+nx^2}< \epsilon$ and:
        \begin{align*}
            |f_n' - 0| &= |f_n'|\\
            &= |\frac{1-nx^2}{(1+nx^2)(1+nx^2)}|\\
            &\leq |\frac{1-nx^2}{1+2nx^2+n^2x^4}| \\
            &\leq |\frac{1-nx^2}{1-n^2x^4}|\\
            &= |\frac{1-nx^2}{(1-nx^2)(1+nx^2)}|\\
            &= |\frac{1}{1+nx^2}|\\
            &< \epsilon
        \end{align*}
        \item If $x = 0,$ then evidently for any $N\in \bbN,$ we have that $\displaystyle\lim_{n\to \infty}(1) = 1.$
    \end{enumerate}
    Therefore, $f'_n$ converges pointwise to \[g(x) = \begin{cases}
    0 \qquad x\neq 0\\
    1 \qquad x = 0
    \end{cases}\]
    Because $g$ is obviously not continuous but every $f_n'$ is continuous, then by Theorem 17.6, $(f_n')$ does not converge uniformly to $g.$
\end{proof}\vspace{4pt}    \hrule   \vspace{4pt}
\item[e)] Does $f'$ agree with $g$?
\vspace{4pt}    \hrule   \vspace{4pt}\begin{proof}:\\
Consider that because $f(x) = 0,$ then $f'(x) =0.$ Therefore, $g(x)$ does not agree with $f'(x)$ on $x=0,$ and $g(0)=1.$
\end{proof}\vspace{4pt}    \hrule   \vspace{4pt}
\end{enumerate}

{\em Compare this example with Theorem 17.8}
\vspace{4pt}    \hrule   \vspace{4pt}\begin{proof}:\\
In this example, $f_n$ was a sequence of functions such that $f_n'$ were continuous on any closed interval (by 11.12) and thus integrable. Moreover, $f_n$ converged pointwise to $f$ BUT $f'_n$ does not converge uniformly to $g$ on $[a,b].$ This is the reason why Theorem 17.8 fails, and why $\displaystyle\lim_{n\to \infty}f_n'(x)\neq f'(x).$
\end{proof}\vspace{4pt}    \hrule   \vspace{4pt}






\item
Let $f_n(x) = \frac{\sin(x)}{n}-\frac{\sin(x)}{n+2}$ and $s_n(x) = \sum\limits_{k=1}^n f_k(x)$. Find the pointwise limit of $(s_n)$ and determine if the convergence is uniform.



\end{enumerate}

\section*{Acknowledgments} 
Thanks, as always, to Professor Oron Propp for being a great mentor in both Office Hours and during class. Thank you to Richard Gale for showing me a smart way of doing 13.4 (I included both his (first one) and my proof (second)). Thanks also to Lina Piao for working with me to figure out a couple of proofs, such as 13.19, 13.20, and 13.29.
\begin{thebibliography}{9}




\end{thebibliography}

\end{document}

